\documentclass{ximera}

%\usepackage{todonotes}

\newcommand{\todo}{}


\graphicspath{
{./}
{../functionsOfSeveralVariables/}
{../normalVectors/}
{../lagrangeMultipliers/}
}


\usepackage{tkz-euclide}
\tikzset{>=stealth} %% cool arrow head
\tikzset{shorten <>/.style={ shorten >=#1, shorten <=#1 } } %% allows shorter vectors

\usetikzlibrary{backgrounds} %% for boxes around graphs
\usetikzlibrary{shapes,positioning}  %% Clouds and stars
\usetikzlibrary{matrix} %% for matrix
\usepgfplotslibrary{polar} %% for polar plots
\usetkzobj{all}
\usepackage[makeroom]{cancel} %% for strike outs
%\usepackage{mathtools} %% for pretty underbrace % Breaks Ximera
\usepackage{multicol}





\usepackage{array}
\setlength{\extrarowheight}{+.1cm}   
\newdimen\digitwidth
\settowidth\digitwidth{9}
\def\divrule#1#2{
\noalign{\moveright#1\digitwidth
\vbox{\hrule width#2\digitwidth}}}





\newcommand{\RR}{\mathbb R}
\newcommand{\R}{\mathbb R}
\newcommand{\N}{\mathbb N}
\newcommand{\Z}{\mathbb Z}

\newcommand{\sage}{\textsf{SageMath}}


%\renewcommand{\d}{\,d\!}
\renewcommand{\d}{\mathop{}\!d}
\newcommand{\dd}[2][]{\frac{\d #1}{\d #2}}
\newcommand{\pp}[2][]{\frac{\partial #1}{\partial #2}}
\renewcommand{\l}{\ell}
\newcommand{\ddx}{\frac{d}{\d x}}

\newcommand{\zeroOverZero}{\ensuremath{\boldsymbol{\tfrac{0}{0}}}}
\newcommand{\inftyOverInfty}{\ensuremath{\boldsymbol{\tfrac{\infty}{\infty}}}}
\newcommand{\zeroOverInfty}{\ensuremath{\boldsymbol{\tfrac{0}{\infty}}}}
\newcommand{\zeroTimesInfty}{\ensuremath{\small\boldsymbol{0\cdot \infty}}}
\newcommand{\inftyMinusInfty}{\ensuremath{\small\boldsymbol{\infty - \infty}}}
\newcommand{\oneToInfty}{\ensuremath{\boldsymbol{1^\infty}}}
\newcommand{\zeroToZero}{\ensuremath{\boldsymbol{0^0}}}
\newcommand{\inftyToZero}{\ensuremath{\boldsymbol{\infty^0}}}



\newcommand{\numOverZero}{\ensuremath{\boldsymbol{\tfrac{\#}{0}}}}
\newcommand{\dfn}{\textbf}
%\newcommand{\unit}{\,\mathrm}
\newcommand{\unit}{\mathop{}\!\mathrm}
\newcommand{\eval}[1]{\bigg[ #1 \bigg]}
\newcommand{\seq}[1]{\left( #1 \right)}
\renewcommand{\epsilon}{\varepsilon}
\renewcommand{\iff}{\Leftrightarrow}

\DeclareMathOperator{\arccot}{arccot}
\DeclareMathOperator{\arcsec}{arcsec}
\DeclareMathOperator{\arccsc}{arccsc}
\DeclareMathOperator{\si}{Si}
\DeclareMathOperator{\proj}{\vec{proj}}
\DeclareMathOperator{\scal}{scal}
\DeclareMathOperator{\sign}{sign}


%% \newcommand{\tightoverset}[2]{% for arrow vec
%%   \mathop{#2}\limits^{\vbox to -.5ex{\kern-0.75ex\hbox{$#1$}\vss}}}
\newcommand{\arrowvec}{\overrightarrow}
%\renewcommand{\vec}[1]{\arrowvec{\mathbf{#1}}}
\renewcommand{\vec}{\mathbf}
\newcommand{\veci}{{\boldsymbol{\hat{\imath}}}}
\newcommand{\vecj}{{\boldsymbol{\hat{\jmath}}}}
\newcommand{\veck}{{\boldsymbol{\hat{k}}}}
\newcommand{\vecl}{\boldsymbol{\l}}
\newcommand{\utan}{\mathbf{\hat{t}}}
\newcommand{\unormal}{\mathbf{\hat{n}}}
\newcommand{\ubinormal}{\mathbf{\hat{b}}}

\newcommand{\dotp}{\bullet}
\newcommand{\cross}{\boldsymbol\times}
\newcommand{\grad}{\boldsymbol\nabla}
\newcommand{\divergence}{\grad\dotp}
\newcommand{\curl}{\grad\cross}
%\DeclareMathOperator{\divergence}{divergence}
%\DeclareMathOperator{\curl}[1]{\grad\cross #1}
\newcommand{\lto}{\mathop{\longrightarrow\,}\limits}


\colorlet{textColor}{black} 
\colorlet{background}{white}
\colorlet{penColor}{blue!50!black} % Color of a curve in a plot
\colorlet{penColor2}{red!50!black}% Color of a curve in a plot
\colorlet{penColor3}{red!50!blue} % Color of a curve in a plot
\colorlet{penColor4}{green!50!black} % Color of a curve in a plot
\colorlet{penColor5}{orange!80!black} % Color of a curve in a plot
\colorlet{fill1}{penColor!20} % Color of fill in a plot
\colorlet{fill2}{penColor2!20} % Color of fill in a plot
\colorlet{fillp}{fill1} % Color of positive area
\colorlet{filln}{penColor2!20} % Color of negative area
\colorlet{fill3}{penColor3!20} % Fill
\colorlet{fill4}{penColor4!20} % Fill
\colorlet{fill5}{penColor5!20} % Fill
\colorlet{gridColor}{gray!50} % Color of grid in a plot

\newcommand{\surfaceColor}{violet}
\newcommand{\surfaceColorTwo}{redyellow}
\newcommand{\sliceColor}{greenyellow}




\pgfmathdeclarefunction{gauss}{2}{% gives gaussian
  \pgfmathparse{1/(#2*sqrt(2*pi))*exp(-((x-#1)^2)/(2*#2^2))}%
}


%%%%%%%%%%%%%
%% Vectors
%%%%%%%%%%%%%

%% Simple horiz vectors
\renewcommand{\vector}[1]{\left\langle #1\right\rangle}


%% %% Complex Horiz Vectors with angle brackets
%% \makeatletter
%% \renewcommand{\vector}[2][ , ]{\left\langle%
%%   \def\nextitem{\def\nextitem{#1}}%
%%   \@for \el:=#2\do{\nextitem\el}\right\rangle%
%% }
%% \makeatother

%% %% Vertical Vectors
%% \def\vector#1{\begin{bmatrix}\vecListA#1,,\end{bmatrix}}
%% \def\vecListA#1,{\if,#1,\else #1\cr \expandafter \vecListA \fi}

%%%%%%%%%%%%%
%% End of vectors
%%%%%%%%%%%%%

%\newcommand{\fullwidth}{}
%\newcommand{\normalwidth}{}



%% makes a snazzy t-chart for evaluating functions
%\newenvironment{tchart}{\rowcolors{2}{}{background!90!textColor}\array}{\endarray}

%%This is to help with formatting on future title pages.
\newenvironment{sectionOutcomes}{}{} 



%% Flowchart stuff
%\tikzstyle{startstop} = [rectangle, rounded corners, minimum width=3cm, minimum height=1cm,text centered, draw=black]
%\tikzstyle{question} = [rectangle, minimum width=3cm, minimum height=1cm, text centered, draw=black]
%\tikzstyle{decision} = [trapezium, trapezium left angle=70, trapezium right angle=110, minimum width=3cm, minimum height=1cm, text centered, draw=black]
%\tikzstyle{question} = [rectangle, rounded corners, minimum width=3cm, minimum height=1cm,text centered, draw=black]
%\tikzstyle{process} = [rectangle, minimum width=3cm, minimum height=1cm, text centered, draw=black]
%\tikzstyle{decision} = [trapezium, trapezium left angle=70, trapezium right angle=110, minimum width=3cm, minimum height=1cm, text centered, draw=black]


\outcome{Work with double integrals in rectangular coordinates.}
\outcome{Work with double integrals in polar coordinates.}
\outcome{Work with triple integrals in spherical cooridnates.}

\title[Dig-In:]{Computations and interpretations}

\begin{document}
\begin{abstract}
  We practice more computations and think about what integrals mean.
\end{abstract}
\maketitle

In this section we will continue to set-up (and sometimes compute)
double and triple integrals and think about what these mean.

\section{Triangles}

\begin{example}
  Let $T$ be the triangle with vertices $(0,0)$ and $(\pi,0)$ and
  $(\pi,\pi)$.  Then we can write $T$ as
  \[
  T = \{ (x,y)  : 0 \leq y \leq \pi \text{ and } \answer[given]{y} \leq x \leq \answer[given]{\pi} \},
  \]
  but we could also write
  \[
    T = \{ (x,y)  : 0 \leq x \leq \pi \text{ and } \answer[given]{0} \leq y \leq \answer[given]{x} \}.
  \]
  Which of these might be a better choice to compute
  \[
  \int_T 2 \sin(x^2) \d A?
  \]
  \begin{explanation}
    Based on these two different descriptions of $T$, we can evaluate
    the double integral $\int_T 2 \sin (x^2) \d A$ through two rather
    different looking iterated integrals.  Write with me,
    \[
    \int_T 2 \sin (x^2) \d A =\begin{cases}
    &= \int_{\answer{0}}^{\answer{\pi}} \int_{y}^\pi 2 \sin (x^2) \d x \d y \\
    &= \int_{0}^\pi \int_{\answer{0}}^{\answer{x}} 2 \sin (x^2) \d y \d x.
    \end{cases}
    \]
    If we were to use the first description of $T$, we might have
    trouble finding an $x$-antiderivative of $\sin (x^2)$, so let's
    try the second description.  In that case,
    \begin{align*}
    \int_T 2 \sin (x^2) \d A &= \int_{0}^\pi 2 x \sin (x^2) \d x\\
    &= \answer[given]{1 - \cos (\pi^2)}.
    \end{align*}
  \end{explanation}
\end{example}

\begin{question}
  It's important to do a self-check to see if our purported value for
  an integral is at all plausible.

  The region $T$ is a triangle with base $\answer{\pi}$ and height
  $\answer{\pi}$, so the area of the region $T$ is $\answer{\pi^2/2}$
  which is about $5$ square units.  In other words,
  \[
  \int_T 1 \d A = \answer{\pi^2/2}
  \]
  which also means that
  \[
  \int_T 2 \d A = \answer{\pi^2}.
  \]
  We are claiming that $\int_T 2 \sin (x^2) \d A$ equals
  $1 - \cos (\pi^2)$, which is about $1.9$.

  When $0 \leq x \leq \pi$, the value of $\sin (x^2)$ is sometimes
  positive, sometimes negative, but at least we know that
  \[
  \answer{-2} \leq 2 \sin (x^2) \leq \answer{2},
  \]
  and this inequality then implies that
  \[
  \left| \int_T 2 \sin (x^2) \d A \right| \leq \left| \int_T 2 \d A \right| \answer{\pi^2}.
  \]
  So $1.9$ is certainly in the ballpark of plausibility.
\end{question}

\section{Polar coordinates}

\begin{example}
  Evaluate the integral $\int_Q \left(x + y\right) \d A$ where $Q$ is the quarter circle,
  \[
  Q = \{ (x,y)  : \text{$x \geq 0$, $y \geq 0$, $x^2 + y^2 \leq 1$} \}.
  \]
  
  \begin{explanation}
    We use \wordChoice{\choice{rectangular}\choice[correct]{polar}} coordinates.
    \begin{align*}
      \int_Q \left(x + y\right) \d A
      &= \int_{\answer[given]{0}}^{\answer[given]{\pi/2}} \int_{0}^{1} (r \cos \theta + r \sin \theta) r \d r \d \theta \\
      &= \int_{\answer[given]{0}}^{\answer[given]{\pi/2}} \eval{\answer[given]{
          (r^3/3) \cos \theta + (r^3/3) \sin \theta
      }}_0^1\d \theta \\  
      &= \int_{\answer[given]{0}}^{\answer[given]{\pi/2}}\answer[given]{\frac{(\sin \theta + \cos \theta)}{3}}\d\theta
      &= \eval\answer[given]{\frac{\sin \theta - \cos \theta}{3}}}_{0}^{\pi/2} \\
      &= \answer[given]{2/3}.
    \end{align*}
  \end{explanation}
\end{example}

\begin{question}
  Again consider the region
  \[
    Q = \{ (x,y)  : x \geq 0,\hspace{1ex} y \geq 0,\hspace{1ex} x^2 + y^2 \leq 1 \}.
  \]
  How does
  \[
    A = \int_Q x \d A  
  \]
  compare to 
  \[
    B = \int_Q y \d A?
  \]

  \begin{multipleChoice}
    \choice{$A < B$}
    \choice[correct]{$A = B$}
    \choice{$A > B$}
  \end{multipleChoice}
  
  \begin{feedback}[correct]
    The region $Q$ is symmetric across the line $x = y$.  As a
    consequence of this, we might have computed that
    \[
    A = \int_Q x \d A = 1/3,
    \]
    and likewise $B = 1/3$.  Because of this, and the fact that the
    integral of a sum is the sum of integrals, we could have deduced
    \begin{align*}
      A + B &= (1/3) + (1/3) \\
      &= \int_Q x \d A + \int_Q y \d A \\
      &= \int_Q (x+y) \d A = 2/3.
    \end{align*}
  \end{feedback}
\end{question}

\section{Spheres and hemispheres}

\begin{example}
  Let $B$ be the region $B = \{ (x,y,z)  : x^2 + y^2 + z^2 \leq 1\}$.
  
  Explain why $\int_B z^3 \d V = \answer[given]{0}$.
  
  \begin{explanation}
    This integral vanishes because integral over the northern
    hemisphere of $B$ will cancel the contribution from the southern
    hemisphere of $B$.
  \end{explanation}
\end{example}

\begin{example}
  Let $H$ be the region
  \[
  H = \{ (x,y,z) : x^2 + y^2 + z^2 \leq 1 \text{ and } z \geq 0 \}.
  \]

  Show that $\int_H z^3 \d V = \answer[given]{\pi/12}$.

  \begin{explanation}
    Unlike the previous example, this does not vanish.
    
    We use \wordChoice{\choice{cylindrical}\choice[correct]{spherical}} coordinates.
    \begin{align*}
      \int_D z^3 \d V
      &= \int_{0}^{2\pi} \int_{0}^{\pi/2} \int_{0}^1 (\rho \cos \varphi)^3 \rho^2 \sin \varphi \d \rho \d \varphi \d \theta \\
      &= \int_{0}^{2\pi} \int_{0}^{\pi/2} \int_{0}^1 \rho^5 \cos^3 \varphi \sin \varphi \d \rho \d \varphi \d \theta \\
      &= \int_{0}^{2\pi} \int_{0}^{\pi/2} \eval{\answer[given]{\frac{\rho^6 \cos^3 \varphi \sin \varphi}{6}}}_0^1 \d \varphi \d \theta \\
      &= \int_{0}^{2\pi} \int_{0}^{\pi/2} \answer[given]{\frac{\cos^3 \varphi \sin \varphi}{6}} \d \varphi \d \theta \\
      &= \int_{0}^{2\pi} \eval{\answer[given]{\frac{-\cos^4 \varphi}{24}}}_{0}^{\pi/2} \d \theta \\
      &= \int_{0}^{2\pi} \answer[given]{\frac{1}{24}} \d \theta \\
      &= \answer[given]{\frac{\pi}{12}}.
    \end{align*}
  \end{explanation}
\end{example}

\begin{question}
  Again let $H$ be the region
  \[
  H = \{ (x,y,z): \text{$x^2 + y^2 +
    z^2 \leq 1$ and $z \geq 0 $}\}.
  \]
  Set $A = \int_H z^3 \d V$ and $B = \int_H z^{10} \d V$.  How does $A$ relate to $B$?

  \begin{multipleChoice}
    \choice{$A < B$}
    \choice{$A = B$}
    \choice[correct]{$A > B$}
  \end{multipleChoice}

  \begin{feedback}[correct]
    Indeed, $A > B$ because, for points $(x,y,z) \in H$, we have
    $-1 \leq z \leq 1$ and $z^3 \geq z^{10}$.  Moreover, except when
    $z = \pm 1$, it is the case that $z^3 > z^{10}$.  By comparing the
    integrands, we can gain insight into the relative sizes of the
    integrals.

    This same kind of thinking can lend insight into the question of
    what happens when $\int_H z^N \d V$ when $N$ is very
    large.
  \end{feedback}
\end{question}

\end{document}
