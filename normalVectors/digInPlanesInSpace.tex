\documentclass{ximera}

%\usepackage{todonotes}

\newcommand{\todo}{}


\graphicspath{
{./}
{../functionsOfSeveralVariables/}
{../normalVectors/}
{../lagrangeMultipliers/}
}


\usepackage{tkz-euclide}
\tikzset{>=stealth} %% cool arrow head
\tikzset{shorten <>/.style={ shorten >=#1, shorten <=#1 } } %% allows shorter vectors

\usetikzlibrary{backgrounds} %% for boxes around graphs
\usetikzlibrary{shapes,positioning}  %% Clouds and stars
\usetikzlibrary{matrix} %% for matrix
\usepgfplotslibrary{polar} %% for polar plots
\usetkzobj{all}
\usepackage[makeroom]{cancel} %% for strike outs
%\usepackage{mathtools} %% for pretty underbrace % Breaks Ximera
\usepackage{multicol}





\usepackage{array}
\setlength{\extrarowheight}{+.1cm}   
\newdimen\digitwidth
\settowidth\digitwidth{9}
\def\divrule#1#2{
\noalign{\moveright#1\digitwidth
\vbox{\hrule width#2\digitwidth}}}





\newcommand{\RR}{\mathbb R}
\newcommand{\R}{\mathbb R}
\newcommand{\N}{\mathbb N}
\newcommand{\Z}{\mathbb Z}

\newcommand{\sage}{\textsf{SageMath}}


%\renewcommand{\d}{\,d\!}
\renewcommand{\d}{\mathop{}\!d}
\newcommand{\dd}[2][]{\frac{\d #1}{\d #2}}
\newcommand{\pp}[2][]{\frac{\partial #1}{\partial #2}}
\renewcommand{\l}{\ell}
\newcommand{\ddx}{\frac{d}{\d x}}

\newcommand{\zeroOverZero}{\ensuremath{\boldsymbol{\tfrac{0}{0}}}}
\newcommand{\inftyOverInfty}{\ensuremath{\boldsymbol{\tfrac{\infty}{\infty}}}}
\newcommand{\zeroOverInfty}{\ensuremath{\boldsymbol{\tfrac{0}{\infty}}}}
\newcommand{\zeroTimesInfty}{\ensuremath{\small\boldsymbol{0\cdot \infty}}}
\newcommand{\inftyMinusInfty}{\ensuremath{\small\boldsymbol{\infty - \infty}}}
\newcommand{\oneToInfty}{\ensuremath{\boldsymbol{1^\infty}}}
\newcommand{\zeroToZero}{\ensuremath{\boldsymbol{0^0}}}
\newcommand{\inftyToZero}{\ensuremath{\boldsymbol{\infty^0}}}



\newcommand{\numOverZero}{\ensuremath{\boldsymbol{\tfrac{\#}{0}}}}
\newcommand{\dfn}{\textbf}
%\newcommand{\unit}{\,\mathrm}
\newcommand{\unit}{\mathop{}\!\mathrm}
\newcommand{\eval}[1]{\bigg[ #1 \bigg]}
\newcommand{\seq}[1]{\left( #1 \right)}
\renewcommand{\epsilon}{\varepsilon}
\renewcommand{\iff}{\Leftrightarrow}

\DeclareMathOperator{\arccot}{arccot}
\DeclareMathOperator{\arcsec}{arcsec}
\DeclareMathOperator{\arccsc}{arccsc}
\DeclareMathOperator{\si}{Si}
\DeclareMathOperator{\proj}{\vec{proj}}
\DeclareMathOperator{\scal}{scal}
\DeclareMathOperator{\sign}{sign}


%% \newcommand{\tightoverset}[2]{% for arrow vec
%%   \mathop{#2}\limits^{\vbox to -.5ex{\kern-0.75ex\hbox{$#1$}\vss}}}
\newcommand{\arrowvec}{\overrightarrow}
%\renewcommand{\vec}[1]{\arrowvec{\mathbf{#1}}}
\renewcommand{\vec}{\mathbf}
\newcommand{\veci}{{\boldsymbol{\hat{\imath}}}}
\newcommand{\vecj}{{\boldsymbol{\hat{\jmath}}}}
\newcommand{\veck}{{\boldsymbol{\hat{k}}}}
\newcommand{\vecl}{\boldsymbol{\l}}
\newcommand{\utan}{\mathbf{\hat{t}}}
\newcommand{\unormal}{\mathbf{\hat{n}}}
\newcommand{\ubinormal}{\mathbf{\hat{b}}}

\newcommand{\dotp}{\bullet}
\newcommand{\cross}{\boldsymbol\times}
\newcommand{\grad}{\boldsymbol\nabla}
\newcommand{\divergence}{\grad\dotp}
\newcommand{\curl}{\grad\cross}
%\DeclareMathOperator{\divergence}{divergence}
%\DeclareMathOperator{\curl}[1]{\grad\cross #1}
\newcommand{\lto}{\mathop{\longrightarrow\,}\limits}


\colorlet{textColor}{black} 
\colorlet{background}{white}
\colorlet{penColor}{blue!50!black} % Color of a curve in a plot
\colorlet{penColor2}{red!50!black}% Color of a curve in a plot
\colorlet{penColor3}{red!50!blue} % Color of a curve in a plot
\colorlet{penColor4}{green!50!black} % Color of a curve in a plot
\colorlet{penColor5}{orange!80!black} % Color of a curve in a plot
\colorlet{fill1}{penColor!20} % Color of fill in a plot
\colorlet{fill2}{penColor2!20} % Color of fill in a plot
\colorlet{fillp}{fill1} % Color of positive area
\colorlet{filln}{penColor2!20} % Color of negative area
\colorlet{fill3}{penColor3!20} % Fill
\colorlet{fill4}{penColor4!20} % Fill
\colorlet{fill5}{penColor5!20} % Fill
\colorlet{gridColor}{gray!50} % Color of grid in a plot

\newcommand{\surfaceColor}{violet}
\newcommand{\surfaceColorTwo}{redyellow}
\newcommand{\sliceColor}{greenyellow}




\pgfmathdeclarefunction{gauss}{2}{% gives gaussian
  \pgfmathparse{1/(#2*sqrt(2*pi))*exp(-((x-#1)^2)/(2*#2^2))}%
}


%%%%%%%%%%%%%
%% Vectors
%%%%%%%%%%%%%

%% Simple horiz vectors
\renewcommand{\vector}[1]{\left\langle #1\right\rangle}


%% %% Complex Horiz Vectors with angle brackets
%% \makeatletter
%% \renewcommand{\vector}[2][ , ]{\left\langle%
%%   \def\nextitem{\def\nextitem{#1}}%
%%   \@for \el:=#2\do{\nextitem\el}\right\rangle%
%% }
%% \makeatother

%% %% Vertical Vectors
%% \def\vector#1{\begin{bmatrix}\vecListA#1,,\end{bmatrix}}
%% \def\vecListA#1,{\if,#1,\else #1\cr \expandafter \vecListA \fi}

%%%%%%%%%%%%%
%% End of vectors
%%%%%%%%%%%%%

%\newcommand{\fullwidth}{}
%\newcommand{\normalwidth}{}



%% makes a snazzy t-chart for evaluating functions
%\newenvironment{tchart}{\rowcolors{2}{}{background!90!textColor}\array}{\endarray}

%%This is to help with formatting on future title pages.
\newenvironment{sectionOutcomes}{}{} 



%% Flowchart stuff
%\tikzstyle{startstop} = [rectangle, rounded corners, minimum width=3cm, minimum height=1cm,text centered, draw=black]
%\tikzstyle{question} = [rectangle, minimum width=3cm, minimum height=1cm, text centered, draw=black]
%\tikzstyle{decision} = [trapezium, trapezium left angle=70, trapezium right angle=110, minimum width=3cm, minimum height=1cm, text centered, draw=black]
%\tikzstyle{question} = [rectangle, rounded corners, minimum width=3cm, minimum height=1cm,text centered, draw=black]
%\tikzstyle{process} = [rectangle, minimum width=3cm, minimum height=1cm, text centered, draw=black]
%\tikzstyle{decision} = [trapezium, trapezium left angle=70, trapezium right angle=110, minimum width=3cm, minimum height=1cm, text centered, draw=black]


\title[Dig-In:]{Planes in space}

\begin{document}
\begin{abstract}
  We discuss how to find implicit and explicit formulas for planes.
\end{abstract}
\maketitle

Planes are the three-dimensional analogue of lines in two-dimensions.

\section{Implicit planes}

Remember an implicit function in $\R^3$ is one of the form:
\[
F(x,y,z) = 0
\]
We would like to know the implicit formula for a plane. Here the dot
product saves the day. Recall that if $\vec{v} = \vector{a,b,c}$ is
any vector, and $\vec{x}= \vector{x,y,z}$, then the equation
\[
\vec{v}\dotp\vec{x} = 0
\]
is solved by all vectors $\vec{x}$ that are orthogonal to
$\vec{v}$. We plotted several such vectors below:
\begin{image}
          \begin{tikzpicture}
          \begin{axis}%
            [tick label style={font=\scriptsize},axis on top,
	      axis lines=center,
	      view={135}{25},
	      name=myplot,
	      %xtick={-3,3},minor tick num=2,
	      %ytick={-3,3},
	      %ztick={-3,3},
	      ymin=-4,ymax=4,
	      xmin=-4,xmax=4,
	      zmin=-4, zmax=4,
	      every axis x label/.style={at={(axis cs:\pgfkeysvalueof{/pgfplots/xmax},0,0)},xshift=-3pt,yshift=-3pt},
	      xlabel={\scriptsize $x$},
	      every axis y label/.style={at={(axis cs:0,\pgfkeysvalueof{/pgfplots/ymax},0)},xshift=0pt,yshift=-5pt},
	      ylabel={\scriptsize $y$},
	      every axis z label/.style={at={(axis cs:0,0,\pgfkeysvalueof{/pgfplots/zmax})},xshift=0pt,yshift=4pt},
	      zlabel={\scriptsize $z$}
	    ]
            \draw[thick,->,penColor2] (axis cs: 2,1,1) -- (axis cs: 2/3,2/3,2/3);
            \draw[thick,->,penColor2] (axis cs: 2,1,1) -- (axis cs: 1.057,.057,1.47);
            \draw[thick,->,penColor2] (axis cs: 2,1,1) -- (axis cs: 2,0,2);
            \draw[thick,->,penColor2] (axis cs: 2,1,1) -- (axis cs: 2.943,.529,1.943);
            \draw[thick,->,penColor2] (axis cs: 2,1,1) -- (axis cs: 10/3,4/3,4/3);
            \draw[thick,->,penColor2] (axis cs: 2,1,1) -- (axis cs: 2.943,1.943,0.529);
            \draw[thick,->,penColor2] (axis cs: 2,1,1) -- (axis cs: 2,2,0);
            \draw[thick,->,penColor2] (axis cs: 2,1,1) -- (axis cs: 1.057,1.471,0.057);
            
            
            \draw[thick,->,penColor] (axis cs: 2,1,1) -- (axis cs: 1,3,3);
          \end{axis}
        \end{tikzpicture}
\end{image}
From this we see that
\begin{align*}
\vec{v}\dotp\vec{x} &=0\\
ax+by+cz &= 0
\end{align*}
gives the formula for a plane. Since $\vec{0} = \vec{x}$ is a
solution, this plane must pass through the origin. If we want our
plane to be located anywhere in space, we must know a point on the
plane, call it $\vec{p}=\vector{x_0,y_0,z_0}$. Putting this together, we can
now see that if you know
\begin{itemize}
  \item a vector $\vec{v} = \vector{a,b,c}$ and
  \item a point (given by a vector) $\vec{p} = \vector{x_0,y_0,z_0}$
\end{itemize}
then,
\begin{align*}
  (\vec{v}-\vec{p})\dotp \vec{x} &= 0\\
  a(x-x_0) + b(y-y_0) + c(z-z_0) &= 0
\end{align*}
is a formula for a plane passing through the point $(x_0,y_0,z_0)$
with normal vector $\vec{v}$.

\begin{question}
  Find the implicit equation of a plane that passes through the point
  $(5,-5,-1)$ and with normal vector $\vector{-5,5,5}$.
  \begin{onlineOnly}
    Check your answer by modifying the definition of $F(x,y,z)$ in
    the \sage\ code below:
  \begin{sageCell}
vector=arrow3d((3,-5,-1),(-2,0,4),5,color='blue');

F(x,y,z) = 1*(x+2)+1*(y+2)+1*(z+2)

plane=implicit_plot3d(F(x,y,z)==1,(x,-10,10),(y,-10,10),(z,-10,10),color='red');
plane+vector
  \end{sageCell}
  \end{onlineOnly}
\end{question}

Normal vectors not only allow us to define equations for planes but also 
they help us describe properties of planes.

\begin{definition}
  Two planes are said to be \dfn{parallel} if their normal vectors are
  parallel. Two planes are said to be \dfn{orthogonal} if their normal
  vectors are orthogonal.
\end{definition}

\begin{question}
  What is the (most obvious) normal vector for the plane
  \[
  -6x+10y-2z = 4?
  \]
  \begin{prompt}
    \[
    \vec{n} = \vector{\answer{-6},\answer{10},\answer{-2}}
    \]
  \end{prompt}
  \begin{question}
    Which of the following planes are parallel to the plane $-6x+10y-2z = 4$?
    \begin{selectAll}
      \choice{$-2.46x+3.9y-0.82z=3$}
      \choice[correct]{$-2.07x+3.45y-0.69z = 8$}
      \choice[correct]{$10.38x-17.3y+3.46z=0$}
      \choice{$15.03x-25.1y+5.02z=0$}
    \end{selectAll}
      \begin{question}
    Which of the following planes are orthogonal to the plane $-6x+10y-2z = 4$?
    \begin{selectAll}
      \choice[correct]{$34x+16y-2z=0$}
      \choice{$-54x-32y+8z=-3$}
      \choice[correct]{$-36x-42y-102z=-17$}
      \choice{$54x+31y-2z=4$}
    \end{selectAll}
  \end{question}
  \end{question}
\end{question}


\section{Parametric planes}


Given \textbf{any} two nonzero vectors, $\vec{v}$ and $\vec{w}$ such that
\[
\vec{v}\cross \vec{w} \ne \vec{0}
\]
we can produce a parametric formula for a plane by writing
\[
\vec{L}(s,t) = \vec{p} + s\vec{v} + t\vec{w},
\]
where $\vec{p}$ is a vector whose ``tip'' is on the plane, and
$\vec{v}$ and $\vec{w}$ are in the plane.
\begin{question}
  Given two nonzero vectors, $\vec{v}$ and $\vec{w}$, what does it
  mean for $\vec{v}\cross \vec{w} \ne \vec{0}$?
  \begin{prompt}
    \begin{multipleChoice}
      \choice{It means these vectors are parallel.}
      \choice[correct]{It means these vectors are not parallel.}
      \choice{It means these vectors are orthogonal.}
      \choice{It means these vectors are not orthogonal.}
    \end{multipleChoice}
  \end{prompt}
\end{question}

This formula is very similar to our formula for a line,
\[
\vecl(t) = \vec{p} + t\vec{v}
\]
where $\vec{v}$ is a vector that points in the direction of the line,
both represent linear relationships, and hence we use similar notation
for both.

Now that we have \textbf{two} methods of graphing planes, let's use
both of the representations at once!

\begin{question}
  Let $\vec{v} = \vector{1,-2,1}$ and $\vector{w} =
  \vector{-1,-2,1}$. Compute $\vec{v}\cross\vec{w}$.
  \begin{prompt}
    \[
    \vec{v}\cross\vec{w} = \vector{\answer{0},\answer{-2},\answer{-4}}
    \]
  \end{prompt}
  \begin{question}
    Use your answer above to give an \textbf{implicit} equation for
    the plane that passes through the point $(1,2,3)$.
    \begin{onlineOnly}
      Check your answer by modifying the definition of $F(x,y,z)$ in
      the \sage\ code below:
      \begin{sageCell}
vector=arrow3d((1,2,3),(1,-2,-4),5,color='blue');
        
F(x,y,z) = 1*(x+2)+1*(y+2)+1*(z+2)

plane=implicit_plot3d(F(x,y,z)==1,(x,-10,10),(y,-10,10),(z,-10,10),color='red');
plane+vector
      \end{sageCell}
    \end{onlineOnly}
    \begin{question}
      Now give a parametric formula for the same plane using the vectors given above.
      \begin{sageCell}
vector=arrow3d((1,2,3),(1,-2,-4),5,color='blue');

L(s,t) = (1,2,3) + t*(1-2,1) + s*(-1,-2,1)

paraPlane = parametric_plot(L(s,t),(s,-2,3),(t,-2,3))
paraPlane+vector
      \end{sageCell}
    \end{question}
  \end{question}
\end{question}


\end{document}
