\documentclass{ximera}

%\usepackage{todonotes}

\newcommand{\todo}{}


\graphicspath{
{./}
{../functionsOfSeveralVariables/}
{../normalVectors/}
{../lagrangeMultipliers/}
}


\usepackage{tkz-euclide}
\tikzset{>=stealth} %% cool arrow head
\tikzset{shorten <>/.style={ shorten >=#1, shorten <=#1 } } %% allows shorter vectors

\usetikzlibrary{backgrounds} %% for boxes around graphs
\usetikzlibrary{shapes,positioning}  %% Clouds and stars
\usetikzlibrary{matrix} %% for matrix
\usepgfplotslibrary{polar} %% for polar plots
\usetkzobj{all}
\usepackage[makeroom]{cancel} %% for strike outs
%\usepackage{mathtools} %% for pretty underbrace % Breaks Ximera
\usepackage{multicol}





\usepackage{array}
\setlength{\extrarowheight}{+.1cm}   
\newdimen\digitwidth
\settowidth\digitwidth{9}
\def\divrule#1#2{
\noalign{\moveright#1\digitwidth
\vbox{\hrule width#2\digitwidth}}}





\newcommand{\RR}{\mathbb R}
\newcommand{\R}{\mathbb R}
\newcommand{\N}{\mathbb N}
\newcommand{\Z}{\mathbb Z}

\newcommand{\sage}{\textsf{SageMath}}


%\renewcommand{\d}{\,d\!}
\renewcommand{\d}{\mathop{}\!d}
\newcommand{\dd}[2][]{\frac{\d #1}{\d #2}}
\newcommand{\pp}[2][]{\frac{\partial #1}{\partial #2}}
\renewcommand{\l}{\ell}
\newcommand{\ddx}{\frac{d}{\d x}}

\newcommand{\zeroOverZero}{\ensuremath{\boldsymbol{\tfrac{0}{0}}}}
\newcommand{\inftyOverInfty}{\ensuremath{\boldsymbol{\tfrac{\infty}{\infty}}}}
\newcommand{\zeroOverInfty}{\ensuremath{\boldsymbol{\tfrac{0}{\infty}}}}
\newcommand{\zeroTimesInfty}{\ensuremath{\small\boldsymbol{0\cdot \infty}}}
\newcommand{\inftyMinusInfty}{\ensuremath{\small\boldsymbol{\infty - \infty}}}
\newcommand{\oneToInfty}{\ensuremath{\boldsymbol{1^\infty}}}
\newcommand{\zeroToZero}{\ensuremath{\boldsymbol{0^0}}}
\newcommand{\inftyToZero}{\ensuremath{\boldsymbol{\infty^0}}}



\newcommand{\numOverZero}{\ensuremath{\boldsymbol{\tfrac{\#}{0}}}}
\newcommand{\dfn}{\textbf}
%\newcommand{\unit}{\,\mathrm}
\newcommand{\unit}{\mathop{}\!\mathrm}
\newcommand{\eval}[1]{\bigg[ #1 \bigg]}
\newcommand{\seq}[1]{\left( #1 \right)}
\renewcommand{\epsilon}{\varepsilon}
\renewcommand{\iff}{\Leftrightarrow}

\DeclareMathOperator{\arccot}{arccot}
\DeclareMathOperator{\arcsec}{arcsec}
\DeclareMathOperator{\arccsc}{arccsc}
\DeclareMathOperator{\si}{Si}
\DeclareMathOperator{\proj}{\vec{proj}}
\DeclareMathOperator{\scal}{scal}
\DeclareMathOperator{\sign}{sign}


%% \newcommand{\tightoverset}[2]{% for arrow vec
%%   \mathop{#2}\limits^{\vbox to -.5ex{\kern-0.75ex\hbox{$#1$}\vss}}}
\newcommand{\arrowvec}{\overrightarrow}
%\renewcommand{\vec}[1]{\arrowvec{\mathbf{#1}}}
\renewcommand{\vec}{\mathbf}
\newcommand{\veci}{{\boldsymbol{\hat{\imath}}}}
\newcommand{\vecj}{{\boldsymbol{\hat{\jmath}}}}
\newcommand{\veck}{{\boldsymbol{\hat{k}}}}
\newcommand{\vecl}{\boldsymbol{\l}}
\newcommand{\utan}{\mathbf{\hat{t}}}
\newcommand{\unormal}{\mathbf{\hat{n}}}
\newcommand{\ubinormal}{\mathbf{\hat{b}}}

\newcommand{\dotp}{\bullet}
\newcommand{\cross}{\boldsymbol\times}
\newcommand{\grad}{\boldsymbol\nabla}
\newcommand{\divergence}{\grad\dotp}
\newcommand{\curl}{\grad\cross}
%\DeclareMathOperator{\divergence}{divergence}
%\DeclareMathOperator{\curl}[1]{\grad\cross #1}
\newcommand{\lto}{\mathop{\longrightarrow\,}\limits}


\colorlet{textColor}{black} 
\colorlet{background}{white}
\colorlet{penColor}{blue!50!black} % Color of a curve in a plot
\colorlet{penColor2}{red!50!black}% Color of a curve in a plot
\colorlet{penColor3}{red!50!blue} % Color of a curve in a plot
\colorlet{penColor4}{green!50!black} % Color of a curve in a plot
\colorlet{penColor5}{orange!80!black} % Color of a curve in a plot
\colorlet{fill1}{penColor!20} % Color of fill in a plot
\colorlet{fill2}{penColor2!20} % Color of fill in a plot
\colorlet{fillp}{fill1} % Color of positive area
\colorlet{filln}{penColor2!20} % Color of negative area
\colorlet{fill3}{penColor3!20} % Fill
\colorlet{fill4}{penColor4!20} % Fill
\colorlet{fill5}{penColor5!20} % Fill
\colorlet{gridColor}{gray!50} % Color of grid in a plot

\newcommand{\surfaceColor}{violet}
\newcommand{\surfaceColorTwo}{redyellow}
\newcommand{\sliceColor}{greenyellow}




\pgfmathdeclarefunction{gauss}{2}{% gives gaussian
  \pgfmathparse{1/(#2*sqrt(2*pi))*exp(-((x-#1)^2)/(2*#2^2))}%
}


%%%%%%%%%%%%%
%% Vectors
%%%%%%%%%%%%%

%% Simple horiz vectors
\renewcommand{\vector}[1]{\left\langle #1\right\rangle}


%% %% Complex Horiz Vectors with angle brackets
%% \makeatletter
%% \renewcommand{\vector}[2][ , ]{\left\langle%
%%   \def\nextitem{\def\nextitem{#1}}%
%%   \@for \el:=#2\do{\nextitem\el}\right\rangle%
%% }
%% \makeatother

%% %% Vertical Vectors
%% \def\vector#1{\begin{bmatrix}\vecListA#1,,\end{bmatrix}}
%% \def\vecListA#1,{\if,#1,\else #1\cr \expandafter \vecListA \fi}

%%%%%%%%%%%%%
%% End of vectors
%%%%%%%%%%%%%

%\newcommand{\fullwidth}{}
%\newcommand{\normalwidth}{}



%% makes a snazzy t-chart for evaluating functions
%\newenvironment{tchart}{\rowcolors{2}{}{background!90!textColor}\array}{\endarray}

%%This is to help with formatting on future title pages.
\newenvironment{sectionOutcomes}{}{} 



%% Flowchart stuff
%\tikzstyle{startstop} = [rectangle, rounded corners, minimum width=3cm, minimum height=1cm,text centered, draw=black]
%\tikzstyle{question} = [rectangle, minimum width=3cm, minimum height=1cm, text centered, draw=black]
%\tikzstyle{decision} = [trapezium, trapezium left angle=70, trapezium right angle=110, minimum width=3cm, minimum height=1cm, text centered, draw=black]
%\tikzstyle{question} = [rectangle, rounded corners, minimum width=3cm, minimum height=1cm,text centered, draw=black]
%\tikzstyle{process} = [rectangle, minimum width=3cm, minimum height=1cm, text centered, draw=black]
%\tikzstyle{decision} = [trapezium, trapezium left angle=70, trapezium right angle=110, minimum width=3cm, minimum height=1cm, text centered, draw=black]


\title[Dig-In:]{Unit tangent and unit normal vectors}

\begin{document}
\begin{abstract}
  We introduce two important unit vectors. 
\end{abstract}
\maketitle


Given a smooth vector-valued function $\vec{p}(t)$, \textit{any}
vector parallel to $\vec{p}'(t_0)$ is \textit{tangent} to the graph of
$\vec{p}(t)$ at $t=t_0$. It is often useful to consider just the
\textit{direction} of $\vec{p}'(t)$ and not its magnitude. Therefore we are
interested in the unit vector in the direction of $\vec{p}'(t)$. This
leads to a definition.
\begin{definition}
Let $\vec{p}(t)$ be a smooth function on an open interval $I$. The
\dfn{unit tangent vector} $\vec{t}(t)$ is \index{unit tangent
  vector!definition} \index{unit vector!unit tangent vector}
\[
\vec{t}(t) = \frac{\vec{p}'(t)}{|\vec{p}'(t)|}.
\]
\end{definition}

\begin{question}
  Let $\vec{p}(t) = \vector{3\cos(t), 3\sin(t), 4t}$. Find $\vec{t}(t)$.
  \begin{prompt}
    \[
    \vec{t}(t) = \vector{\answer{\frac{-3}{5}\sin(t)},\answer{\frac{3}{5}\cos(t)},\answer{\frac{4}{5}}}
    \]
    \begin{feedback}
      The unit tangent vector $\vec{t}(t)$ \textbf{always} has a constant
      magnitude of $1$.
    \end{feedback}
  \end{prompt}
\end{question}

Just as knowing the direction tangent to a path is important, knowing
a direction orthogonal to a path is important. When dealing with
real-valued functions, we defined the \dfn{normal line} at a point to
the be the line through the point that was perpendicular to the
tangent line at that point. We can do a similar thing with
vector-valued functions. Given $\vec{p}(t)$ in $\R^2$, we have $2$
directions perpendicular to the tangent vector
\begin{image}
  \begin{tikzpicture}
    \begin{axis}%
      [width=175pt,tick label style={font=\scriptsize},axis on top,
	axis lines=center,
	view={115}{25},
	name=myplot,
	%xtick={-3,3},minor tick num=2,
	%ytick={-3,3},
	%ztick={-3,3},
	ymin=-3.5,ymax=3.5,
	xmin=-3.5,xmax=3.5,
	zmin=-15.9, zmax=15.9,
	every axis x label/.style={at={(axis cs:\pgfkeysvalueof{/pgfplots/xmax},0,0)},xshift=-3pt,yshift=-3pt},
	xlabel={\scriptsize $x$},
	every axis y label/.style={at={(axis cs:0,\pgfkeysvalueof{/pgfplots/ymax},0)},xshift=0pt,yshift=-5pt},
	ylabel={\scriptsize $y$},
	every axis z label/.style={at={(axis cs:0,0,\pgfkeysvalueof{/pgfplots/zmax})},xshift=0pt,yshift=4pt},
	zlabel={\scriptsize $z$}
      ]
      
      \addplot3[domain=-3.14:3.14,,thick,smooth,samples y=0,penColor,samples=30,] ({3*cos(x*180/3.14)},{3*sin(x*180/3.14)},{4*x});

      
      \draw[thick,->,penColor2] (axis cs: 3,0,0) -- (axis cs: 3,.6,.8);
      \draw[thick,->,penColor2] (axis cs: 1.6,2.5,4) -- (axis cs: 1.12,2.85,4.8);
    \end{axis}
  \end{tikzpicture}
\end{image}
The young mathematician wonders ``Is one of these two directions
preferable over the other?''  This question only gets harder in higher
dimensions.  Given $\vec{p}(t)$ in $\R^3$, there are infinite vectors
orthogonal to the tangent vector at a given point. Again, we might
wonder ``Is one of these infinite choices preferable over the others?
Is one of these the `right' choice?''

The answer in both $\R^2$ and $\R^3$ is ``Yes, there is one vector
that is preferable, and it is the `right' one to choose!'' Recall
\begin{quote}
If $\vec{p}(t)$ has constant length, then $\vec{p}(t)$ is orthogonal
to $\vec{p}'(t)$ for all $t$.
\end{quote}
Since $\vec{t}(t)$, the unit tangent vector, it necessarily has
constant length. Therefore
\[\index{vector-valued function!of constant length}
\vec{t}(t)\text{ is orthogonal to }\vec{t}'(t).
\]

The vector-valued function $\vec{t}'(t)$ is more than just a
convenient choice of vector that is orthogonal to $\vec{p}'(t)$;
rather, it is the ``right'' choice. We will use this to construct our
\textit{unit normal vector}:

\begin{definition}
Let $\vec{p}(t)$ be a vector-valued function where the unit tangent
vector, $\vec{t}(t)$, is smooth on an open interval $I$. The
\dfn{unit normal vector} $\vec{n}(t)$ is \index{unit normal
  vector!definition}\index{unit vector!unit normal vector}
\[
\vec{n}(t) = \frac{\vec{t}'(t)}{|\vec{t}'(t)|}.
\]
\end{definition}
\begin{warning}
  Even though $\vec{t}(t)$ is a unit vector, this \textbf{does not}
  imply that $\vec{t}'(t)$ is also a unit vector.
\end{warning}

\begin{question}
  Let $\vec{p}(t) = \vector{3\cos t, 3\sin t, 4t}$ as before. Find
  $\vec{n}(t)$.
  \begin{prompt}
    \[
    \vec{n}(t) =\vector{\answer{-\cos(t)},\answer{-\sin(t)},\answer{0}}.
    \]
    \begin{feedback}
      As a gesture of friendship, we present you with the following
      graph of the situation.
      \begin{image}
        \begin{tikzpicture}
          \begin{axis}%
            [width=175pt,tick label style={font=\scriptsize},axis on top,
	      axis lines=center,
	      view={135}{25},
	      name=myplot,
	      %xtick={-3,3},minor tick num=2,
	      %ytick={-3,3},
	      %ztick={-3,3},
	      ymin=-3.5,ymax=3.5,
	      xmin=-3.5,xmax=3.5,
	      zmin=-15.9, zmax=15.9,
	      every axis x label/.style={at={(axis cs:\pgfkeysvalueof{/pgfplots/xmax},0,0)},xshift=-3pt,yshift=-3pt},
	      xlabel={\scriptsize $x$},
	      every axis y label/.style={at={(axis cs:0,\pgfkeysvalueof{/pgfplots/ymax},0)},xshift=0pt,yshift=-5pt},
	      ylabel={\scriptsize $y$},
	      every axis z label/.style={at={(axis cs:0,0,\pgfkeysvalueof{/pgfplots/zmax})},xshift=0pt,yshift=4pt},
	      zlabel={\scriptsize $z$}
	    ]
            \addplot3[domain=-3.14:3.14,,thick,smooth,samples y=0,penColor,samples=30,] ({3*cos(x*180/3.14)},{3*sin(x*180/3.14)},{4*x});

            
            \draw[thick,->,penColor2] (axis cs: 0,3,6.28) -- (axis cs: -.6,3,7.08);
            \draw[thick,->,penColor2] (axis cs: 0,3,6.28) -- (axis cs: 0,2,6.28);
          \end{axis}
        \end{tikzpicture}
      \end{image}
    \end{feedback}
  \end{prompt}
\end{question}

\end{document}
