\documentclass{ximera}

%\usepackage{todonotes}

\newcommand{\todo}{}


\graphicspath{
{./}
{../functionsOfSeveralVariables/}
{../normalVectors/}
{../lagrangeMultipliers/}
}


\usepackage{tkz-euclide}
\tikzset{>=stealth} %% cool arrow head
\tikzset{shorten <>/.style={ shorten >=#1, shorten <=#1 } } %% allows shorter vectors

\usetikzlibrary{backgrounds} %% for boxes around graphs
\usetikzlibrary{shapes,positioning}  %% Clouds and stars
\usetikzlibrary{matrix} %% for matrix
\usepgfplotslibrary{polar} %% for polar plots
\usetkzobj{all}
\usepackage[makeroom]{cancel} %% for strike outs
%\usepackage{mathtools} %% for pretty underbrace % Breaks Ximera
\usepackage{multicol}





\usepackage{array}
\setlength{\extrarowheight}{+.1cm}   
\newdimen\digitwidth
\settowidth\digitwidth{9}
\def\divrule#1#2{
\noalign{\moveright#1\digitwidth
\vbox{\hrule width#2\digitwidth}}}





\newcommand{\RR}{\mathbb R}
\newcommand{\R}{\mathbb R}
\newcommand{\N}{\mathbb N}
\newcommand{\Z}{\mathbb Z}

\newcommand{\sage}{\textsf{SageMath}}


%\renewcommand{\d}{\,d\!}
\renewcommand{\d}{\mathop{}\!d}
\newcommand{\dd}[2][]{\frac{\d #1}{\d #2}}
\newcommand{\pp}[2][]{\frac{\partial #1}{\partial #2}}
\renewcommand{\l}{\ell}
\newcommand{\ddx}{\frac{d}{\d x}}

\newcommand{\zeroOverZero}{\ensuremath{\boldsymbol{\tfrac{0}{0}}}}
\newcommand{\inftyOverInfty}{\ensuremath{\boldsymbol{\tfrac{\infty}{\infty}}}}
\newcommand{\zeroOverInfty}{\ensuremath{\boldsymbol{\tfrac{0}{\infty}}}}
\newcommand{\zeroTimesInfty}{\ensuremath{\small\boldsymbol{0\cdot \infty}}}
\newcommand{\inftyMinusInfty}{\ensuremath{\small\boldsymbol{\infty - \infty}}}
\newcommand{\oneToInfty}{\ensuremath{\boldsymbol{1^\infty}}}
\newcommand{\zeroToZero}{\ensuremath{\boldsymbol{0^0}}}
\newcommand{\inftyToZero}{\ensuremath{\boldsymbol{\infty^0}}}



\newcommand{\numOverZero}{\ensuremath{\boldsymbol{\tfrac{\#}{0}}}}
\newcommand{\dfn}{\textbf}
%\newcommand{\unit}{\,\mathrm}
\newcommand{\unit}{\mathop{}\!\mathrm}
\newcommand{\eval}[1]{\bigg[ #1 \bigg]}
\newcommand{\seq}[1]{\left( #1 \right)}
\renewcommand{\epsilon}{\varepsilon}
\renewcommand{\iff}{\Leftrightarrow}

\DeclareMathOperator{\arccot}{arccot}
\DeclareMathOperator{\arcsec}{arcsec}
\DeclareMathOperator{\arccsc}{arccsc}
\DeclareMathOperator{\si}{Si}
\DeclareMathOperator{\proj}{\vec{proj}}
\DeclareMathOperator{\scal}{scal}
\DeclareMathOperator{\sign}{sign}


%% \newcommand{\tightoverset}[2]{% for arrow vec
%%   \mathop{#2}\limits^{\vbox to -.5ex{\kern-0.75ex\hbox{$#1$}\vss}}}
\newcommand{\arrowvec}{\overrightarrow}
%\renewcommand{\vec}[1]{\arrowvec{\mathbf{#1}}}
\renewcommand{\vec}{\mathbf}
\newcommand{\veci}{{\boldsymbol{\hat{\imath}}}}
\newcommand{\vecj}{{\boldsymbol{\hat{\jmath}}}}
\newcommand{\veck}{{\boldsymbol{\hat{k}}}}
\newcommand{\vecl}{\boldsymbol{\l}}
\newcommand{\utan}{\mathbf{\hat{t}}}
\newcommand{\unormal}{\mathbf{\hat{n}}}
\newcommand{\ubinormal}{\mathbf{\hat{b}}}

\newcommand{\dotp}{\bullet}
\newcommand{\cross}{\boldsymbol\times}
\newcommand{\grad}{\boldsymbol\nabla}
\newcommand{\divergence}{\grad\dotp}
\newcommand{\curl}{\grad\cross}
%\DeclareMathOperator{\divergence}{divergence}
%\DeclareMathOperator{\curl}[1]{\grad\cross #1}
\newcommand{\lto}{\mathop{\longrightarrow\,}\limits}


\colorlet{textColor}{black} 
\colorlet{background}{white}
\colorlet{penColor}{blue!50!black} % Color of a curve in a plot
\colorlet{penColor2}{red!50!black}% Color of a curve in a plot
\colorlet{penColor3}{red!50!blue} % Color of a curve in a plot
\colorlet{penColor4}{green!50!black} % Color of a curve in a plot
\colorlet{penColor5}{orange!80!black} % Color of a curve in a plot
\colorlet{fill1}{penColor!20} % Color of fill in a plot
\colorlet{fill2}{penColor2!20} % Color of fill in a plot
\colorlet{fillp}{fill1} % Color of positive area
\colorlet{filln}{penColor2!20} % Color of negative area
\colorlet{fill3}{penColor3!20} % Fill
\colorlet{fill4}{penColor4!20} % Fill
\colorlet{fill5}{penColor5!20} % Fill
\colorlet{gridColor}{gray!50} % Color of grid in a plot

\newcommand{\surfaceColor}{violet}
\newcommand{\surfaceColorTwo}{redyellow}
\newcommand{\sliceColor}{greenyellow}




\pgfmathdeclarefunction{gauss}{2}{% gives gaussian
  \pgfmathparse{1/(#2*sqrt(2*pi))*exp(-((x-#1)^2)/(2*#2^2))}%
}


%%%%%%%%%%%%%
%% Vectors
%%%%%%%%%%%%%

%% Simple horiz vectors
\renewcommand{\vector}[1]{\left\langle #1\right\rangle}


%% %% Complex Horiz Vectors with angle brackets
%% \makeatletter
%% \renewcommand{\vector}[2][ , ]{\left\langle%
%%   \def\nextitem{\def\nextitem{#1}}%
%%   \@for \el:=#2\do{\nextitem\el}\right\rangle%
%% }
%% \makeatother

%% %% Vertical Vectors
%% \def\vector#1{\begin{bmatrix}\vecListA#1,,\end{bmatrix}}
%% \def\vecListA#1,{\if,#1,\else #1\cr \expandafter \vecListA \fi}

%%%%%%%%%%%%%
%% End of vectors
%%%%%%%%%%%%%

%\newcommand{\fullwidth}{}
%\newcommand{\normalwidth}{}



%% makes a snazzy t-chart for evaluating functions
%\newenvironment{tchart}{\rowcolors{2}{}{background!90!textColor}\array}{\endarray}

%%This is to help with formatting on future title pages.
\newenvironment{sectionOutcomes}{}{} 



%% Flowchart stuff
%\tikzstyle{startstop} = [rectangle, rounded corners, minimum width=3cm, minimum height=1cm,text centered, draw=black]
%\tikzstyle{question} = [rectangle, minimum width=3cm, minimum height=1cm, text centered, draw=black]
%\tikzstyle{decision} = [trapezium, trapezium left angle=70, trapezium right angle=110, minimum width=3cm, minimum height=1cm, text centered, draw=black]
%\tikzstyle{question} = [rectangle, rounded corners, minimum width=3cm, minimum height=1cm,text centered, draw=black]
%\tikzstyle{process} = [rectangle, minimum width=3cm, minimum height=1cm, text centered, draw=black]
%\tikzstyle{decision} = [trapezium, trapezium left angle=70, trapezium right angle=110, minimum width=3cm, minimum height=1cm, text centered, draw=black]



\title[Dig-In:]{Taylor polynomials}

\begin{document}
\begin{abstract}
  We introduce Taylor polynomials for functions of several variables.
\end{abstract}
\maketitle


Recall the definition of a \textit{Taylor polynomial}:

\begin{definition}
  Let $f:\R\to\R$ be a function whose first $d$ derivatives exist at $x=c$.
  The \dfn{Taylor polynomial} of degree $d$ of $f$ at $x=c$ is
  \begin{align*}
    p_d(x) = \sum_{k=0}^d\frac{f^{(k)}(c)}{k!}(x-c)^k.
  \end{align*}
\end{definition}

\begin{question}
  Let $f(x) = \sin(x)$. Compute
  \[
  p_7(x)\begin{prompt}
    = \answer{x-x^3/3!+x^5/5!-x^7/7!}
  \end{prompt}
  \]
  \begin{question}
  Let $f(x) = \cos(x)$. Compute
  \[
  p_7(x)\begin{prompt}
    = \answer{1-x^2/2+x^4/4!-x^6/6!}
  \end{prompt}
  \]
  \begin{question}
    Let $f(x) = e^x$. Compute
    \[
    p_7(x)\begin{prompt}
      = \answer{1 + x + x^2/2+ + x^3/3! + x^4/4!+ x^5/5! + x^6/6!+x^7/7!}
    \end{prompt}
    \]
  \end{question}
\end{question}
\end{question}

We have a similar formula for functions $F:\R^n\to \R$:
\begin{definition}
  Let $F:\R^n\to\R$ be a function whose first $d$ derivatives exist at
  $\vec{x}=\vec{c}$.  The \dfn{Taylor polynomial} of degree $d$ of $F$
  at $\vec{x}=\vec{c}$ is
  \[
  P_d(\vec{x}) = \sum_{k=0}^d \eval{\eval{\frac{(\vec{a}\dotp \grad)^k F(\vec{x})}{k!}}_{\vec{x}=\vec{c}}}_{\vec{a}=\vec{x}-\vec{c}}
  \]
\end{definition}
This will take some unpacking. First note that
\begin{align*}
  P_0(\vec(x)) &= \sum_{k=0}^0 \eval{\eval{\frac{(\vec{a}\dotp \grad)^k F(\vec{x})}{k!}}_{\vec{x}=\vec{c}}}_{\vec{a}=\vec{x}-\vec{c}}\\
  &=\eval{\eval{\frac{(\vec{a}\dotp \grad)^0 F(\vec{x})}{0!}}_{\vec{x}=\vec{c}}}_{\vec{a}=\vec{x}-\vec{c}}\\
  &=\eval{\eval{F(\vec{x})}_{\vec{x}=\vec{c}}}_{\vec{a}=\vec{x}-\vec{c}}\\
  &=F(\vec{c}).
\end{align*}
This means for any function $F:\R^n\to\R$, the $0$th degree Taylor
polynomial for $F$ at $\vec{x}=\vec{c}$ is just
\[
P_0(\vec{x})=F(\vec{c}).
\]
Now let's look at the $1$st degree Taylor polynomial:
\begin{align*}
  P_1(\vec{x})&= \sum_{k=0}^1 \eval{\eval{\frac{(\vec{a}\dotp \grad)^k F(\vec{x})}{k!}}_{\vec{x}=\vec{c}}}_{\vec{a}=\vec{x}-\vec{c}}\\
  &=P_0(\vec{x}) + \eval{\eval{\frac{(\vec{a}\dotp \grad)^1 F(\vec{x})}{1!}}_{\vec{x}=\vec{c}}}_{\vec{a}=\vec{x}-\vec{c}}\\
  &=F(\vec{c}) + \eval{\eval{(\vec{a}\dotp \grad) F(\vec{x})}_{\vec{x}=\vec{c}}}_{\vec{a}=\vec{x}-\vec{c}}\\
  &= F(\vec{c}) + \grad F(\vec{c})\dotp (\vec{x}-\vec{c}).
\end{align*}
This means for any function $F:\R^n\to\R$, the $1$st degree Taylor
polynomial for $F$ at $\vec{x}=\vec{c}$ is just
\[
P_1(\vec{x}) = F(\vec{c}) + \grad F(\vec{c})\dotp (\vec{x}-\vec{c}),
\]
and this is the tangent ``plane'' for $F$ at $\vec{x}= \vec{c}$.


\begin{question}
  Consider $F(x,y)= 3+4x-5y$. Compute the $1$st degree Taylor
  polynomial for $F$ at $\vector{x,y} =\vector{0,0}$.
  \begin{prompt}
    \[
    P_1(x,y) = \answer{3+4x-5y}
    \]
  \end{prompt}
\end{question}

To get our hands on the $2$nd degree Taylor polynomial, we will
specialize to functions $F:\R^2\to\R$. Let $\vec{c}=\vector{c_1,c_2}$
and let $\vec{x} = \vector{x,y}$.
Write with me:
\[
P_2(\vec{x}) = F(\vec{c})
+ \grad F(\vec{c})\dotp (\vec{x}-\vec{c})
+\eval{\eval{\frac{(\vec{a}\dotp \grad)^2 F(\vec{x})}{2!}}_{\vec{x}=\vec{c}}}_{\vec{a}=\vec{x}-\vec{c}}
\]
In this case,
\[
(\vec{a}\dotp\grad)F = a_1 \pp[F]{x} + a_2 \pp[F]{y}
\]
and 
\begin{align*}
(\vec{a}\dotp\grad)^2F &=(\vec{a}\dotp\grad)(\vec{a}\dotp\grad)F\\
  &= (\vec{a}\dotp\grad)\left(a_1 \pp[F]{x} + a_2 \pp[F]{y}\right)\\
  &= \left(a_1 \pp[F]{x} + a_2 \pp[F]{y}\right)\left(a_1 \pp[F]{x} + a_2 \pp[F]{y}\right)
\end{align*}
Now we use the distributivity property and since we are assuming that
all derivatives of $F$ exist, we have that
\[
\frac{\partial^2F}{\partial x\partial y}  = \frac{\partial^2F}{\partial y\partial x}
\]
so
\[
(\vec{a}\dotp\grad)^2F = a_1^2\frac{\partial^2F}{\partial x^2} + 2a_1
a_2\frac{\partial^2F}{\partial x\partial y} +
a_2^2\frac{\partial^2F}{\partial y^2}.
\]



\[
+ \frac{F^{(2,0)}(\vec{c}) (x-c_1)^2 + F^{(1,1)}(\vec{c})(x-c_1)(y-c_2)  + F^{(0,2)}(\vec{c})(y-c_2)^2}{2}
\]


\section{Try it, you might like it}

Examples:

\[
\sin(x+ y)
\]

\begin{example}
  Compute the Taylor series for $f(x) = \sin(x)$ centered at $x=0$.
  \begin{explanation}
    We'll start by making a table of derivatives:
    \[
    \begin{array}{lcl}
      f(x) = \sin(x) & \Rightarrow &f(0) = 0\\
      f'(x) = \answer[given]{\cos(x)} & \Rightarrow & f'(0) = \answer[given]{1}\\
      f''(x) = \answer[given]{-\sin(x)} &\Rightarrow &f''(0) = \answer[given]{0}\\
      f'''(x) = \answer[given]{-\cos(x)} &\Rightarrow &f'''(0) = \answer[given]{-1}\\
      f^{(4)}(x) = \answer[given]{\sin(x)} &\Rightarrow &f^{(4)}(0) = \answer[given]{0}\\
      f^{(5)}(x) = \answer[given]{\cos(x)} &\Rightarrow &f^{(5)}(0) = \answer[given]{1}\\
      f^{(6)}(x) = \answer[given]{-\sin(x)} &\Rightarrow &f^{(6)}(0) =\answer[given]{0}\\
      f^{(7)}(x) = \answer[given]{-\cos(x)} &\Rightarrow &f^{(7)}(0) = \answer[given]{-1}\\
      f^{(8)}(x) = \answer[given]{\sin(x)} &\Rightarrow &f^{(8)}(0) = \answer[given]{0}\\
      f^{(9)}(x) = \answer[given]{\cos(x)} &\Rightarrow &f^{(9)}(0) = \answer[given]{1}\\
    \end{array}
    \]
    Since a repeating pattern has emerged, we see that the Maclaurin
    series for $\sin(x)$ is:
    \[
    x-\frac{x^3}{3!}+\frac{x^5}{5!}-\frac{x^7}{7!}+\cdots = \sum_{n=0}^\infty \frac{(-1)^{n+1}}{(2n+1)!} x^{2n+1}
    \]
  \end{explanation}
\end{example}

\end{document}
