\documentclass{ximera}

%\usepackage{todonotes}

\newcommand{\todo}{}


\graphicspath{
{./}
{../functionsOfSeveralVariables/}
{../normalVectors/}
{../lagrangeMultipliers/}
}


\usepackage{tkz-euclide}
\tikzset{>=stealth} %% cool arrow head
\tikzset{shorten <>/.style={ shorten >=#1, shorten <=#1 } } %% allows shorter vectors

\usetikzlibrary{backgrounds} %% for boxes around graphs
\usetikzlibrary{shapes,positioning}  %% Clouds and stars
\usetikzlibrary{matrix} %% for matrix
\usepgfplotslibrary{polar} %% for polar plots
\usetkzobj{all}
\usepackage[makeroom]{cancel} %% for strike outs
%\usepackage{mathtools} %% for pretty underbrace % Breaks Ximera
\usepackage{multicol}





\usepackage{array}
\setlength{\extrarowheight}{+.1cm}   
\newdimen\digitwidth
\settowidth\digitwidth{9}
\def\divrule#1#2{
\noalign{\moveright#1\digitwidth
\vbox{\hrule width#2\digitwidth}}}





\newcommand{\RR}{\mathbb R}
\newcommand{\R}{\mathbb R}
\newcommand{\N}{\mathbb N}
\newcommand{\Z}{\mathbb Z}

\newcommand{\sage}{\textsf{SageMath}}


%\renewcommand{\d}{\,d\!}
\renewcommand{\d}{\mathop{}\!d}
\newcommand{\dd}[2][]{\frac{\d #1}{\d #2}}
\newcommand{\pp}[2][]{\frac{\partial #1}{\partial #2}}
\renewcommand{\l}{\ell}
\newcommand{\ddx}{\frac{d}{\d x}}

\newcommand{\zeroOverZero}{\ensuremath{\boldsymbol{\tfrac{0}{0}}}}
\newcommand{\inftyOverInfty}{\ensuremath{\boldsymbol{\tfrac{\infty}{\infty}}}}
\newcommand{\zeroOverInfty}{\ensuremath{\boldsymbol{\tfrac{0}{\infty}}}}
\newcommand{\zeroTimesInfty}{\ensuremath{\small\boldsymbol{0\cdot \infty}}}
\newcommand{\inftyMinusInfty}{\ensuremath{\small\boldsymbol{\infty - \infty}}}
\newcommand{\oneToInfty}{\ensuremath{\boldsymbol{1^\infty}}}
\newcommand{\zeroToZero}{\ensuremath{\boldsymbol{0^0}}}
\newcommand{\inftyToZero}{\ensuremath{\boldsymbol{\infty^0}}}



\newcommand{\numOverZero}{\ensuremath{\boldsymbol{\tfrac{\#}{0}}}}
\newcommand{\dfn}{\textbf}
%\newcommand{\unit}{\,\mathrm}
\newcommand{\unit}{\mathop{}\!\mathrm}
\newcommand{\eval}[1]{\bigg[ #1 \bigg]}
\newcommand{\seq}[1]{\left( #1 \right)}
\renewcommand{\epsilon}{\varepsilon}
\renewcommand{\iff}{\Leftrightarrow}

\DeclareMathOperator{\arccot}{arccot}
\DeclareMathOperator{\arcsec}{arcsec}
\DeclareMathOperator{\arccsc}{arccsc}
\DeclareMathOperator{\si}{Si}
\DeclareMathOperator{\proj}{\vec{proj}}
\DeclareMathOperator{\scal}{scal}
\DeclareMathOperator{\sign}{sign}


%% \newcommand{\tightoverset}[2]{% for arrow vec
%%   \mathop{#2}\limits^{\vbox to -.5ex{\kern-0.75ex\hbox{$#1$}\vss}}}
\newcommand{\arrowvec}{\overrightarrow}
%\renewcommand{\vec}[1]{\arrowvec{\mathbf{#1}}}
\renewcommand{\vec}{\mathbf}
\newcommand{\veci}{{\boldsymbol{\hat{\imath}}}}
\newcommand{\vecj}{{\boldsymbol{\hat{\jmath}}}}
\newcommand{\veck}{{\boldsymbol{\hat{k}}}}
\newcommand{\vecl}{\boldsymbol{\l}}
\newcommand{\utan}{\mathbf{\hat{t}}}
\newcommand{\unormal}{\mathbf{\hat{n}}}
\newcommand{\ubinormal}{\mathbf{\hat{b}}}

\newcommand{\dotp}{\bullet}
\newcommand{\cross}{\boldsymbol\times}
\newcommand{\grad}{\boldsymbol\nabla}
\newcommand{\divergence}{\grad\dotp}
\newcommand{\curl}{\grad\cross}
%\DeclareMathOperator{\divergence}{divergence}
%\DeclareMathOperator{\curl}[1]{\grad\cross #1}
\newcommand{\lto}{\mathop{\longrightarrow\,}\limits}


\colorlet{textColor}{black} 
\colorlet{background}{white}
\colorlet{penColor}{blue!50!black} % Color of a curve in a plot
\colorlet{penColor2}{red!50!black}% Color of a curve in a plot
\colorlet{penColor3}{red!50!blue} % Color of a curve in a plot
\colorlet{penColor4}{green!50!black} % Color of a curve in a plot
\colorlet{penColor5}{orange!80!black} % Color of a curve in a plot
\colorlet{fill1}{penColor!20} % Color of fill in a plot
\colorlet{fill2}{penColor2!20} % Color of fill in a plot
\colorlet{fillp}{fill1} % Color of positive area
\colorlet{filln}{penColor2!20} % Color of negative area
\colorlet{fill3}{penColor3!20} % Fill
\colorlet{fill4}{penColor4!20} % Fill
\colorlet{fill5}{penColor5!20} % Fill
\colorlet{gridColor}{gray!50} % Color of grid in a plot

\newcommand{\surfaceColor}{violet}
\newcommand{\surfaceColorTwo}{redyellow}
\newcommand{\sliceColor}{greenyellow}




\pgfmathdeclarefunction{gauss}{2}{% gives gaussian
  \pgfmathparse{1/(#2*sqrt(2*pi))*exp(-((x-#1)^2)/(2*#2^2))}%
}


%%%%%%%%%%%%%
%% Vectors
%%%%%%%%%%%%%

%% Simple horiz vectors
\renewcommand{\vector}[1]{\left\langle #1\right\rangle}


%% %% Complex Horiz Vectors with angle brackets
%% \makeatletter
%% \renewcommand{\vector}[2][ , ]{\left\langle%
%%   \def\nextitem{\def\nextitem{#1}}%
%%   \@for \el:=#2\do{\nextitem\el}\right\rangle%
%% }
%% \makeatother

%% %% Vertical Vectors
%% \def\vector#1{\begin{bmatrix}\vecListA#1,,\end{bmatrix}}
%% \def\vecListA#1,{\if,#1,\else #1\cr \expandafter \vecListA \fi}

%%%%%%%%%%%%%
%% End of vectors
%%%%%%%%%%%%%

%\newcommand{\fullwidth}{}
%\newcommand{\normalwidth}{}



%% makes a snazzy t-chart for evaluating functions
%\newenvironment{tchart}{\rowcolors{2}{}{background!90!textColor}\array}{\endarray}

%%This is to help with formatting on future title pages.
\newenvironment{sectionOutcomes}{}{} 



%% Flowchart stuff
%\tikzstyle{startstop} = [rectangle, rounded corners, minimum width=3cm, minimum height=1cm,text centered, draw=black]
%\tikzstyle{question} = [rectangle, minimum width=3cm, minimum height=1cm, text centered, draw=black]
%\tikzstyle{decision} = [trapezium, trapezium left angle=70, trapezium right angle=110, minimum width=3cm, minimum height=1cm, text centered, draw=black]
%\tikzstyle{question} = [rectangle, rounded corners, minimum width=3cm, minimum height=1cm,text centered, draw=black]
%\tikzstyle{process} = [rectangle, minimum width=3cm, minimum height=1cm, text centered, draw=black]
%\tikzstyle{decision} = [trapezium, trapezium left angle=70, trapezium right angle=110, minimum width=3cm, minimum height=1cm, text centered, draw=black]


\title[Dig-In:]{Curl and line integrals}

\begin{document}
\begin{abstract}
Green's Theorem is a fundamental theorem of calculus.
\end{abstract}
\maketitle

In this section we will learn the \textit{fundamental derivative} for
vector fields, as well as a new fundamental theorem of calculus.


\section{The curl of a vector field}

Calculus has taught us that knowing the derivative of a function
$f:\R\to\R$ can tell us important information about the function.  In
a similar way we have that seen that if we wish to understand a
function of several variables $F:\R^n\to\R$, then the gradient, $\grad
F$, contains similar useful information. But what if you have a vector
field
\[
\vec{F}:\R^n\to\R^n
\]
what is the natural analogue of a derivative in this setting? We will
give the answer when the vector field is two or three dimensional. You
can take another course to learn more about deritivatives of
$n$-dimensional vector fields.


\begin{definition}
  In two-dimensions, given a vector field $\vec{F}:\R^2\to \R^2$, where
  \[
  \vec{F}(x,y) = \vector{M(x,y),N(x,y)}
  \]
  the \dfn{curl} is given by
  \[
  \curl \vec F = \pp[N]{x}-\pp[M]{y}.
  \]
  In three-dimensions, given a vector field $\vec{F}:\R^3\to\R^3$< where
  \[
  \vec{F}(x,y,z) = \vector{U(x,y,z)),V(x,y,z)),W(x,y,z)}
  \]
  the \dfn{curl} is given by
  \begin{align*}
  \curl \vec F &= \det
  \begin{bmatrix}
    \veci & \vecj & \veck \\
    \pp{x} & \pp{y} & \pp{z}\\
    U & V & W
  \end{bmatrix}\\
  &= \veci\left(\pp[W]{y}-\pp[V]{z}\right)-
  \vecj\left(\pp[W]{x}-\pp[U]{z}\right)+
  \veck\left(\pp[V]{x}-\pp[U]{y}\right).
  \end{align*}
\end{definition}

\begin{question}
  In two dimensions $\curl\vec{F}$ is a
  \begin{multipleChoice}
    \choice[correct]{number.}
    \choice{vector.}
  \end{multipleChoice}
  \begin{question}
    In three dimensions $\curl\vec{F}$ is a
    \begin{multipleChoice}
      \choice{number.}
      \choice[correct]{vector.}
    \end{multipleChoice}
  \end{question}
\end{question}


\begin{question}
  Consider the vector field $\vec{F}(x,y) = \vector{-y,x}$. Compute:
  \[
  \curl\vec{F}(x,y) \begin{prompt}= \answer{2}\end{prompt}
  \]
  \begin{question}
    Consider the vector field $\vec{F}(x,y,z) = \vector{-z,x,y}$. Compute:
    \[
    \curl\vec{F}(x,y,z)   \begin{prompt}
      = \vector{\answer{1},\answer{1},\answer{1}}
    \end{prompt}
    \]
  \end{question}
\end{question}

Now for something you've seen before, but in a different form.

\begin{question}
  Let $F:\R^2\to\R$. Compute:
  \[
  \curl\grad F(x,y) \begin{prompt}= \answer{0}\end{prompt}
  \]
  \begin{question}
    Let $F:\R^3\to\R$. Compute:
    \[
    \curl\grad F(x,y,z) \begin{prompt}= \vector{\answer{0},\answer{0},\answer{0}}\end{prompt}
    \] 
  \end{question}
\end{question}

\begin{question}
  When $\curl\vec{F} = \vec{0}$, then you know:
  \begin{selectAll}
    \choice[correct]{$\vec{F}$ is a gradient field.}
    \choice[correct]{$\vec{F}$ is a conserttive field.}
    \choice[correct]{$\vec{F}:\R^3\to\R^3$.}
  \end{selectAll}
  \begin{feedback}
    You can be assured that $\vec{F}:\R^3\to\R^3$, since $\vec{0}$ is
    a vector. We only know a definition for curl in two and three
    dimensions; however, the two dimensional definition is a scalar,
    not a vector. So if $\curl\vec{F} = \vec{0}$, then $\vec{F}:\R^3\to\R^3$.
  \end{feedback}
\end{question}


\subsection{What does the curl measure?}


The curl of a vector field measures the rate that the direction of
field vectors ``twist'' as $x$ and $y$ change. Unfortunately, while we
can sometimes identify curl from a graph.

beachball?


\begin{question}
  Discrete example use field $\vector{0,x}$
\end{question}



\[
\{F:\R^2 \to \R\} \lto^{\grad} \{\vec{F}: \R^2 \to \R^2\} \lto^{\curl}
\{F:\R^2 \to \R\} 
\]



Recall,

\section{A new fundamental theorem of calculus}

\begin{theorem}[Green's Theorem]
  If $\vec{F}$ has continuous partial derivatives and $C$ is a
  boundray of a closed region $R$ and $\vec{p}(t)$ paramaterizes $C$
  in a counterclockwise direction with the interior on the left, then
  \[
  \int_R \curl\vec{F}\d A = \int_C \vec{F}\dotp\d\vec{p} 
  \]
\end{theorem}

\end{document}
