\documentclass{ximera}

%\usepackage{todonotes}

\newcommand{\todo}{}


\graphicspath{
{./}
{../functionsOfSeveralVariables/}
{../normalVectors/}
{../lagrangeMultipliers/}
}


\usepackage{tkz-euclide}
\tikzset{>=stealth} %% cool arrow head
\tikzset{shorten <>/.style={ shorten >=#1, shorten <=#1 } } %% allows shorter vectors

\usetikzlibrary{backgrounds} %% for boxes around graphs
\usetikzlibrary{shapes,positioning}  %% Clouds and stars
\usetikzlibrary{matrix} %% for matrix
\usepgfplotslibrary{polar} %% for polar plots
\usetkzobj{all}
\usepackage[makeroom]{cancel} %% for strike outs
%\usepackage{mathtools} %% for pretty underbrace % Breaks Ximera
\usepackage{multicol}





\usepackage{array}
\setlength{\extrarowheight}{+.1cm}   
\newdimen\digitwidth
\settowidth\digitwidth{9}
\def\divrule#1#2{
\noalign{\moveright#1\digitwidth
\vbox{\hrule width#2\digitwidth}}}





\newcommand{\RR}{\mathbb R}
\newcommand{\R}{\mathbb R}
\newcommand{\N}{\mathbb N}
\newcommand{\Z}{\mathbb Z}

\newcommand{\sage}{\textsf{SageMath}}


%\renewcommand{\d}{\,d\!}
\renewcommand{\d}{\mathop{}\!d}
\newcommand{\dd}[2][]{\frac{\d #1}{\d #2}}
\newcommand{\pp}[2][]{\frac{\partial #1}{\partial #2}}
\renewcommand{\l}{\ell}
\newcommand{\ddx}{\frac{d}{\d x}}

\newcommand{\zeroOverZero}{\ensuremath{\boldsymbol{\tfrac{0}{0}}}}
\newcommand{\inftyOverInfty}{\ensuremath{\boldsymbol{\tfrac{\infty}{\infty}}}}
\newcommand{\zeroOverInfty}{\ensuremath{\boldsymbol{\tfrac{0}{\infty}}}}
\newcommand{\zeroTimesInfty}{\ensuremath{\small\boldsymbol{0\cdot \infty}}}
\newcommand{\inftyMinusInfty}{\ensuremath{\small\boldsymbol{\infty - \infty}}}
\newcommand{\oneToInfty}{\ensuremath{\boldsymbol{1^\infty}}}
\newcommand{\zeroToZero}{\ensuremath{\boldsymbol{0^0}}}
\newcommand{\inftyToZero}{\ensuremath{\boldsymbol{\infty^0}}}



\newcommand{\numOverZero}{\ensuremath{\boldsymbol{\tfrac{\#}{0}}}}
\newcommand{\dfn}{\textbf}
%\newcommand{\unit}{\,\mathrm}
\newcommand{\unit}{\mathop{}\!\mathrm}
\newcommand{\eval}[1]{\bigg[ #1 \bigg]}
\newcommand{\seq}[1]{\left( #1 \right)}
\renewcommand{\epsilon}{\varepsilon}
\renewcommand{\iff}{\Leftrightarrow}

\DeclareMathOperator{\arccot}{arccot}
\DeclareMathOperator{\arcsec}{arcsec}
\DeclareMathOperator{\arccsc}{arccsc}
\DeclareMathOperator{\si}{Si}
\DeclareMathOperator{\proj}{\vec{proj}}
\DeclareMathOperator{\scal}{scal}
\DeclareMathOperator{\sign}{sign}


%% \newcommand{\tightoverset}[2]{% for arrow vec
%%   \mathop{#2}\limits^{\vbox to -.5ex{\kern-0.75ex\hbox{$#1$}\vss}}}
\newcommand{\arrowvec}{\overrightarrow}
%\renewcommand{\vec}[1]{\arrowvec{\mathbf{#1}}}
\renewcommand{\vec}{\mathbf}
\newcommand{\veci}{{\boldsymbol{\hat{\imath}}}}
\newcommand{\vecj}{{\boldsymbol{\hat{\jmath}}}}
\newcommand{\veck}{{\boldsymbol{\hat{k}}}}
\newcommand{\vecl}{\boldsymbol{\l}}
\newcommand{\utan}{\mathbf{\hat{t}}}
\newcommand{\unormal}{\mathbf{\hat{n}}}
\newcommand{\ubinormal}{\mathbf{\hat{b}}}

\newcommand{\dotp}{\bullet}
\newcommand{\cross}{\boldsymbol\times}
\newcommand{\grad}{\boldsymbol\nabla}
\newcommand{\divergence}{\grad\dotp}
\newcommand{\curl}{\grad\cross}
%\DeclareMathOperator{\divergence}{divergence}
%\DeclareMathOperator{\curl}[1]{\grad\cross #1}
\newcommand{\lto}{\mathop{\longrightarrow\,}\limits}


\colorlet{textColor}{black} 
\colorlet{background}{white}
\colorlet{penColor}{blue!50!black} % Color of a curve in a plot
\colorlet{penColor2}{red!50!black}% Color of a curve in a plot
\colorlet{penColor3}{red!50!blue} % Color of a curve in a plot
\colorlet{penColor4}{green!50!black} % Color of a curve in a plot
\colorlet{penColor5}{orange!80!black} % Color of a curve in a plot
\colorlet{fill1}{penColor!20} % Color of fill in a plot
\colorlet{fill2}{penColor2!20} % Color of fill in a plot
\colorlet{fillp}{fill1} % Color of positive area
\colorlet{filln}{penColor2!20} % Color of negative area
\colorlet{fill3}{penColor3!20} % Fill
\colorlet{fill4}{penColor4!20} % Fill
\colorlet{fill5}{penColor5!20} % Fill
\colorlet{gridColor}{gray!50} % Color of grid in a plot

\newcommand{\surfaceColor}{violet}
\newcommand{\surfaceColorTwo}{redyellow}
\newcommand{\sliceColor}{greenyellow}




\pgfmathdeclarefunction{gauss}{2}{% gives gaussian
  \pgfmathparse{1/(#2*sqrt(2*pi))*exp(-((x-#1)^2)/(2*#2^2))}%
}


%%%%%%%%%%%%%
%% Vectors
%%%%%%%%%%%%%

%% Simple horiz vectors
\renewcommand{\vector}[1]{\left\langle #1\right\rangle}


%% %% Complex Horiz Vectors with angle brackets
%% \makeatletter
%% \renewcommand{\vector}[2][ , ]{\left\langle%
%%   \def\nextitem{\def\nextitem{#1}}%
%%   \@for \el:=#2\do{\nextitem\el}\right\rangle%
%% }
%% \makeatother

%% %% Vertical Vectors
%% \def\vector#1{\begin{bmatrix}\vecListA#1,,\end{bmatrix}}
%% \def\vecListA#1,{\if,#1,\else #1\cr \expandafter \vecListA \fi}

%%%%%%%%%%%%%
%% End of vectors
%%%%%%%%%%%%%

%\newcommand{\fullwidth}{}
%\newcommand{\normalwidth}{}



%% makes a snazzy t-chart for evaluating functions
%\newenvironment{tchart}{\rowcolors{2}{}{background!90!textColor}\array}{\endarray}

%%This is to help with formatting on future title pages.
\newenvironment{sectionOutcomes}{}{} 



%% Flowchart stuff
%\tikzstyle{startstop} = [rectangle, rounded corners, minimum width=3cm, minimum height=1cm,text centered, draw=black]
%\tikzstyle{question} = [rectangle, minimum width=3cm, minimum height=1cm, text centered, draw=black]
%\tikzstyle{decision} = [trapezium, trapezium left angle=70, trapezium right angle=110, minimum width=3cm, minimum height=1cm, text centered, draw=black]
%\tikzstyle{question} = [rectangle, rounded corners, minimum width=3cm, minimum height=1cm,text centered, draw=black]
%\tikzstyle{process} = [rectangle, minimum width=3cm, minimum height=1cm, text centered, draw=black]
%\tikzstyle{decision} = [trapezium, trapezium left angle=70, trapezium right angle=110, minimum width=3cm, minimum height=1cm, text centered, draw=black]


\title[Dig-In:]{Line integrals}

\begin{document}
\begin{abstract}
We accumulate vectors along a path.
\end{abstract}
\maketitle

In this section we introduce a new type of integrals, \textit{line
  integrals} also known as \textit{path integrals}. Let's start with
the definition of a \textit{line integral}:


\begin{definition}
Let $\vec{F}:\R^2\to\R^2$ be a vector field, $\vec{p}:\R\to\R^2$ be a
vector valued function graphing a curve $C$ as $t$ runs from $a$ to
$b$,
\begin{align*}
  \vec{F}(x,y) &= \vector{M(x,y), N(x,y)}\\
  \vec{p}(t) &= \vector{x(t),y(t)}.
\end{align*}
A \dfn{line integral} is an integral of the form:
\[
\int_C \vec{F}\dotp \d \vec{p} = \int_C \vector{M,N}\dotp\vector{\d x,\d y}
\]
Since $\d x = x'(t)\d t$ and $\d y = y'(t)\d t$, we may write  
\begin{align*}
  &= \int_C M\d x + N\d y\\
  &= \int_a^b M(x(t))\cdot x'(t) \d t + N(y(t))\cdot  y'(t) \d t
\end{align*}
\end{definition}

\begin{question}
  Which of the following are line integrals?
  \begin{multipleChoice}
    \choice{$\int_R (x^2+y^2) \d A$}
    \choice[correct]{$\int_C -y\d x + x\d y$}
    \choice{$\int_3^4\int_2^3 \ln(xy) \d x \d y$}
    \choice[correct]{$\int_0^{2\pi} \vector{-\sin(t),\cos(t)}\dotp\vector{-\sin(t),\cos(t)}\d t$}
  \end{multipleChoice}
\end{question}

Read on to learn the meaning of this new integral.


\section{What do line integrals measure?}

A line integral measures the flow of a vector field along a path. The
basic idea is that there is some vector field given by $\vec{F}$:
\begin{image}
\begin{tikzpicture}
      \begin{axis}%
        [hide axis,
          ymin=-4.5,ymax=2.5,
          xmin=-6,xmax=5.5,
        ]
        \addplot[penColor,thick, ->] coordinates{(-6,2) (-.5,2)};
        \addplot[penColor,thick, ->] coordinates{(0,2) (5.5,2)};

        \addplot[penColor,thick, ->] coordinates{(-6,1) (-2.5,1)};
        \addplot[penColor,thick, ->] coordinates{(-2,1) (1.5,1)};
        \addplot[penColor,thick, ->] coordinates{(2,1) (5.5,1)};

        \addplot[penColor,thick, ->] coordinates{(-6,0) (-3.5,0)};
        \addplot[penColor,thick, ->] coordinates{(-3,0) (-.5,0)};
        \addplot[penColor,thick, ->] coordinates{(0,0) (2.5,0)};
        \addplot[penColor,thick, ->] coordinates{(3,0) (5.5,0)};

        \addplot[penColor,thick, ->] coordinates{(-6,-1) (-4.5,-1)};
        \addplot[penColor,thick, ->] coordinates{(-4,-1) (-2.5,-1)};
        \addplot[penColor,thick, ->] coordinates{(-2,-1) (-.5,-1)};
        \addplot[penColor,thick, ->] coordinates{(0,-1) (1.5,-1)};
        \addplot[penColor,thick, ->] coordinates{(2,-1) (3.5,-1)};
        \addplot[penColor,thick, ->] coordinates{(4,-1) (5.5,-1)};

        \addplot[penColor,thick, ->] coordinates{(-6,-2) (-5.5,-2)};
        \addplot[penColor,thick, ->] coordinates{(-5,-2) (-4.5,-2)};
        \addplot[penColor,thick, ->] coordinates{(-4,-2) (-3.5,-2)};
        \addplot[penColor,thick, ->] coordinates{(-3,-2) (-2.5,-2)};
        \addplot[penColor,thick, ->] coordinates{(-2,-2) (-1.5,-2)};
        \addplot[penColor,thick, ->] coordinates{(-1,-2) (-.5,-2)};
        \addplot[penColor,thick, ->] coordinates{(0,-2) (.5,-2)};
        \addplot[penColor,thick, ->] coordinates{(1,-2) (1.5,-2)};
        \addplot[penColor,thick, ->] coordinates{(2,-2) (2.5,-2)};
        \addplot[penColor,thick, ->] coordinates{(3,-2) (3.5,-2)};
        \addplot[penColor,thick, ->] coordinates{(4,-2) (4.5,-2)};
        \addplot[penColor,thick, ->] coordinates{(5,-2) (5.5,-2)};
        \node[inner sep=0pt,text width=8cm,right,scale=.85] at (axis cs:-6,-3.5)
             {\footnotesize One should imagine a vector at
               \textbf{every} point. We'll assume that the magnitudes
               of the vectors are constant along horizontal lines.};
      \end{axis}
 \end{tikzpicture}
\end{image}


Now we add an oriented path $C$ that is parameterized by $\vec{p}(t) =
\vector{x(t),y(t)}$. This can be thought of as a path that an object
takes through the field. To reason via a specific example, we'll add a
path below:
\begin{image}
  \begin{tikzpicture}
    \begin{axis}%
      [hide axis,
	ymin=-3,ymax=2.5,
	xmin=-6,xmax=5.5,
	]
      \addplot[penColor,thick, ->] coordinates{(-6,2) (-.5,2)};
      \addplot[penColor,thick, ->] coordinates{(0,2) (5.5,2)};
      
      \addplot[penColor,thick, ->] coordinates{(-6,1) (-2.5,1)};
      \addplot[penColor,thick, ->] coordinates{(-2,1) (1.5,1)};
      \addplot[penColor,thick, ->] coordinates{(2,1) (5.5,1)};
      
      \addplot[penColor,thick, ->] coordinates{(-6,0) (-3.5,0)};
      \addplot[penColor,thick, ->] coordinates{(-3,0) (-.5,0)};
      \addplot[penColor,thick, ->] coordinates{(0,0) (2.5,0)};
      \addplot[penColor,thick, ->] coordinates{(3,0) (5.5,0)};
      
      \addplot[penColor,thick, ->] coordinates{(-6,-1) (-4.5,-1)};
      \addplot[penColor,thick, ->] coordinates{(-4,-1) (-2.5,-1)};
      \addplot[penColor,thick, ->] coordinates{(-2,-1) (-.5,-1)};
      \addplot[penColor,thick, ->] coordinates{(0,-1) (1.5,-1)};
      \addplot[penColor,thick, ->] coordinates{(2,-1) (3.5,-1)};
      \addplot[penColor,thick, ->] coordinates{(4,-1) (5.5,-1)};
      
      \addplot[penColor,thick, ->] coordinates{(-6,-2) (-5.5,-2)};
      \addplot[penColor,thick, ->] coordinates{(-5,-2) (-4.5,-2)};
      \addplot[penColor,thick, ->] coordinates{(-4,-2) (-3.5,-2)};
      \addplot[penColor,thick, ->] coordinates{(-3,-2) (-2.5,-2)};
      \addplot[penColor,thick, ->] coordinates{(-2,-2) (-1.5,-2)};
      \addplot[penColor,thick, ->] coordinates{(-1,-2) (-.5,-2)};
      \addplot[penColor,thick, ->] coordinates{(0,-2) (.5,-2)};
      \addplot[penColor,thick, ->] coordinates{(1,-2) (1.5,-2)};
      \addplot[penColor,thick, ->] coordinates{(2,-2) (2.5,-2)};
      \addplot[penColor,thick, ->] coordinates{(3,-2) (3.5,-2)};
      \addplot[penColor,thick, ->] coordinates{(4,-2) (4.5,-2)};
      \addplot[penColor,thick, ->] coordinates{(5,-2) (5.5,-2)};
        
      \addplot[penColor2,ultra thick] coordinates{
        (-3,2.3) (3,2.3)
        (3,-2.3) (-3,-2.3)
      };
      \addplot[penColor2,ultra thick, ->] coordinates{(-3,2.3) (0,2.3)};
      \addplot[penColor2,ultra thick, ->] coordinates{(3,2.3) (3,0)};
      \addplot[penColor2,ultra thick, ->] coordinates{(3,-2.3) (0,-2.3)};
    \end{axis}
  \end{tikzpicture}
\end{image}

To figure out if the flow of the vector field is ``with'' the
direction of the path, we use the dot product:
\[
\underbrace{\vec{F}(x(t),y(t))}_{\text{direction of field}} \dotp \underbrace{\vector{x'(t) \d t,y'(t) \d t}}_{\text{direction of path}}
\]
\begin{question}
When the direction of the field and the direction of the path are in
alignment, the dot product is\dots
\begin{onlineOnly}
  \begin{multipleChoice}
    \choice[correct]{positive}
    \choice{zero}
    \choice{negative}
  \end{multipleChoice}
\end{onlineOnly}
\begin{question}
  When the direction of the
  field and the direction of the path are orthogonal, the dot product is\dots
  \begin{onlineOnly}
  \begin{multipleChoice}
    \choice{positive}
    \choice[correct]{zero}
    \choice{negative}
  \end{multipleChoice}
\end{onlineOnly}
\begin{question}
  When the direction of the field and the direction of the path are in
  opposite direction, the dot product is\dots
  \begin{onlineOnly}
  \begin{multipleChoice}
    \choice{positive}
    \choice{zero}
    \choice[correct]{negative}
  \end{multipleChoice}
  \end{onlineOnly}
\end{question}
\end{question}
\end{question}
Integrating over the path sums these infinitesimal measurements:
\[
\vec{F}(x(t),y(t))\dotp (x'(t),y'(t)) \d t = \vec{F} \dotp\d\vec{p}
\]
Thus the line integral
\[
\int_C \vec{F}\dotp \d \vec{p}
\]
measures the flow of a field along a path. In particular, if the value
of the line integral is positive, then the flow is with the path; if
the value is negative, then the flow is against the path.



\begin{question}
  Consider the following vector field along with a (directed) curve
  $C$.
  \begin{image}
    \begin{tikzpicture}
      \begin{axis}%
        [hide axis,
	  ymin=-3,ymax=2.5,
	  xmin=-6,xmax=5.5,
	]
        \addplot[penColor,thick, ->] coordinates{(-6,2) (-.5,2)};
        \addplot[penColor,thick, ->] coordinates{(0,2) (5.5,2)};

        \addplot[penColor,thick, ->] coordinates{(-6,1) (-2.5,1)};
        \addplot[penColor,thick, ->] coordinates{(-2,1) (1.5,1)};
        \addplot[penColor,thick, ->] coordinates{(2,1) (5.5,1)};

        \addplot[penColor,thick, ->] coordinates{(-6,0) (-3.5,0)};
        \addplot[penColor,thick, ->] coordinates{(-3,0) (-.5,0)};
        \addplot[penColor,thick, ->] coordinates{(0,0) (2.5,0)};
        \addplot[penColor,thick, ->] coordinates{(3,0) (5.5,0)};

        \addplot[penColor,thick, ->] coordinates{(-6,-1) (-4.5,-1)};
        \addplot[penColor,thick, ->] coordinates{(-4,-1) (-2.5,-1)};
        \addplot[penColor,thick, ->] coordinates{(-2,-1) (-.5,-1)};
        \addplot[penColor,thick, ->] coordinates{(0,-1) (1.5,-1)};
        \addplot[penColor,thick, ->] coordinates{(2,-1) (3.5,-1)};
        \addplot[penColor,thick, ->] coordinates{(4,-1) (5.5,-1)};
        
        \addplot[penColor,thick, ->] coordinates{(-6,-2) (-5.5,-2)};
        \addplot[penColor,thick, ->] coordinates{(-5,-2) (-4.5,-2)};
        \addplot[penColor,thick, ->] coordinates{(-4,-2) (-3.5,-2)};
        \addplot[penColor,thick, ->] coordinates{(-3,-2) (-2.5,-2)};
        \addplot[penColor,thick, ->] coordinates{(-2,-2) (-1.5,-2)};
        \addplot[penColor,thick, ->] coordinates{(-1,-2) (-.5,-2)};
        \addplot[penColor,thick, ->] coordinates{(0,-2) (.5,-2)};
        \addplot[penColor,thick, ->] coordinates{(1,-2) (1.5,-2)};
        \addplot[penColor,thick, ->] coordinates{(2,-2) (2.5,-2)};
        \addplot[penColor,thick, ->] coordinates{(3,-2) (3.5,-2)};
        \addplot[penColor,thick, ->] coordinates{(4,-2) (4.5,-2)};
        \addplot[penColor,thick, ->] coordinates{(5,-2) (5.5,-2)};
        
        \addplot[penColor2,ultra thick] coordinates{
          (-3,2.3) (3,2.3)
          (3,-2.3) (-3,-2.3)
        };
        \addplot[penColor2,ultra thick, ->] coordinates{(-3,2.3) (0,2.3)};
        \addplot[penColor2,ultra thick, ->] coordinates{(3,2.3) (3,0)};
        \addplot[penColor2,ultra thick, ->] coordinates{(3,-2.3) (0,-2.3)};
      \end{axis}
    \end{tikzpicture}
  \end{image}
  Do you expect 
  \[
  \int_C \vec{F}\dotp \d \vec{p} 
  \]
  to be positive, zero, or negative?
  \begin{onlineOnly}
  \begin{multipleChoice}
    \choice[correct]{positive}
    \choice{zero}
    \choice{negative}
  \end{multipleChoice}
  \begin{hint}
    We can think about this better if we break the path into pieces:
    $C_1$, $C_2$, $C_3$.
      \begin{image}
    \begin{tikzpicture}
      \begin{axis}%
        [hide axis,
	  ymin=-3,ymax=2.5,
	  xmin=-6,xmax=5.5,
	]
        \addplot[penColor,thick, ->] coordinates{(-6,2) (-.5,2)};
        \addplot[penColor,thick, ->] coordinates{(0,2) (5.5,2)};

        \addplot[penColor,thick, ->] coordinates{(-6,1) (-2.5,1)};
        \addplot[penColor,thick, ->] coordinates{(-2,1) (1.5,1)};
        \addplot[penColor,thick, ->] coordinates{(2,1) (5.5,1)};

        \addplot[penColor,thick, ->] coordinates{(-6,0) (-3.5,0)};
        \addplot[penColor,thick, ->] coordinates{(-3,0) (-.5,0)};
        \addplot[penColor,thick, ->] coordinates{(0,0) (2.5,0)};
        \addplot[penColor,thick, ->] coordinates{(3,0) (5.5,0)};

        \addplot[penColor,thick, ->] coordinates{(-6,-1) (-4.5,-1)};
        \addplot[penColor,thick, ->] coordinates{(-4,-1) (-2.5,-1)};
        \addplot[penColor,thick, ->] coordinates{(-2,-1) (-.5,-1)};
        \addplot[penColor,thick, ->] coordinates{(0,-1) (1.5,-1)};
        \addplot[penColor,thick, ->] coordinates{(2,-1) (3.5,-1)};
        \addplot[penColor,thick, ->] coordinates{(4,-1) (5.5,-1)};
        
        \addplot[penColor,thick, ->] coordinates{(-6,-2) (-5.5,-2)};
        \addplot[penColor,thick, ->] coordinates{(-5,-2) (-4.5,-2)};
        \addplot[penColor,thick, ->] coordinates{(-4,-2) (-3.5,-2)};
        \addplot[penColor,thick, ->] coordinates{(-3,-2) (-2.5,-2)};
        \addplot[penColor,thick, ->] coordinates{(-2,-2) (-1.5,-2)};
        \addplot[penColor,thick, ->] coordinates{(-1,-2) (-.5,-2)};
        \addplot[penColor,thick, ->] coordinates{(0,-2) (.5,-2)};
        \addplot[penColor,thick, ->] coordinates{(1,-2) (1.5,-2)};
        \addplot[penColor,thick, ->] coordinates{(2,-2) (2.5,-2)};
        \addplot[penColor,thick, ->] coordinates{(3,-2) (3.5,-2)};
        \addplot[penColor,thick, ->] coordinates{(4,-2) (4.5,-2)};
        \addplot[penColor,thick, ->] coordinates{(5,-2) (5.5,-2)};
        
        \addplot[penColor2,ultra thick] coordinates{
          (-3,2.3) (3,2.3)
        };
        \addplot[penColor4,ultra thick] coordinates{
          (3,2.3)
          (3,-2.3) 
        };
        \addplot[penColor5,ultra thick] coordinates{
          (3,-2.3) (-3,-2.3)
          };
        \addplot[penColor2,ultra thick, ->] coordinates{(-3,2.3) (0,2.3)};
        \addplot[penColor4,ultra thick, ->] coordinates{(3,2.3) (3,0)};
        \addplot[penColor5,ultra thick, ->] coordinates{(3,-2.3) (0,-2.3)};

        \node[below,penColor2] at (axis cs: 0,2) {$C_1$};
        \node[above right,penColor4] at (axis cs: 3,0) {$C_2$};
        \node[above,penColor5] at (axis cs: 0,-2) {$C_3$};
      \end{axis}
    \end{tikzpicture}
      \end{image}
      We can see that the vectors are flowing with the direction of
      $C_1$. Note that the the magnitude of these vectors is large, so
      this contributs a large postive value to our integral.

      The field vectors are orthogonal to the direction of the path
      $C_2$. So this part contributes nothing to the integral.

      The field vectors are flowing against the direction of
      $C_3$. However, their magnitude is much less than the vectors
      that flowed with $C_1$. So this contributes a small negative
      value to our integral.
  \end{hint}
  \end{onlineOnly}
\end{question}






\begin{question}
  Consider the following vector field along with a (directed) curve
  $C$.
  \begin{image}
    \begin{tikzpicture}
      \begin{axis}%
        [hide axis,
	  ymin=-3,ymax=2.5,
	  xmin=-6,xmax=5.5,
	]
        \addplot[penColor,thick, ->] coordinates{(-4,0) (-4,2.5)};
        \addplot[penColor,thick, ->] coordinates{(-3,0) (-3,2.5)};
        \addplot[penColor,thick, ->] coordinates{(-2,0) (-2,2.5)};
        \addplot[penColor,thick, ->] coordinates{(-1,0) (-1,2.5)};
        \addplot[penColor,thick, ->] coordinates{(0,0) (0,2.5)};
        \addplot[penColor,thick, ->] coordinates{(1,0) (1,2.5)};
        \addplot[penColor,thick, ->] coordinates{(2,0) (2,2.5)};
        \addplot[penColor,thick, ->] coordinates{(3,0) (3,2.5)};
        \addplot[penColor,thick, ->] coordinates{(4,0) (4,2.5)};

        \addplot[penColor,thick, ->] coordinates{(-4,-2) (-4,-.25)};
        \addplot[penColor,thick, ->] coordinates{(-3,-2) (-3,-.25)};
        \addplot[penColor,thick, ->] coordinates{(-2,-2) (-2,-.25)};
        \addplot[penColor,thick, ->] coordinates{(-1,-2) (-1,-.25)};
        \addplot[penColor,thick, ->] coordinates{(0,-2) (0,-.25)};
        \addplot[penColor,thick, ->] coordinates{(1,-2) (1,-.25)};
        \addplot[penColor,thick, ->] coordinates{(2,-2) (2,-.25)};
        \addplot[penColor,thick, ->] coordinates{(3,-2) (3,-.25)};
        \addplot[penColor,thick, ->] coordinates{(4,-2) (4,-.25)};

        \addplot[penColor,thick, ->] coordinates{(-4,-3) (-4,-2.2)};
        \addplot[penColor,thick, ->] coordinates{(-3,-3) (-3,-2.2)};
        \addplot[penColor,thick, ->] coordinates{(-2,-3) (-2,-2.2)};
        \addplot[penColor,thick, ->] coordinates{(-1,-3) (-1,-2.2)};
        \addplot[penColor,thick, ->] coordinates{(0,-3) (0,-2.2)};
        \addplot[penColor,thick, ->] coordinates{(1,-3) (1,-2.2)};
        \addplot[penColor,thick, ->] coordinates{(2,-3) (2,-2.2)};
        \addplot[penColor,thick, ->] coordinates{(3,-3) (3,-2.2)};
        \addplot[penColor,thick, ->] coordinates{(4,-3) (4,-2.2)};

        
        
        
        \addplot[penColor2,ultra thick] coordinates{
          (-3,1.7) (3.5,1.7)
          (3.5,-2.5) (-3,-2.5)
        };
        \addplot[penColor2,ultra thick, ->] coordinates{(-3,1.7) (0,1.7)};
        \addplot[penColor2,ultra thick, ->] coordinates{(3.5,1.7) (3.5,0)};
        \addplot[penColor2,ultra thick, ->] coordinates{(3.5,-2.5) (0,-2.5)};
      \end{axis}
    \end{tikzpicture}
  \end{image}
  Do you expect 
  \[
  \int_C \vec{F}\dotp \d \vec{p} 
  \]
  to be positive, zero, or negative?
  \begin{onlineOnly}
  \begin{multipleChoice}
    \choice{positive}
    \choice{zero}
    \choice[correct]{negative}
  \end{multipleChoice}
  \begin{hint}
    We can think about this better if we break the path into pieces:
    $C_1$, $C_2$, $C_3$.
    \begin{image}
      \begin{tikzpicture}
        \begin{axis}%
          [hide axis,
	    ymin=-3,ymax=2.5,
	    xmin=-6,xmax=5.5,
	  ]
          \addplot[penColor,thick, ->] coordinates{(-4,0) (-4,2.5)};
          \addplot[penColor,thick, ->] coordinates{(-3,0) (-3,2.5)};
          \addplot[penColor,thick, ->] coordinates{(-2,0) (-2,2.5)};
          \addplot[penColor,thick, ->] coordinates{(-1,0) (-1,2.5)};
          \addplot[penColor,thick, ->] coordinates{(0,0) (0,2.5)};
          \addplot[penColor,thick, ->] coordinates{(1,0) (1,2.5)};
          \addplot[penColor,thick, ->] coordinates{(2,0) (2,2.5)};
          \addplot[penColor,thick, ->] coordinates{(3,0) (3,2.5)};
          \addplot[penColor,thick, ->] coordinates{(4,0) (4,2.5)};
          
          \addplot[penColor,thick, ->] coordinates{(-4,-2) (-4,-.25)};
          \addplot[penColor,thick, ->] coordinates{(-3,-2) (-3,-.25)};
          \addplot[penColor,thick, ->] coordinates{(-2,-2) (-2,-.25)};
          \addplot[penColor,thick, ->] coordinates{(-1,-2) (-1,-.25)};
          \addplot[penColor,thick, ->] coordinates{(0,-2) (0,-.25)};
          \addplot[penColor,thick, ->] coordinates{(1,-2) (1,-.25)};
          \addplot[penColor,thick, ->] coordinates{(2,-2) (2,-.25)};
          \addplot[penColor,thick, ->] coordinates{(3,-2) (3,-.25)};
          \addplot[penColor,thick, ->] coordinates{(4,-2) (4,-.25)};
          
          \addplot[penColor,thick, ->] coordinates{(-4,-3) (-4,-2.2)};
          \addplot[penColor,thick, ->] coordinates{(-3,-3) (-3,-2.2)};
          \addplot[penColor,thick, ->] coordinates{(-2,-3) (-2,-2.2)};
          \addplot[penColor,thick, ->] coordinates{(-1,-3) (-1,-2.2)};
          \addplot[penColor,thick, ->] coordinates{(0,-3) (0,-2.2)};
          \addplot[penColor,thick, ->] coordinates{(1,-3) (1,-2.2)};
          \addplot[penColor,thick, ->] coordinates{(2,-3) (2,-2.2)};
          \addplot[penColor,thick, ->] coordinates{(3,-3) (3,-2.2)};
          \addplot[penColor,thick, ->] coordinates{(4,-3) (4,-2.2)};
          
          \addplot[penColor2,ultra thick] coordinates{
            (-3,1.7) (3.5,1.7)
          };

          \addplot[penColor4,ultra thick] coordinates{
            (3.5,1.7)
            (3.5,-2.5) 
          };

          \addplot[penColor5,ultra thick] coordinates{
            (3.5,-2.5) (-3,-2.5)
          };
         
          \addplot[penColor2,ultra thick, ->] coordinates{(-3,1.7) (0,1.7)};
          \addplot[penColor4,ultra thick, ->] coordinates{(3.5,1.7) (3.5,0)};
          \addplot[penColor5,ultra thick, ->] coordinates{(3.5,-2.5) (0,-2.5)};

          \node[above left,penColor2] at (axis cs: 0,1.7) {$C_1$};
          \node[above right,penColor4] at (axis cs: 2,0) {$C_2$};
          \node[above left,penColor5] at (axis cs: 0,-2.3) {$C_3$};
        \end{axis}
      \end{tikzpicture}
    \end{image}
    The field vectors are orthogonal to the direction of the path
    $C_1$. So this part contributes nothing to the integral.

    The field vectors are flowing against the direction of $C_2$. This
    contributes a negative value to our integral.

    The field vectors are again orthogonal to the direction of the
    path $C_3$. So this part contributes nothing to the integral.
  \end{hint}
  \end{onlineOnly}
\end{question}




\begin{question}
  Consider the following vector field along with a (directed) curve
  $C$.
  \begin{image}
    \begin{tikzpicture}
      \begin{axis}%
        [hide axis,
	  ymin=-3,ymax=2.5,
	  xmin=-6,xmax=5.5,
	]
        \addplot[penColor,thick, ->] coordinates{(-6,2) (-.5,2)};
        \addplot[penColor,thick, ->] coordinates{(0,2) (5.5,2)};

        \addplot[penColor,thick, ->] coordinates{(-6,1) (-2.5,1)};
        \addplot[penColor,thick, ->] coordinates{(-2,1) (1.5,1)};
        \addplot[penColor,thick, ->] coordinates{(2,1) (5.5,1)};

        \addplot[penColor,thick, ->] coordinates{(-6,0) (-3.5,0)};
        \addplot[penColor,thick, ->] coordinates{(-3,0) (-.5,0)};
        \addplot[penColor,thick, ->] coordinates{(0,0) (2.5,0)};
        \addplot[penColor,thick, ->] coordinates{(3,0) (5.5,0)};

        \addplot[penColor,thick, ->] coordinates{(-6,-1) (-4.5,-1)};
        \addplot[penColor,thick, ->] coordinates{(-4,-1) (-2.5,-1)};
        \addplot[penColor,thick, ->] coordinates{(-2,-1) (-.5,-1)};
        \addplot[penColor,thick, ->] coordinates{(0,-1) (1.5,-1)};
        \addplot[penColor,thick, ->] coordinates{(2,-1) (3.5,-1)};
        \addplot[penColor,thick, ->] coordinates{(4,-1) (5.5,-1)};
        
        \addplot[penColor,thick, ->] coordinates{(-6,-2) (-5.5,-2)};
        \addplot[penColor,thick, ->] coordinates{(-5,-2) (-4.5,-2)};
        \addplot[penColor,thick, ->] coordinates{(-4,-2) (-3.5,-2)};
        \addplot[penColor,thick, ->] coordinates{(-3,-2) (-2.5,-2)};
        \addplot[penColor,thick, ->] coordinates{(-2,-2) (-1.5,-2)};
        \addplot[penColor,thick, ->] coordinates{(-1,-2) (-.5,-2)};
        \addplot[penColor,thick, ->] coordinates{(0,-2) (.5,-2)};
        \addplot[penColor,thick, ->] coordinates{(1,-2) (1.5,-2)};
        \addplot[penColor,thick, ->] coordinates{(2,-2) (2.5,-2)};
        \addplot[penColor,thick, ->] coordinates{(3,-2) (3.5,-2)};
        \addplot[penColor,thick, ->] coordinates{(4,-2) (4.5,-2)};
        \addplot[penColor,thick, ->] coordinates{(5,-2) (5.5,-2)};
        
        
        \addplot[penColor2,ultra thick,samples=100] {-sqrt(16-x^2)+2.5};

        \addplot[penColor2,->,ultra thick,domain=-1:0,samples=10] {-sqrt(16-x^2)+2.5};
        
      \end{axis}
    \end{tikzpicture}
  \end{image}
  Do you expect 
  \[
  \int_C \vec{F}\dotp \d \vec{p} 
  \]
  to be positive, zero, or negative?
  \begin{onlineOnly}
  \begin{multipleChoice}
    \choice[correct]{positive}
    \choice{zero}
    \choice{negative}
  \end{multipleChoice}
  \begin{hint}
    Think about what the tangent vectors to the parameterized curve
    look like, and whether they point with the field or against the
    field.
  \end{hint}
  \end{onlineOnly}
\end{question}


\begin{question}
  Consider the following vector field along with a (directed) curve
  $C$.
  \begin{image}
    \begin{tikzpicture}
      \begin{axis}%
        [hide axis,
          width=3in,
          height=2in,
	  ymin=-3,ymax=2.5,
	  xmin=-6,xmax=5.5,
	]
        \addplot[penColor,thick, ->] coordinates{(-4,0) (-4,2.5)};
        \addplot[penColor,thick, ->] coordinates{(-3,0) (-3,2.5)};
        \addplot[penColor,thick, ->] coordinates{(-2,0) (-2,2.5)};
        \addplot[penColor,thick, ->] coordinates{(-1,0) (-1,2.5)};
        \addplot[penColor,thick, ->] coordinates{(0,0) (0,2.5)};
        \addplot[penColor,thick, ->] coordinates{(1,0) (1,2.5)};
        \addplot[penColor,thick, ->] coordinates{(2,0) (2,2.5)};
        \addplot[penColor,thick, ->] coordinates{(3,0) (3,2.5)};
        \addplot[penColor,thick, ->] coordinates{(4,0) (4,2.5)};

        \addplot[penColor,thick, ->] coordinates{(-4,-2) (-4,-.25)};
        \addplot[penColor,thick, ->] coordinates{(-3,-2) (-3,-.25)};
        \addplot[penColor,thick, ->] coordinates{(-2,-2) (-2,-.25)};
        \addplot[penColor,thick, ->] coordinates{(-1,-2) (-1,-.25)};
        \addplot[penColor,thick, ->] coordinates{(0,-2) (0,-.25)};
        \addplot[penColor,thick, ->] coordinates{(1,-2) (1,-.25)};
        \addplot[penColor,thick, ->] coordinates{(2,-2) (2,-.25)};
        \addplot[penColor,thick, ->] coordinates{(3,-2) (3,-.25)};
        \addplot[penColor,thick, ->] coordinates{(4,-2) (4,-.25)};

        \addplot[penColor,thick, ->] coordinates{(-4,-3) (-4,-2.2)};
        \addplot[penColor,thick, ->] coordinates{(-3,-3) (-3,-2.2)};
        \addplot[penColor,thick, ->] coordinates{(-2,-3) (-2,-2.2)};
        \addplot[penColor,thick, ->] coordinates{(-1,-3) (-1,-2.2)};
        \addplot[penColor,thick, ->] coordinates{(0,-3) (0,-2.2)};
        \addplot[penColor,thick, ->] coordinates{(1,-3) (1,-2.2)};
        \addplot[penColor,thick, ->] coordinates{(2,-3) (2,-2.2)};
        \addplot[penColor,thick, ->] coordinates{(3,-3) (3,-2.2)};
        \addplot[penColor,thick, ->] coordinates{(4,-3) (4,-2.2)};

        \addplot[penColor2,ultra thick,samples=100] {-1.5*sqrt(9-x^2)+2};

        \addplot[penColor2,->,ultra thick,domain=-1:0,samples=10] {-1.5*sqrt(9-x^2)+2};
      \end{axis}
    \end{tikzpicture}
  \end{image}
  Do you expect 
  \[
  \int_C \vec{F}\dotp \d \vec{p} 
  \]
  to be positive, zero, or negative?
  \begin{onlineOnly}
  \begin{multipleChoice}
    \choice{positive}
    \choice[correct]{zero}
    \choice{negative}
  \end{multipleChoice}
  \begin{hint}
    Think about what the tangent vectors to the parameterized curve
    look like, and whether they point with the field or against the
    field.
  \end{hint}
  \end{onlineOnly}
\end{question}


\section{Computations with line integrals}




\begin{example}
  Let $\vec{F}(x,y) = \vector{-y,x}$ and let $C$ be the unit circle
  centered at the origin. Compute
  \[
  \int_C \vec{F}\dotp d\vec{p}
  \]
  \begin{explanation}
    The path $C$ can be parameterized by
    \begin{align*}
      x(\theta) &= \cos(\theta)\\
      y(\theta) &= \sin(\theta)
    \end{align*}
    with $0\le \theta\le 2\pi$. To compute the integral, write with me
    \begin{align*}
      \int_C \vec{F}\dotp d\vec{p} &= \int_0^{2\pi} F(x(\theta),y(\theta))\dotp \vector{x'(\theta),y'(\theta)}\d \theta\\
      &= \int_0^{2\pi} \vector{-\sin(\theta),\cos(\theta)}\dotp \vector{-\sin(\theta),\cos(\theta)}\d \theta\\
      &= \int_0^{2\pi}\left(\sin^2(\theta)+\cos^2(\theta)\right)\d \theta\\
      &= \int_0^{2\pi} 1\d \theta\\
      &=2\pi.
    \end{align*}
    Later we'll understand how to interpert this result using
    \textit{Green's Theorem}.
  \end{explanation}
\end{example}

Any smooth path can be approximated with a polygonal path. These can
be quite easy to integrate. Check out our next example.

\begin{example}
  Let $\vec{F}(x,y) = \vector{0,x}$ and let $C$ be the polygonal path
  below parameteized in a counterclockwise direction:
  \begin{image}
    \begin{tikzpicture}
      \begin{axis}%
        [
	  ymin=-.5,ymax=2.5,
	  xmin=-.5,xmax=4.5,
          axis lines =middle, xlabel=$x$, ylabel=$y$,
          every axis y label/.style={at=(current axis.above origin),anchor=south},
          every axis x label/.style={at=(current axis.right of origin),anchor=west},
          grid=both,
          grid style={dashed, gridColor},
         % xtick={-2,...,4},
         % ytick={-3,...,3},
	]
        \addplot[penColor,ultra thick] coordinates{
            (0,0) (1,2) (3,2) (4,0) (0,0) (1,2)
          };

      \end{axis}
    \end{tikzpicture}    
  \end{image}
  Compute
  \[
  \int_C \vec{F}\dotp d\vec{p}
  \]
  \begin{explanation}
    We need to parameterize our paths in a counterclockwise
    direction. We'll break it into four line segments each parametrized
    as $t$ runs from $0$ to $1$.
    \begin{image}
      NEED AN IMAGE
    \end{image}
    \begin{align*}
      \vecl_1(t) &= \vector{4t,0}\\
      \vecl_2(t) &= \vector{4-t,2t}\\
      \vecl_3(t) &= \vector{3-2t,2}\\
      \vecl_4(t) &= \vector{1-t,2-2t}
    \end{align*}
  \end{explanation}
\end{example}



\section{The fundamental theorems of calculus}


We will soon see that there are many ``fundamental theorems of
calculus.'' What makes them similar is that they all share the
following rather vague description:
\begin{quote}
  To compute a certain sort of integral over a region, we may do a
  computation on the boundary of the region that involves one fewer
  integrations.
\end{quote}

EXPLAIN WHAT THIS MEANS



We now come to the first of three important theorems that extend the
Fundamental Theorem of Calculus to higher dimensions. The Fundamental
Theorem of Line Integrals has already done this in one way, but in
that case we were still dealing with an essentially one-dimensional
integral.

They all share with the Fundamental Theorem the following
rather vague description:

To compute a certain sort of integral over a
region, we may do a computation on the boundary of the region that
involves one fewer integrations.


\begin{theorem}[Fundamental Theorem for Line Integrals]
  If $C$ is a curve that starts at $\vec{a}$ and ends at $\vec{b}$
  \[
  \int_C \grad F\dotp \d \vec{p} = F(\vec{b}) - F(\vec{a})
  \]
\end{theorem}

\begin{question}
\end{question}

\begin{example}
\end{example}

\subsection{Consertative fields}





\end{document}
