\documentclass{ximera}

%\usepackage{todonotes}

\newcommand{\todo}{}


\graphicspath{
{./}
{../functionsOfSeveralVariables/}
{../normalVectors/}
{../lagrangeMultipliers/}
}


\usepackage{tkz-euclide}
\tikzset{>=stealth} %% cool arrow head
\tikzset{shorten <>/.style={ shorten >=#1, shorten <=#1 } } %% allows shorter vectors

\usetikzlibrary{backgrounds} %% for boxes around graphs
\usetikzlibrary{shapes,positioning}  %% Clouds and stars
\usetikzlibrary{matrix} %% for matrix
\usepgfplotslibrary{polar} %% for polar plots
\usetkzobj{all}
\usepackage[makeroom]{cancel} %% for strike outs
%\usepackage{mathtools} %% for pretty underbrace % Breaks Ximera
\usepackage{multicol}





\usepackage{array}
\setlength{\extrarowheight}{+.1cm}   
\newdimen\digitwidth
\settowidth\digitwidth{9}
\def\divrule#1#2{
\noalign{\moveright#1\digitwidth
\vbox{\hrule width#2\digitwidth}}}





\newcommand{\RR}{\mathbb R}
\newcommand{\R}{\mathbb R}
\newcommand{\N}{\mathbb N}
\newcommand{\Z}{\mathbb Z}

\newcommand{\sage}{\textsf{SageMath}}


%\renewcommand{\d}{\,d\!}
\renewcommand{\d}{\mathop{}\!d}
\newcommand{\dd}[2][]{\frac{\d #1}{\d #2}}
\newcommand{\pp}[2][]{\frac{\partial #1}{\partial #2}}
\renewcommand{\l}{\ell}
\newcommand{\ddx}{\frac{d}{\d x}}

\newcommand{\zeroOverZero}{\ensuremath{\boldsymbol{\tfrac{0}{0}}}}
\newcommand{\inftyOverInfty}{\ensuremath{\boldsymbol{\tfrac{\infty}{\infty}}}}
\newcommand{\zeroOverInfty}{\ensuremath{\boldsymbol{\tfrac{0}{\infty}}}}
\newcommand{\zeroTimesInfty}{\ensuremath{\small\boldsymbol{0\cdot \infty}}}
\newcommand{\inftyMinusInfty}{\ensuremath{\small\boldsymbol{\infty - \infty}}}
\newcommand{\oneToInfty}{\ensuremath{\boldsymbol{1^\infty}}}
\newcommand{\zeroToZero}{\ensuremath{\boldsymbol{0^0}}}
\newcommand{\inftyToZero}{\ensuremath{\boldsymbol{\infty^0}}}



\newcommand{\numOverZero}{\ensuremath{\boldsymbol{\tfrac{\#}{0}}}}
\newcommand{\dfn}{\textbf}
%\newcommand{\unit}{\,\mathrm}
\newcommand{\unit}{\mathop{}\!\mathrm}
\newcommand{\eval}[1]{\bigg[ #1 \bigg]}
\newcommand{\seq}[1]{\left( #1 \right)}
\renewcommand{\epsilon}{\varepsilon}
\renewcommand{\iff}{\Leftrightarrow}

\DeclareMathOperator{\arccot}{arccot}
\DeclareMathOperator{\arcsec}{arcsec}
\DeclareMathOperator{\arccsc}{arccsc}
\DeclareMathOperator{\si}{Si}
\DeclareMathOperator{\proj}{\vec{proj}}
\DeclareMathOperator{\scal}{scal}
\DeclareMathOperator{\sign}{sign}


%% \newcommand{\tightoverset}[2]{% for arrow vec
%%   \mathop{#2}\limits^{\vbox to -.5ex{\kern-0.75ex\hbox{$#1$}\vss}}}
\newcommand{\arrowvec}{\overrightarrow}
%\renewcommand{\vec}[1]{\arrowvec{\mathbf{#1}}}
\renewcommand{\vec}{\mathbf}
\newcommand{\veci}{{\boldsymbol{\hat{\imath}}}}
\newcommand{\vecj}{{\boldsymbol{\hat{\jmath}}}}
\newcommand{\veck}{{\boldsymbol{\hat{k}}}}
\newcommand{\vecl}{\boldsymbol{\l}}
\newcommand{\utan}{\mathbf{\hat{t}}}
\newcommand{\unormal}{\mathbf{\hat{n}}}
\newcommand{\ubinormal}{\mathbf{\hat{b}}}

\newcommand{\dotp}{\bullet}
\newcommand{\cross}{\boldsymbol\times}
\newcommand{\grad}{\boldsymbol\nabla}
\newcommand{\divergence}{\grad\dotp}
\newcommand{\curl}{\grad\cross}
%\DeclareMathOperator{\divergence}{divergence}
%\DeclareMathOperator{\curl}[1]{\grad\cross #1}
\newcommand{\lto}{\mathop{\longrightarrow\,}\limits}


\colorlet{textColor}{black} 
\colorlet{background}{white}
\colorlet{penColor}{blue!50!black} % Color of a curve in a plot
\colorlet{penColor2}{red!50!black}% Color of a curve in a plot
\colorlet{penColor3}{red!50!blue} % Color of a curve in a plot
\colorlet{penColor4}{green!50!black} % Color of a curve in a plot
\colorlet{penColor5}{orange!80!black} % Color of a curve in a plot
\colorlet{fill1}{penColor!20} % Color of fill in a plot
\colorlet{fill2}{penColor2!20} % Color of fill in a plot
\colorlet{fillp}{fill1} % Color of positive area
\colorlet{filln}{penColor2!20} % Color of negative area
\colorlet{fill3}{penColor3!20} % Fill
\colorlet{fill4}{penColor4!20} % Fill
\colorlet{fill5}{penColor5!20} % Fill
\colorlet{gridColor}{gray!50} % Color of grid in a plot

\newcommand{\surfaceColor}{violet}
\newcommand{\surfaceColorTwo}{redyellow}
\newcommand{\sliceColor}{greenyellow}




\pgfmathdeclarefunction{gauss}{2}{% gives gaussian
  \pgfmathparse{1/(#2*sqrt(2*pi))*exp(-((x-#1)^2)/(2*#2^2))}%
}


%%%%%%%%%%%%%
%% Vectors
%%%%%%%%%%%%%

%% Simple horiz vectors
\renewcommand{\vector}[1]{\left\langle #1\right\rangle}


%% %% Complex Horiz Vectors with angle brackets
%% \makeatletter
%% \renewcommand{\vector}[2][ , ]{\left\langle%
%%   \def\nextitem{\def\nextitem{#1}}%
%%   \@for \el:=#2\do{\nextitem\el}\right\rangle%
%% }
%% \makeatother

%% %% Vertical Vectors
%% \def\vector#1{\begin{bmatrix}\vecListA#1,,\end{bmatrix}}
%% \def\vecListA#1,{\if,#1,\else #1\cr \expandafter \vecListA \fi}

%%%%%%%%%%%%%
%% End of vectors
%%%%%%%%%%%%%

%\newcommand{\fullwidth}{}
%\newcommand{\normalwidth}{}



%% makes a snazzy t-chart for evaluating functions
%\newenvironment{tchart}{\rowcolors{2}{}{background!90!textColor}\array}{\endarray}

%%This is to help with formatting on future title pages.
\newenvironment{sectionOutcomes}{}{} 



%% Flowchart stuff
%\tikzstyle{startstop} = [rectangle, rounded corners, minimum width=3cm, minimum height=1cm,text centered, draw=black]
%\tikzstyle{question} = [rectangle, minimum width=3cm, minimum height=1cm, text centered, draw=black]
%\tikzstyle{decision} = [trapezium, trapezium left angle=70, trapezium right angle=110, minimum width=3cm, minimum height=1cm, text centered, draw=black]
%\tikzstyle{question} = [rectangle, rounded corners, minimum width=3cm, minimum height=1cm,text centered, draw=black]
%\tikzstyle{process} = [rectangle, minimum width=3cm, minimum height=1cm, text centered, draw=black]
%\tikzstyle{decision} = [trapezium, trapezium left angle=70, trapezium right angle=110, minimum width=3cm, minimum height=1cm, text centered, draw=black]


\title[Dig-In:]{Line integrals}

\begin{document}
\begin{abstract}
We accumulate vectors along a path.
\end{abstract}
\maketitle

In this section we introduce a new type of integrals, \textit{line integrals} also known as \textit{path integrals}.

\section{Line integrals}

A \textit{line integral} measures the flow of a vector field along a
path. The basic idea is that there is some vector field given by
$\vec{F}$:
\begin{image}
\begin{tikzpicture}
      \begin{axis}%
        [hide axis,
          ymin=-4,ymax=4,
          xmin=-6,xmax=5.5,
        ]
        \addplot[penColor,thick, ->] coordinates{(-6,2) (-.5,2)};
        \addplot[penColor,thick, ->] coordinates{(0,2) (5.5,2)};

        \addplot[penColor,thick, ->] coordinates{(-6,1) (-2.5,1)};
        \addplot[penColor,thick, ->] coordinates{(-2,1) (1.5,1)};
        \addplot[penColor,thick, ->] coordinates{(2,1) (5.5,1)};

        \addplot[penColor,thick, ->] coordinates{(-6,0) (-3.5,0)};
        \addplot[penColor,thick, ->] coordinates{(-3,0) (-.5,0)};
        \addplot[penColor,thick, ->] coordinates{(0,0) (2.5,0)};
        \addplot[penColor,thick, ->] coordinates{(3,0) (5.5,0)};

        \addplot[penColor,thick, ->] coordinates{(-6,-1) (-4.5,-1)};
        \addplot[penColor,thick, ->] coordinates{(-4,-1) (-2.5,-1)};
        \addplot[penColor,thick, ->] coordinates{(-2,-1) (-.5,-1)};
        \addplot[penColor,thick, ->] coordinates{(0,-1) (1.5,-1)};
        \addplot[penColor,thick, ->] coordinates{(2,-1) (3.5,-1)};
        \addplot[penColor,thick, ->] coordinates{(4,-1) (5.5,-1)};

        \addplot[penColor,thick, ->] coordinates{(-6,-2) (-5.5,-2)};
        \addplot[penColor,thick, ->] coordinates{(-5,-2) (-4.5,-2)};
        \addplot[penColor,thick, ->] coordinates{(-4,-2) (-3.5,-2)};
        \addplot[penColor,thick, ->] coordinates{(-3,-2) (-2.5,-2)};
        \addplot[penColor,thick, ->] coordinates{(-2,-2) (-1.5,-2)};
        \addplot[penColor,thick, ->] coordinates{(-1,-2) (-.5,-2)};
        \addplot[penColor,thick, ->] coordinates{(0,-2) (.5,-2)};
        \addplot[penColor,thick, ->] coordinates{(1,-2) (1.5,-2)};
        \addplot[penColor,thick, ->] coordinates{(2,-2) (2.5,-2)};
        \addplot[penColor,thick, ->] coordinates{(3,-2) (3.5,-2)};
        \addplot[penColor,thick, ->] coordinates{(4,-2) (4.5,-2)};
        \addplot[penColor,thick, ->] coordinates{(5,-2) (5.5,-2)};
      \end{axis}
 \end{tikzpicture}
\end{image}
Note, there is a vector at \textbf{every} point above; however, we are
just showing a few vectors, so that the magnitude of the vectors can
be seen. Now we add an oriented path $C$ that is parameterized by
$\vec{p}(t) = \vector{x(t),y(t)}$. This can be thought of as a path
that an object takes through the field:
\begin{image}
  \begin{tikzpicture}
    \begin{axis}%
      [hide axis,
	ymin=-4,ymax=4,
	xmin=-6,xmax=5.5,
	]
      \addplot[penColor,thick, ->] coordinates{(-6,2) (-.5,2)};
      \addplot[penColor,thick, ->] coordinates{(0,2) (5.5,2)};
      
      \addplot[penColor,thick, ->] coordinates{(-6,1) (-2.5,1)};
      \addplot[penColor,thick, ->] coordinates{(-2,1) (1.5,1)};
      \addplot[penColor,thick, ->] coordinates{(2,1) (5.5,1)};
      
      \addplot[penColor,thick, ->] coordinates{(-6,0) (-3.5,0)};
      \addplot[penColor,thick, ->] coordinates{(-3,0) (-.5,0)};
      \addplot[penColor,thick, ->] coordinates{(0,0) (2.5,0)};
      \addplot[penColor,thick, ->] coordinates{(3,0) (5.5,0)};
      
      \addplot[penColor,thick, ->] coordinates{(-6,-1) (-4.5,-1)};
      \addplot[penColor,thick, ->] coordinates{(-4,-1) (-2.5,-1)};
      \addplot[penColor,thick, ->] coordinates{(-2,-1) (-.5,-1)};
      \addplot[penColor,thick, ->] coordinates{(0,-1) (1.5,-1)};
      \addplot[penColor,thick, ->] coordinates{(2,-1) (3.5,-1)};
      \addplot[penColor,thick, ->] coordinates{(4,-1) (5.5,-1)};
      
      \addplot[penColor,thick, ->] coordinates{(-6,-2) (-5.5,-2)};
      \addplot[penColor,thick, ->] coordinates{(-5,-2) (-4.5,-2)};
      \addplot[penColor,thick, ->] coordinates{(-4,-2) (-3.5,-2)};
      \addplot[penColor,thick, ->] coordinates{(-3,-2) (-2.5,-2)};
      \addplot[penColor,thick, ->] coordinates{(-2,-2) (-1.5,-2)};
      \addplot[penColor,thick, ->] coordinates{(-1,-2) (-.5,-2)};
      \addplot[penColor,thick, ->] coordinates{(0,-2) (.5,-2)};
      \addplot[penColor,thick, ->] coordinates{(1,-2) (1.5,-2)};
      \addplot[penColor,thick, ->] coordinates{(2,-2) (2.5,-2)};
      \addplot[penColor,thick, ->] coordinates{(3,-2) (3.5,-2)};
      \addplot[penColor,thick, ->] coordinates{(4,-2) (4.5,-2)};
      \addplot[penColor,thick, ->] coordinates{(5,-2) (5.5,-2)};
        
      \addplot[penColor2,ultra thick] coordinates{
        (-3,2.3) (3,2.3)
        (3,-2.3) (-3,-2.3)
      };
      \addplot[penColor2,ultra thick, ->] coordinates{(-3,2.3) (0,2.3)};
      \addplot[penColor2,ultra thick, ->] coordinates{(3,2.3) (3,0)};
      \addplot[penColor2,ultra thick, ->] coordinates{(3,-2.3) (0,-2.3)};
    \end{axis}
  \end{tikzpicture}
\end{image}

To figure out if the flow of the vector field is ``with'' the
direction of the path, we use the dot product:
\[
\underbrace{\vec{F}(x(t),y(t))}_{\text{direction of field}} \dotp \underbrace{\vector{x'(t) \d t,y'(t) \d t}}_{\text{direction of path}}
\]
\begin{question}
When the direction of the field and the direction path are in
alignment, the dot product is\dots
\begin{onlineOnly}
  \begin{multipleChoice}
    \choice[correct]{positive}
    \choice{zero}
    \choice{negative}
  \end{multipleChoice}
\end{onlineOnly}
\begin{question}
  When the direction of the
  field and the direction of the path are orthogonal, the dot product is\dots
  \begin{onlineOnly}
  \begin{multipleChoice}
    \choice{positive}
    \choice[correct]{zero}
    \choice{negative}
  \end{multipleChoice}
\end{onlineOnly}
\begin{question}
  When the direction of the field and the direction of the path are in
  opposite direction, the dot product is\dots
  \begin{onlineOnly}
  \begin{multipleChoice}
    \choice{positive}
    \choice{zero}
    \choice[correct]{negative}
  \end{multipleChoice}
  \end{onlineOnly}
\end{question}
\end{question}
\end{question}
Integrating over the path allows us to sum these infinitesimal
measurements together to help measure if the flow of the field is
with the path, against the path, or orthogonal to the path.


\begin{definition}
Let $\vec{F}:\R^2\to\R^2$ be a vector field, $\vec{p}:\R\to\R^2$ be a
vector valued function,
\begin{align*}
  \vec{F}(x,y) &= \vector{M(x,y), N(x,y)}\\
  \vec{p}(t) &= \vector{x(t),y(t)}.
\end{align*}
A \dfn{line integral} is an integral of the form:
\[
\int_C \vec{F}\dotp \d \vec{p} = \int_C \vector{M,N}\dotp\vector{\d x,\d y}
\]
Since $\d x = x'(t)\d t$ and $\d y = y'(t)\d t$, we may write  
\begin{align*}
  &= \int_C M\cdot \d x + N\cdot \d y\\
  &= \int_C M(x(t))\cdot x'(t) \d t + N(y(t))\cdot  y'(t) \d t\\
  &= \int_C M\cdot \d x  + N\cdot  \d y
\end{align*}
\end{definition}

\begin{question}
  Consider the following vector field along with a (directed) curve
  $C$.
  \begin{image}
    \begin{tikzpicture}
      \begin{axis}%
        [hide axis,
	  ymin=-4,ymax=4,
	  xmin=-6,xmax=5.5,
	]
        \addplot[penColor,thick, ->] coordinates{(-6,2) (-.5,2)};
        \addplot[penColor,thick, ->] coordinates{(0,2) (5.5,2)};

        \addplot[penColor,thick, ->] coordinates{(-6,1) (-2.5,1)};
        \addplot[penColor,thick, ->] coordinates{(-2,1) (1.5,1)};
        \addplot[penColor,thick, ->] coordinates{(2,1) (5.5,1)};

        \addplot[penColor,thick, ->] coordinates{(-6,0) (-3.5,0)};
        \addplot[penColor,thick, ->] coordinates{(-3,0) (-.5,0)};
        \addplot[penColor,thick, ->] coordinates{(0,0) (2.5,0)};
        \addplot[penColor,thick, ->] coordinates{(3,0) (5.5,0)};

        \addplot[penColor,thick, ->] coordinates{(-6,-1) (-4.5,-1)};
        \addplot[penColor,thick, ->] coordinates{(-4,-1) (-2.5,-1)};
        \addplot[penColor,thick, ->] coordinates{(-2,-1) (-.5,-1)};
        \addplot[penColor,thick, ->] coordinates{(0,-1) (1.5,-1)};
        \addplot[penColor,thick, ->] coordinates{(2,-1) (3.5,-1)};
        \addplot[penColor,thick, ->] coordinates{(4,-1) (5.5,-1)};
        
        \addplot[penColor,thick, ->] coordinates{(-6,-2) (-5.5,-2)};
        \addplot[penColor,thick, ->] coordinates{(-5,-2) (-4.5,-2)};
        \addplot[penColor,thick, ->] coordinates{(-4,-2) (-3.5,-2)};
        \addplot[penColor,thick, ->] coordinates{(-3,-2) (-2.5,-2)};
        \addplot[penColor,thick, ->] coordinates{(-2,-2) (-1.5,-2)};
        \addplot[penColor,thick, ->] coordinates{(-1,-2) (-.5,-2)};
        \addplot[penColor,thick, ->] coordinates{(0,-2) (.5,-2)};
        \addplot[penColor,thick, ->] coordinates{(1,-2) (1.5,-2)};
        \addplot[penColor,thick, ->] coordinates{(2,-2) (2.5,-2)};
        \addplot[penColor,thick, ->] coordinates{(3,-2) (3.5,-2)};
        \addplot[penColor,thick, ->] coordinates{(4,-2) (4.5,-2)};
        \addplot[penColor,thick, ->] coordinates{(5,-2) (5.5,-2)};
        
        \addplot[penColor2,ultra thick] coordinates{
          (-3,2.3) (3,2.3)
          (3,-2.3) (-3,-2.3)
        };
        \addplot[penColor2,ultra thick, ->] coordinates{(-3,2.3) (0,2.3)};
        \addplot[penColor2,ultra thick, ->] coordinates{(3,2.3) (3,0)};
        \addplot[penColor2,ultra thick, ->] coordinates{(3,-2.3) (0,-2.3)};
      \end{axis}
    \end{tikzpicture}
  \end{image}
  Do you expect 
  \[
  \int_C \vec{F}\dotp \d \vec{p} 
  \]
  to be positive, zero, or negative?
  \begin{onlineOnly}
  \begin{multipleChoice}
    \choice[correct]{positive}
    \choice{zero}
    \choice{negative}
  \end{multipleChoice}
  \end{onlineOnly}
\end{question}






\begin{question}
  Consider the following vector field along with a (directed) curve
  $C$.
  \begin{image}
    \begin{tikzpicture}
      \begin{axis}%
        [hide axis,
	  ymin=-4,ymax=4,
	  xmin=-6,xmax=5.5,
	]
        \addplot[penColor,thick, ->] coordinates{(-4,0) (-4,2.5)};
        \addplot[penColor,thick, ->] coordinates{(-3,0) (-3,2.5)};
        \addplot[penColor,thick, ->] coordinates{(-2,0) (-2,2.5)};
        \addplot[penColor,thick, ->] coordinates{(-1,0) (-1,2.5)};
        \addplot[penColor,thick, ->] coordinates{(0,0) (0,2.5)};
        \addplot[penColor,thick, ->] coordinates{(1,0) (1,2.5)};
        \addplot[penColor,thick, ->] coordinates{(2,0) (2,2.5)};
        \addplot[penColor,thick, ->] coordinates{(3,0) (3,2.5)};
        \addplot[penColor,thick, ->] coordinates{(4,0) (4,2.5)};

        \addplot[penColor,thick, ->] coordinates{(-4,-2) (-4,-.5)};
        \addplot[penColor,thick, ->] coordinates{(-3,-2) (-3,-.5)};
        \addplot[penColor,thick, ->] coordinates{(-2,-2) (-2,-.5)};
        \addplot[penColor,thick, ->] coordinates{(-1,-2) (-1,-.5)};
        \addplot[penColor,thick, ->] coordinates{(0,-2) (0,-.5)};
        \addplot[penColor,thick, ->] coordinates{(1,-2) (1,-.5)};
        \addplot[penColor,thick, ->] coordinates{(2,-2) (2,-.5)};
        \addplot[penColor,thick, ->] coordinates{(3,-2) (3,-.5)};
        \addplot[penColor,thick, ->] coordinates{(4,-2) (4,-.5)};

        \addplot[penColor,thick, ->] coordinates{(-4,-3) (-4,-2.2)};
        \addplot[penColor,thick, ->] coordinates{(-3,-3) (-3,-2.2)};
        \addplot[penColor,thick, ->] coordinates{(-2,-3) (-2,-2.2)};
        \addplot[penColor,thick, ->] coordinates{(-1,-3) (-1,-2.2)};
        \addplot[penColor,thick, ->] coordinates{(0,-3) (0,-2.2)};
        \addplot[penColor,thick, ->] coordinates{(1,-3) (1,-2.2)};
        \addplot[penColor,thick, ->] coordinates{(2,-3) (2,-2.2)};
        \addplot[penColor,thick, ->] coordinates{(3,-3) (3,-2.2)};
        \addplot[penColor,thick, ->] coordinates{(4,-3) (4,-2.2)};

        
        
        
        \addplot[penColor2,ultra thick] coordinates{
          (-3,1.7) (3,1.7)
          (3,-2.3) (-3,-2.3)
        };
        \addplot[penColor2,ultra thick, ->] coordinates{(-3,1.7) (0,1.7)};
        \addplot[penColor2,ultra thick, ->] coordinates{(3,1.7) (3,0)};
        \addplot[penColor2,ultra thick, ->] coordinates{(3,-2.3) (0,-2.3)};
      \end{axis}
    \end{tikzpicture}
  \end{image}
  Do you expect 
  \[
  \int_C \vec{F}\dotp \d \vec{p} 
  \]
  to be positive, zero, or negative?
  \begin{onlineOnly}
  \begin{multipleChoice}
    \choice[correct]{positive}
    \choice{zero}
    \choice{negative}
  \end{multipleChoice}
  \end{onlineOnly}
\end{question}











\begin{example}
  Let $\vec{F}(x,y) = \vector{-y,x}$ and let $C$ be the unit circle
  centered at the origin. Compute
  \[
  \int_C \vec{F}\dotp d\vec{p}
  \]
  \begin{explanation}
    The path $C$ can be parameterized by
    \begin{align*}
      x(\theta) &= \cos(\theta)\\
      y(\theta) &= \sin(\theta)
    \end{align*}
    with $0\le \theta\le 2\pi$. To compute the integral, write with me
    \begin{align*}
      \int_C \vec{F}\dotp d\vec{p} &= \int_0^{2\pi} F(x(\theta),y(\theta))\dotp \vector{x'(\theta),y'(\theta)}\d \theta\\
      &= \int_0^{2\pi} \vector{-\sin(\theta),\cos(\theta)}\dotp \vector{-\sin(\theta),\cos(\theta)}\d \theta\\
      &= \int_0^{2\pi}\left(\sin^2(\theta)+\cos^2(\theta)\right)\d \theta\\
      &= \int_0^{2\pi} 1\d \theta\\
      &=2\pi.
    \end{align*}
  \end{explanation}
\end{example}



\section{The fundamental theorems of calculus}



We now come to the first of three important theorems that extend the
Fundamental Theorem of Calculus to higher dimensions. (The Fundamental
Theorem of Line Integrals has already done this in one way, but in
that case we were still dealing with an essentially one-dimensional
integral.)

They all share with the Fundamental Theorem the following
rather vague description:

To compute a certain sort of integral over a
region, we may do a computation on the boundary of the region that
involves one fewer integrations.


\begin{theorem}[Fundamental Theorem for Line Integrals]
  If $C$ is a curve that starts at $\vec{a}$ and ends at $\vec{b}$
  \[
  \int_C \grad F\dotp \d \vec{p} = F(\vec{b}) - F(\vec{a})
  \]
\end{theorem}

\begin{question}
\end{question}

\begin{example}
\end{example}



\end{document}
