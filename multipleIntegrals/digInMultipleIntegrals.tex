\documentclass{ximera}

%\usepackage{todonotes}

\newcommand{\todo}{}


\graphicspath{
{./}
{../functionsOfSeveralVariables/}
{../normalVectors/}
{../lagrangeMultipliers/}
}


\usepackage{tkz-euclide}
\tikzset{>=stealth} %% cool arrow head
\tikzset{shorten <>/.style={ shorten >=#1, shorten <=#1 } } %% allows shorter vectors

\usetikzlibrary{backgrounds} %% for boxes around graphs
\usetikzlibrary{shapes,positioning}  %% Clouds and stars
\usetikzlibrary{matrix} %% for matrix
\usepgfplotslibrary{polar} %% for polar plots
\usetkzobj{all}
\usepackage[makeroom]{cancel} %% for strike outs
%\usepackage{mathtools} %% for pretty underbrace % Breaks Ximera
\usepackage{multicol}





\usepackage{array}
\setlength{\extrarowheight}{+.1cm}   
\newdimen\digitwidth
\settowidth\digitwidth{9}
\def\divrule#1#2{
\noalign{\moveright#1\digitwidth
\vbox{\hrule width#2\digitwidth}}}





\newcommand{\RR}{\mathbb R}
\newcommand{\R}{\mathbb R}
\newcommand{\N}{\mathbb N}
\newcommand{\Z}{\mathbb Z}

\newcommand{\sage}{\textsf{SageMath}}


%\renewcommand{\d}{\,d\!}
\renewcommand{\d}{\mathop{}\!d}
\newcommand{\dd}[2][]{\frac{\d #1}{\d #2}}
\newcommand{\pp}[2][]{\frac{\partial #1}{\partial #2}}
\renewcommand{\l}{\ell}
\newcommand{\ddx}{\frac{d}{\d x}}

\newcommand{\zeroOverZero}{\ensuremath{\boldsymbol{\tfrac{0}{0}}}}
\newcommand{\inftyOverInfty}{\ensuremath{\boldsymbol{\tfrac{\infty}{\infty}}}}
\newcommand{\zeroOverInfty}{\ensuremath{\boldsymbol{\tfrac{0}{\infty}}}}
\newcommand{\zeroTimesInfty}{\ensuremath{\small\boldsymbol{0\cdot \infty}}}
\newcommand{\inftyMinusInfty}{\ensuremath{\small\boldsymbol{\infty - \infty}}}
\newcommand{\oneToInfty}{\ensuremath{\boldsymbol{1^\infty}}}
\newcommand{\zeroToZero}{\ensuremath{\boldsymbol{0^0}}}
\newcommand{\inftyToZero}{\ensuremath{\boldsymbol{\infty^0}}}



\newcommand{\numOverZero}{\ensuremath{\boldsymbol{\tfrac{\#}{0}}}}
\newcommand{\dfn}{\textbf}
%\newcommand{\unit}{\,\mathrm}
\newcommand{\unit}{\mathop{}\!\mathrm}
\newcommand{\eval}[1]{\bigg[ #1 \bigg]}
\newcommand{\seq}[1]{\left( #1 \right)}
\renewcommand{\epsilon}{\varepsilon}
\renewcommand{\iff}{\Leftrightarrow}

\DeclareMathOperator{\arccot}{arccot}
\DeclareMathOperator{\arcsec}{arcsec}
\DeclareMathOperator{\arccsc}{arccsc}
\DeclareMathOperator{\si}{Si}
\DeclareMathOperator{\proj}{\vec{proj}}
\DeclareMathOperator{\scal}{scal}
\DeclareMathOperator{\sign}{sign}


%% \newcommand{\tightoverset}[2]{% for arrow vec
%%   \mathop{#2}\limits^{\vbox to -.5ex{\kern-0.75ex\hbox{$#1$}\vss}}}
\newcommand{\arrowvec}{\overrightarrow}
%\renewcommand{\vec}[1]{\arrowvec{\mathbf{#1}}}
\renewcommand{\vec}{\mathbf}
\newcommand{\veci}{{\boldsymbol{\hat{\imath}}}}
\newcommand{\vecj}{{\boldsymbol{\hat{\jmath}}}}
\newcommand{\veck}{{\boldsymbol{\hat{k}}}}
\newcommand{\vecl}{\boldsymbol{\l}}
\newcommand{\utan}{\mathbf{\hat{t}}}
\newcommand{\unormal}{\mathbf{\hat{n}}}
\newcommand{\ubinormal}{\mathbf{\hat{b}}}

\newcommand{\dotp}{\bullet}
\newcommand{\cross}{\boldsymbol\times}
\newcommand{\grad}{\boldsymbol\nabla}
\newcommand{\divergence}{\grad\dotp}
\newcommand{\curl}{\grad\cross}
%\DeclareMathOperator{\divergence}{divergence}
%\DeclareMathOperator{\curl}[1]{\grad\cross #1}
\newcommand{\lto}{\mathop{\longrightarrow\,}\limits}


\colorlet{textColor}{black} 
\colorlet{background}{white}
\colorlet{penColor}{blue!50!black} % Color of a curve in a plot
\colorlet{penColor2}{red!50!black}% Color of a curve in a plot
\colorlet{penColor3}{red!50!blue} % Color of a curve in a plot
\colorlet{penColor4}{green!50!black} % Color of a curve in a plot
\colorlet{penColor5}{orange!80!black} % Color of a curve in a plot
\colorlet{fill1}{penColor!20} % Color of fill in a plot
\colorlet{fill2}{penColor2!20} % Color of fill in a plot
\colorlet{fillp}{fill1} % Color of positive area
\colorlet{filln}{penColor2!20} % Color of negative area
\colorlet{fill3}{penColor3!20} % Fill
\colorlet{fill4}{penColor4!20} % Fill
\colorlet{fill5}{penColor5!20} % Fill
\colorlet{gridColor}{gray!50} % Color of grid in a plot

\newcommand{\surfaceColor}{violet}
\newcommand{\surfaceColorTwo}{redyellow}
\newcommand{\sliceColor}{greenyellow}




\pgfmathdeclarefunction{gauss}{2}{% gives gaussian
  \pgfmathparse{1/(#2*sqrt(2*pi))*exp(-((x-#1)^2)/(2*#2^2))}%
}


%%%%%%%%%%%%%
%% Vectors
%%%%%%%%%%%%%

%% Simple horiz vectors
\renewcommand{\vector}[1]{\left\langle #1\right\rangle}


%% %% Complex Horiz Vectors with angle brackets
%% \makeatletter
%% \renewcommand{\vector}[2][ , ]{\left\langle%
%%   \def\nextitem{\def\nextitem{#1}}%
%%   \@for \el:=#2\do{\nextitem\el}\right\rangle%
%% }
%% \makeatother

%% %% Vertical Vectors
%% \def\vector#1{\begin{bmatrix}\vecListA#1,,\end{bmatrix}}
%% \def\vecListA#1,{\if,#1,\else #1\cr \expandafter \vecListA \fi}

%%%%%%%%%%%%%
%% End of vectors
%%%%%%%%%%%%%

%\newcommand{\fullwidth}{}
%\newcommand{\normalwidth}{}



%% makes a snazzy t-chart for evaluating functions
%\newenvironment{tchart}{\rowcolors{2}{}{background!90!textColor}\array}{\endarray}

%%This is to help with formatting on future title pages.
\newenvironment{sectionOutcomes}{}{} 



%% Flowchart stuff
%\tikzstyle{startstop} = [rectangle, rounded corners, minimum width=3cm, minimum height=1cm,text centered, draw=black]
%\tikzstyle{question} = [rectangle, minimum width=3cm, minimum height=1cm, text centered, draw=black]
%\tikzstyle{decision} = [trapezium, trapezium left angle=70, trapezium right angle=110, minimum width=3cm, minimum height=1cm, text centered, draw=black]
%\tikzstyle{question} = [rectangle, rounded corners, minimum width=3cm, minimum height=1cm,text centered, draw=black]
%\tikzstyle{process} = [rectangle, minimum width=3cm, minimum height=1cm, text centered, draw=black]
%\tikzstyle{decision} = [trapezium, trapezium left angle=70, trapezium right angle=110, minimum width=3cm, minimum height=1cm, text centered, draw=black]


\title[Dig-In:]{Introduction to multiple integrals}

\begin{document}
\begin{abstract}
NO TITLE YET.
\end{abstract}
\maketitle


Now we move on to finding \textit{volumes} under surfaces. We start by
only considering simple regions.

\section{Integrals over trivial regions}

\begin{image}
  \begin{tikzpicture}
    \begin{axis}%
      [width=175pt,height=200pt,
        tick label style={font=\scriptsize},axis on top,
	axis lines=center,
	view={110}{25},
	name=myplot,
	%xtick=\empty,
	%ytick={5},
	%ztick={.7,-.7},
	minor xtick=1,
	minor ytick=1,
	ymin=-1.2,ymax=1.2,
	xmin=-.5,xmax=2.5,
	zmin=-.1, zmax=2.1,
	every axis x label/.style={at={(axis cs:\pgfkeysvalueof{/pgfplots/xmax},0,0)},xshift=-1pt,yshift=-4pt},
	xlabel={\scriptsize $x$},
	every axis y label/.style={at={(axis cs:0,\pgfkeysvalueof{/pgfplots/ymax},0)},xshift=5pt,yshift=-3pt},
	ylabel={\scriptsize $y$},
	every axis z label/.style={at={(axis cs:0,0,\pgfkeysvalueof{/pgfplots/zmax})},xshift=0pt,yshift=4pt},
	zlabel={\scriptsize $z$},
        colormap/cool,
      ]
      %% 12 lines
      \pgfplotsinvokeforeach{0,0.1667,...,2}{
        \draw[gray] (axis cs: #1,-1,0) -- (axis cs: #1 , 1,0);
      }

      %% 16 lines
      \pgfplotsinvokeforeach{-1,-.875,...,1}{
        \draw[gray] (axis cs: 0,#1,0) -- (axis cs: 2 ,#1,0);
      }
           

      %% Box vol
      \draw [gray,thick]
      (axis cs: 1.833,.25,0) --
      (axis cs: 1.833,.375,0) --
      (axis cs: 1.833,.375,1.665) --
      (axis cs: 1.833,.25,1.665) --
      (axis cs: 1.833,.25,0);

      \draw [gray,thick]
      (axis cs: 1.833,.375,0) --
      (axis cs: 1.833,.375,1.665) --
      (axis cs: 1.667,.375,1.665) --
      (axis cs: 1.667,.375,0) --
      (axis cs: 1.833,.375,0);

      \draw [gray,thick]
      (axis cs: 1.667,.375,1.665) --
      (axis cs: 1.667,.25,1.665) --
      (axis cs: 1.833,.25,1.665);

      %% Surface %% 12 x 16
      \addplot3[domain=0:2,y domain=-1:1,mesh,samples=13,samples y=17,very thin,z buffer=sort] {-.5*(x-1)^2-.5*(y)^2+2};

      %% Surface curves
      \addplot3[domain=0:2,%fill=white,
        penColor,very thick,samples=13,samples y=0] (
               {x},
               {1},
               {-.5*(x-1)^2-.5*(1)^2+2});

      \addplot3[domain=0:.5,%fill=white,
        penColor,very thick,dashed,samples=13,samples y=0] (
               {x},
               {-1},
               {-.5*(x-1)^2-.5*(-1)^2+2});

      \addplot3[domain=.5:2,%fill=white,
        penColor,very thick,samples=13,samples y=0] (
               {x},
               {-1},
               {-.5*(x-1)^2-.5*(-1)^2+2});

      \addplot3[domain=-1:1,%fill=white,
        penColor,very thick,samples=13,samples y=18] (
               {2},
               {y},
               {-.5*(2-1)^2-.5*(y)^2+2});

      \addplot3[domain=-1:.8,%fill=white,
        dashed,penColor,very thick,samples=13,samples y=18] (
               {0},
               {y},
               {-.5*(0-1)^2-.5*(y)^2+2});

      \addplot3[domain=.75:1,%fill=white,
        penColor,very thick,samples=13,samples y=18] (
               {0},
               {y},
               {-.5*(0-1)^2-.5*(y)^2+2});

      %% dxdydz         
      \draw [penColor, thick]
      (axis cs: 1.667,.25,1.746) --(axis cs: 1.667,.375,1.71) --
      (axis cs: 1.833,.375,1.583) --(axis cs: 1.833,.25,1.621) --
      (axis cs: 1.667,.25,1.746) -- (axis cs: 1.667,.375,1.71);
      
      
    \end{axis}
  \end{tikzpicture}
\end{image}

In the diagram above, the volume enclosed by the surface is
\[

\]


\[
\int_R F(x,y) \d A = \lim_{n\to\infty}\sum_{i=1}^n\sum_{j=1}^n F(x_i^*,y_j^*)\Delta A
\]





\section{Integrals with trivial integrands}




\end{document}
