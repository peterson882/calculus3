\documentclass{ximera}

%\usepackage{todonotes}

\newcommand{\todo}{}


\graphicspath{
{./}
{../functionsOfSeveralVariables/}
{../normalVectors/}
{../lagrangeMultipliers/}
}


\usepackage{tkz-euclide}
\tikzset{>=stealth} %% cool arrow head
\tikzset{shorten <>/.style={ shorten >=#1, shorten <=#1 } } %% allows shorter vectors

\usetikzlibrary{backgrounds} %% for boxes around graphs
\usetikzlibrary{shapes,positioning}  %% Clouds and stars
\usetikzlibrary{matrix} %% for matrix
\usepgfplotslibrary{polar} %% for polar plots
\usetkzobj{all}
\usepackage[makeroom]{cancel} %% for strike outs
%\usepackage{mathtools} %% for pretty underbrace % Breaks Ximera
\usepackage{multicol}





\usepackage{array}
\setlength{\extrarowheight}{+.1cm}   
\newdimen\digitwidth
\settowidth\digitwidth{9}
\def\divrule#1#2{
\noalign{\moveright#1\digitwidth
\vbox{\hrule width#2\digitwidth}}}





\newcommand{\RR}{\mathbb R}
\newcommand{\R}{\mathbb R}
\newcommand{\N}{\mathbb N}
\newcommand{\Z}{\mathbb Z}

\newcommand{\sage}{\textsf{SageMath}}


%\renewcommand{\d}{\,d\!}
\renewcommand{\d}{\mathop{}\!d}
\newcommand{\dd}[2][]{\frac{\d #1}{\d #2}}
\newcommand{\pp}[2][]{\frac{\partial #1}{\partial #2}}
\renewcommand{\l}{\ell}
\newcommand{\ddx}{\frac{d}{\d x}}

\newcommand{\zeroOverZero}{\ensuremath{\boldsymbol{\tfrac{0}{0}}}}
\newcommand{\inftyOverInfty}{\ensuremath{\boldsymbol{\tfrac{\infty}{\infty}}}}
\newcommand{\zeroOverInfty}{\ensuremath{\boldsymbol{\tfrac{0}{\infty}}}}
\newcommand{\zeroTimesInfty}{\ensuremath{\small\boldsymbol{0\cdot \infty}}}
\newcommand{\inftyMinusInfty}{\ensuremath{\small\boldsymbol{\infty - \infty}}}
\newcommand{\oneToInfty}{\ensuremath{\boldsymbol{1^\infty}}}
\newcommand{\zeroToZero}{\ensuremath{\boldsymbol{0^0}}}
\newcommand{\inftyToZero}{\ensuremath{\boldsymbol{\infty^0}}}



\newcommand{\numOverZero}{\ensuremath{\boldsymbol{\tfrac{\#}{0}}}}
\newcommand{\dfn}{\textbf}
%\newcommand{\unit}{\,\mathrm}
\newcommand{\unit}{\mathop{}\!\mathrm}
\newcommand{\eval}[1]{\bigg[ #1 \bigg]}
\newcommand{\seq}[1]{\left( #1 \right)}
\renewcommand{\epsilon}{\varepsilon}
\renewcommand{\iff}{\Leftrightarrow}

\DeclareMathOperator{\arccot}{arccot}
\DeclareMathOperator{\arcsec}{arcsec}
\DeclareMathOperator{\arccsc}{arccsc}
\DeclareMathOperator{\si}{Si}
\DeclareMathOperator{\proj}{\vec{proj}}
\DeclareMathOperator{\scal}{scal}
\DeclareMathOperator{\sign}{sign}


%% \newcommand{\tightoverset}[2]{% for arrow vec
%%   \mathop{#2}\limits^{\vbox to -.5ex{\kern-0.75ex\hbox{$#1$}\vss}}}
\newcommand{\arrowvec}{\overrightarrow}
%\renewcommand{\vec}[1]{\arrowvec{\mathbf{#1}}}
\renewcommand{\vec}{\mathbf}
\newcommand{\veci}{{\boldsymbol{\hat{\imath}}}}
\newcommand{\vecj}{{\boldsymbol{\hat{\jmath}}}}
\newcommand{\veck}{{\boldsymbol{\hat{k}}}}
\newcommand{\vecl}{\boldsymbol{\l}}
\newcommand{\utan}{\mathbf{\hat{t}}}
\newcommand{\unormal}{\mathbf{\hat{n}}}
\newcommand{\ubinormal}{\mathbf{\hat{b}}}

\newcommand{\dotp}{\bullet}
\newcommand{\cross}{\boldsymbol\times}
\newcommand{\grad}{\boldsymbol\nabla}
\newcommand{\divergence}{\grad\dotp}
\newcommand{\curl}{\grad\cross}
%\DeclareMathOperator{\divergence}{divergence}
%\DeclareMathOperator{\curl}[1]{\grad\cross #1}
\newcommand{\lto}{\mathop{\longrightarrow\,}\limits}


\colorlet{textColor}{black} 
\colorlet{background}{white}
\colorlet{penColor}{blue!50!black} % Color of a curve in a plot
\colorlet{penColor2}{red!50!black}% Color of a curve in a plot
\colorlet{penColor3}{red!50!blue} % Color of a curve in a plot
\colorlet{penColor4}{green!50!black} % Color of a curve in a plot
\colorlet{penColor5}{orange!80!black} % Color of a curve in a plot
\colorlet{fill1}{penColor!20} % Color of fill in a plot
\colorlet{fill2}{penColor2!20} % Color of fill in a plot
\colorlet{fillp}{fill1} % Color of positive area
\colorlet{filln}{penColor2!20} % Color of negative area
\colorlet{fill3}{penColor3!20} % Fill
\colorlet{fill4}{penColor4!20} % Fill
\colorlet{fill5}{penColor5!20} % Fill
\colorlet{gridColor}{gray!50} % Color of grid in a plot

\newcommand{\surfaceColor}{violet}
\newcommand{\surfaceColorTwo}{redyellow}
\newcommand{\sliceColor}{greenyellow}




\pgfmathdeclarefunction{gauss}{2}{% gives gaussian
  \pgfmathparse{1/(#2*sqrt(2*pi))*exp(-((x-#1)^2)/(2*#2^2))}%
}


%%%%%%%%%%%%%
%% Vectors
%%%%%%%%%%%%%

%% Simple horiz vectors
\renewcommand{\vector}[1]{\left\langle #1\right\rangle}


%% %% Complex Horiz Vectors with angle brackets
%% \makeatletter
%% \renewcommand{\vector}[2][ , ]{\left\langle%
%%   \def\nextitem{\def\nextitem{#1}}%
%%   \@for \el:=#2\do{\nextitem\el}\right\rangle%
%% }
%% \makeatother

%% %% Vertical Vectors
%% \def\vector#1{\begin{bmatrix}\vecListA#1,,\end{bmatrix}}
%% \def\vecListA#1,{\if,#1,\else #1\cr \expandafter \vecListA \fi}

%%%%%%%%%%%%%
%% End of vectors
%%%%%%%%%%%%%

%\newcommand{\fullwidth}{}
%\newcommand{\normalwidth}{}



%% makes a snazzy t-chart for evaluating functions
%\newenvironment{tchart}{\rowcolors{2}{}{background!90!textColor}\array}{\endarray}

%%This is to help with formatting on future title pages.
\newenvironment{sectionOutcomes}{}{} 



%% Flowchart stuff
%\tikzstyle{startstop} = [rectangle, rounded corners, minimum width=3cm, minimum height=1cm,text centered, draw=black]
%\tikzstyle{question} = [rectangle, minimum width=3cm, minimum height=1cm, text centered, draw=black]
%\tikzstyle{decision} = [trapezium, trapezium left angle=70, trapezium right angle=110, minimum width=3cm, minimum height=1cm, text centered, draw=black]
%\tikzstyle{question} = [rectangle, rounded corners, minimum width=3cm, minimum height=1cm,text centered, draw=black]
%\tikzstyle{process} = [rectangle, minimum width=3cm, minimum height=1cm, text centered, draw=black]
%\tikzstyle{decision} = [trapezium, trapezium left angle=70, trapezium right angle=110, minimum width=3cm, minimum height=1cm, text centered, draw=black]


\outcome{Understand the definiton of a multiple integral.}

\outcome{Use iterated integrals to compute multiple integrals.}

\outcome{Apply Fubini's Theorem.}

\title[Dig-In:]{Integrals with trivial integrands}

\begin{document}
\begin{abstract}
  We study integrals over general regions with and integrand equaling one.
\end{abstract}
\maketitle


Now we will allow the region we are integrating over to be complex
than a rectangle. In particular we will find areas of regions bounded
by curves in $\R^2$, and volumes of regions bounded by surfaces in
$\R^3$.

\section{Double integrals and area}

We start with a more powerful version of Fubini's
Theorem and while we will state it generally, in this section $F$ will
always be $1$.

\begin{theorem}[Fubini's Theorem]\index{Fubini's Theorem}
  Let $R$ be a closed, bounded region in the $(x,y)$-plane and let
  $z=F(x,y)$ be a continuous function on $R$.
  \begin{itemize}
  \item If
    \[
    R=\{(x,y):\text{$a\leq x\leq b$ and $g_1(x)\leq y\leq g_2(x)$}\}
    \]
    where $g_1$ and $g_2$ are continuous functions on $[a,b]$, then
    \[
    \int_R F(x,y) \d A = \int_a^b\int_{g_1(x)}^{g_2(x)} F(x,y)\d y\d x.
    \]
  \item If
    \[
    R=\{(x,y):\text{$c\leq y\leq d$ and $g_1(y)\leq x\leq g_2(y)$}\}
    \]
    where $g_1$ and $g_2$ are continuous functions on $[c,d]$, then
    \[
    \int_R f(x,y)\d A = \int_c^d\int_{g_1(y)}^{g_2(y)} F(x,y)\d x\d y.
    \]
\end{itemize}
\end{theorem}

When the integrand of a double integral is $1$, like this
\[
\int_R\d A
\]
this integral simply computes the \textbf{area} of the region. This is
because you are finding the ``signed volume'' between the curve and
the plane $z=1$. In our first example, we will use a ``hammer to swat
a fly,'' meaning, we are essentially using a much more difficult
method than necessary! We apologize in advance, and include the
example because it is illustrative of the method of using double
integrals to compute areas.

\begin{example}
  Set-up and evaluate a double integral that will compute the area of
  the region below:
  \begin{image}
    \begin{tikzpicture}
      \begin{axis}[
          tick label style={font=\scriptsize},
          axis y line=middle,axis x line=middle,
          axis on top,name=myplot,
	  xtick={1,2,3},
	  ytick={1,2,3},
	  ymin=-.5,ymax=2.5,%
	  xmin=-.5,xmax=3.5,
          rounded corners=.5pt
        ]
        \draw [ultra thick,penColor] (axis cs:1,1) -- (axis cs: 3,1)-- (axis cs: 1,2)  -- cycle;
        
        \draw (axis cs: 1.5,1.3) node {$R$};
      \end{axis}
      
      \node [right] at (myplot.right of origin) {\scriptsize $x$};
      \node [above] at (myplot.above origin) {\scriptsize $y$};
    \end{tikzpicture}
  \end{image}
  \begin{explanation}
    Our region above can be defined by:
    \[
    R=\{(x,y):\text{$1\leq x\leq 3$ and $1\leq y\leq -x/2+5/2$}\}
    \]
    The area of this region is given by, write with me, 
    \begin{align*}
      \int_{\answer[given]{1}}^{\answer[given]{3}}\int_{\answer[given]{1}}^{\answer[given]{-x/2+5/2}}\d y \d x &= \int_{\answer[given]{1}}^{\answer[given]{3}} \eval{\answer[given]{y}}_{\answer[given]{1}}^{\answer[given]{-x/2+5/2}} \d x\\
      &=  \int_{\answer[given]{1}}^{\answer[given]{3}} \left(\answer[given]{-x/2+3/2}\right) \d x\\
      &=  \eval{\answer[given]{-x^2/4+3x/2}}_{\answer[given]{1}}^{\answer[given]{3}}\\
      &= \answer[given]{1}. 
    \end{align*}
  \end{explanation}
\end{example}

\begin{question}
  In the previous example, we integrated with respect to $y$
  first. Set-up an integral that computes the area of $R$ that
  integrates with respect to $x$ first.
  \begin{prompt}
    \[
    \int_R \d A = \int_{\answer{1}}^{\answer{3}}\int_{\answer{1}}^{\answer{-2y+5}} \d x \d y
    \]
  \end{prompt}
\end{question}

It is simple to compute the area of the region found in the previous
example. However, calculus really helps out when dealing with regions
of the sort found in the next two examples.

\begin{example}
  Set-up and evaluate a double integral that will compute the area of
  the region below:
  \begin{image}
    \begin{tikzpicture}
      \begin{axis}[
          tick label style={font=\scriptsize},axis y line=middle,axis x line=middle,name=myplot,axis on top,%
	  xtick={2,4,...,12},
	  ytick={-2,-4,-6,6,4,2},
	  ymin=-6.9,ymax=6.9,%
	  xmin=-.5,xmax=12.9%
        ]
        \draw (axis cs: 6,2) node {$R$}
	(axis cs: 4,4.5) node [rotate=23] {\scriptsize $x=y^2/3$};
        
        \addplot [penColor,ultra thick, smooth,domain=-6:6,samples=20] ({x^2/3},{x}) -- cycle;
      \end{axis}
      \node [right] at (myplot.right of origin) {\scriptsize $x$};
      \node [above] at (myplot.above origin) {\scriptsize $y$};
    \end{tikzpicture}
  \end{image}
  \begin{explanation}
    Our region above can be defined by:
    \[
    R=\{(x,y):\text{$-6\leq y\leq 6$ and $y^2/3\leq x\leq 12$}\}
    \]
    The area of this region is given by, write with me, 
    \begin{align*}
      \int_{\answer[given]{-6}}^{\answer[given]{6}}\int_{\answer[given]{y^2/3}}^{\answer[given]{12}}\d x \d y &= \int_{\answer[given]{-6}}^{\answer[given]{6}} \eval{\answer[given]{x}}_{\answer[given]{y^2/3}}^{\answer[given]{12}} \d y\\
      &=  \int_{\answer[given]{-6}}^{\answer[given]{6}} \left(\answer[given]{12-y^2/3}\right) \d y\\
      &=  \eval{\answer[given]{12y-y^3/9}}_{\answer[given]{-6}}^{\answer[given]{6}}\\
      &= \answer[given]{96}. 
    \end{align*}
  \end{explanation}
\end{example}

\begin{question}
  In the previous example, we integrated with respect to $x$
  first. Set-up an integral that computes the area of $R$ that
  integrates with respect to $y$ first.
  \begin{prompt}
    \[
    \int_R \d A = \int_{\answer{0}}^{\answer{12}}\int_{\answer{-\sqrt{3x}}}^{\answer{\sqrt{3x}}} \d y \d x
    \]
  \end{prompt}
\end{question}

And now for one more example of using a double integral to compute the
area of a region.

\begin{example}
  Set-up and evaluate a double integral that will compute the area of
  the region below:
  \begin{image}
    \begin{tikzpicture}
      \begin{axis}[
          tick label style={font=\scriptsize},axis y line=middle,axis x line=middle,name=myplot,axis on top,%
	  ymin=-.5,ymax=1.1,%
	  xmin=-.5,xmax=1.1%
        ]
        \draw (axis cs: .5,.4) node {$R$}
	(axis cs: .5,.8) node [rotate=26] {\scriptsize $y=\sqrt{x}$}
	(axis cs: .75,.16) node [rotate=29] {\scriptsize $y=x^4$};
        
        \addplot [penColor,ultra thick, smooth,domain=0:1.05,samples=20] ({x},{x^4});
        \addplot [penColor,ultra thick, smooth,domain=0:1.05,samples=20] ({x^2},{x});
      \end{axis}
      
      \node [right] at (myplot.right of origin) {\scriptsize $x$};
      \node [above] at (myplot.above origin) {\scriptsize $y$};
    \end{tikzpicture}
  \end{image}
  \begin{explanation}
    Our region above can be defined by:
    \[
    R=\{(x,y):\text{$0\leq x\leq 1$ and $x^4\leq y\leq \sqrt{x}$}\}
    \]
    The area of this region is given by, write with me, 
    \begin{align*}
      \int_{\answer[given]{0}}^{\answer[given]{1}}\int_{\answer[given]{x^4}}^{\answer[given]{\sqrt{x}}}\d y \d x &= \int_{\answer[given]{0}}^{\answer[given]{1}} \eval{\answer[given]{y}}_{\answer[given]{x^4}}^{\answer[given]{\sqrt{x}}} \d x\\
      &=  \int_{\answer[given]{0}}^{\answer[given]{1}} \left(\answer[given]{\sqrt{x}-x^4}\right) \d x\\
      &=  \eval{\answer[given]{2x^{3/2}/3 - x^5/5}}_{\answer[given]{0}}^{\answer[given]{1}}\\
      &= \answer[given]{7/15}. 
    \end{align*}
  \end{explanation}
\end{example}

\begin{question}
  In the previous example, we integrated with respect to $y$
  first. Set-up an integral that computes the area of $R$ that
  integrates with respect to $x$ first.
  \begin{prompt}
    \[
    \int_R \d A = \int_{\answer{0}}^{\answer{1}}\int_{\answer{x^2}}^{\answer{\sqrt[4]{x}}} \d x \d y
    \]
  \end{prompt}
\end{question}


\section{Triple integrals and volume}

We start by again(!) introducing another version of Fubini's Theorem.

\begin{theorem}[Fubini]\index{Fubini's Theorem}
  Let $R$ be a closed, bounded region in the $(x,y)$-plane and let
  $z=F(x,y)$ be a continuous function on $R$.
  \begin{itemize}
  \item If
    \[
    R=\{(x,y,z):\text{$a\leq x\leq b$, $g_1(x)\leq y\leq g_2(x)$, and $H_1(x,y) \leq z \leq H_2(x,y)$}\}
    \]
    where $g_1$ and $g_2$ are continuous functions on $[a,b]$, and $H_1$ and $H_2$ are continuous on the region
    \[
    S =\{(x,y):\text{$a\leq x\leq b$ and $g_1(x)\leq y\leq g_2(x)$}\}
    \]
    then
    \[
    \int_R F(x,y,z) \d V = \int_a^b\int_{g_1(x)}^{g_2(x)}\int_{H_1(x,y)}^{H_2(x,y)} F(x,y,z)\d z\d y\d x.
    \]
    There are six reorderings total with three variables. We will spare
    the young mathematician the details, and trust that you will sort
    it out.
\end{itemize}
\end{theorem}

Now with Fubini's help, we will use triple integrals to compute
volumes.

\begin{example}
  Set-up and evaluate a double integral that will compute the volume of
  the region bounded by:
  \begin{itemize}
  \item The plane $x=0$.
  \item The plane $y=0$.
  \item The plane $z=0$.
  \item The plane $3x + 4y + 6z = 12$.
  \end{itemize}
  \begin{explanation}
    Our region above produces a
    \textit{tetrahedron}\index{tetrahedron}, a triangular-based
    pyramid. It intersects the $x$-axis at $(4,0,0)$, the $y$-axis at
    $(0,3,0)$, and the $z$-axis at $(0,0,2)$.  The region can be
    defined by:
    \[
    R=\{(x,y,z):\text{$0\leq x\leq 4$, $0\leq y\leq -3x/4+4$, and $0\leq z \leq \frac{12-3x-4y}{6}$}\}
    \]
    The volume of this region is given by, write with me, 
    \begin{align*}
      \int_{\answer[given]{0}}^{\answer[given]{4}} \int_{\answer[given]{0}}^{\answer[given]{-3x/4+4}} \int_{\answer[given]{0}}^{\answer[given]{\frac{12-3x-4y}{6}}}\d z \d y\d x &=\int_{\answer[given]{0}}^{\answer[given]{4}} \int_{\answer[given]{0}}^{\answer[given]{-3x/4+4}} \answer[given]{\frac{12-3x-4y}{6}} \d y \d x\\
      &= \int_{\answer[given]{0}}^{\answer[given]{4}}  \eval{\answer[given]{\frac{12-3x-2y^2}{6}}}_{\answer[given]{0}}^{\answer[given]{-3x/4+4}} \d x\\
      &= \int_{\answer[given]{0}}^{\answer[given]{4}}  \left(\answer[given]{\frac{12-3x-2(-3x/4+4)^2}{6} - \frac{12-3x}{6}}\right) \d x\\
      &= \eval{\answer[given]{8x/3-3x^2/4+x^3/16}}_{\answer[given]{0}}^{\answer[given]{4}} \d x\\
      &=\answer[given]{8/3}.
    \end{align*}    
  \end{explanation}
\end{example}

\begin{question}
  In the previous example, we integrated with respect to $z$, then
  $y$, then $x$. Set-up an integral that computes the volume of $R$
  that integrates with respect to $x$, then $z$, then $y$.
  \begin{prompt}
    \[
    \int_R \d V = \int_{\answer{0}}^{\answer{3}}\int_{\answer{0}}^{\answer{-2y/3+2}} \int_{\answer{0}}^{\answer{\frac{12-4y-6z}{3}}} \d x \d z \d y
    \]
  \end{prompt}
\end{question}


\begin{example}
  Set-up a double integral that will compute the volume of the region
  bounded by the cone below:
  \begin{image}
    \begin{tikzpicture}
      \begin{axis}%
        [
          tick label style={font=\scriptsize},%axis on top,
	  axis lines=center,
	  view={135}{25},
	  name=myplot,
	  %% xtick=\empty,
	  %% ytick=\empty,
	  %% ztick=\empty,
	  %% extra x ticks={1},
	  %% extra x tick labels={$a$},
	  %% extra y ticks={1},
	  %% extra y tick labels={$a$},
	  %% extra z ticks={1},
	  %% extra z tick labels={$h$},
	  ymin=-1.3,ymax=1.3,
	  xmin=-1.3,xmax=1.3,
	  zmin=-.1, zmax=1.1,
	  every axis x label/.style={at={(axis cs:\pgfkeysvalueof{/pgfplots/xmax},0,0)},xshift=-1pt,yshift=-4pt},
	  xlabel={\scriptsize $x$},
	  every axis y label/.style={at={(axis cs:0,\pgfkeysvalueof{/pgfplots/ymax},0)},xshift=5pt,yshift=-3pt},
	  ylabel={\scriptsize $y$},
	  every axis z label/.style={at={(axis cs:0,0,\pgfkeysvalueof{/pgfplots/zmax})},xshift=0pt,yshift=4pt},
	  zlabel={\scriptsize $z$},colormap/cool
	]
        
        \addplot3[domain=0:1,,y domain=0:360,mesh,samples=10,samples y=36,very thin,z buffer=sort] ({x*cos(y)}, {x*sin(y)},{1-x});
      \end{axis}
    \end{tikzpicture}
  \end{image}
  \begin{explanation}
    First note that 
    \[
    R=\{(x,y,z):\text{$-1\leq x\leq 1$, $-\sqrt{1-x^2}\leq y\leq \sqrt{1-x^2}$, and $\sqrt{x^2+y^2}\leq z \leq 1$}\}
    \]
    Hence, the volume of this region is given by
    \[
    \int_{\answer[given]{-1}}^{\answer[given]{1}} \int_{\answer[given]{-\sqrt{1-x^2}}}^{\answer[given]{\sqrt{1-x^2}}} \int_{\answer[given]{\sqrt{x^2+y^2}}}^{\answer[given]{1}}\d z \d y\d x.
    \]
    \end{explanation}
\end{example}

\begin{question}
  In the previous example, we integrated with respect to $z$, then
  $y$, then $x$. Set-up an integral that computes the volume of $R$
  that integrates with respect to $x$, then $y$, then $z$.
  \begin{prompt}
    \[
    \int_R \d V = \int_{\answer{0}}^{\answer{2}}\int_{\answer{-z}}^{\answer{z}} \int_{\answer{-\sqrt{z^2-y^2}}}^{\answer{\sqrt{z^2-y^2}}} \d x \d y \d z
    \]
  \end{prompt}
\end{question}

\section{When the integrand isn't one}

What if the integrand isn't $1$? Computationally, you just do what you
did before, but you have an additional antiderivative to
compute. Conceptually, much more can be happening. We'll see this soon.

\end{document}
