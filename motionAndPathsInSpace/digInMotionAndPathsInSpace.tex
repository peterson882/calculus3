\documentclass{ximera}

%\usepackage{todonotes}

\newcommand{\todo}{}


\graphicspath{
{./}
{../functionsOfSeveralVariables/}
{../normalVectors/}
{../lagrangeMultipliers/}
}


\usepackage{tkz-euclide}
\tikzset{>=stealth} %% cool arrow head
\tikzset{shorten <>/.style={ shorten >=#1, shorten <=#1 } } %% allows shorter vectors

\usetikzlibrary{backgrounds} %% for boxes around graphs
\usetikzlibrary{shapes,positioning}  %% Clouds and stars
\usetikzlibrary{matrix} %% for matrix
\usepgfplotslibrary{polar} %% for polar plots
\usetkzobj{all}
\usepackage[makeroom]{cancel} %% for strike outs
%\usepackage{mathtools} %% for pretty underbrace % Breaks Ximera
\usepackage{multicol}





\usepackage{array}
\setlength{\extrarowheight}{+.1cm}   
\newdimen\digitwidth
\settowidth\digitwidth{9}
\def\divrule#1#2{
\noalign{\moveright#1\digitwidth
\vbox{\hrule width#2\digitwidth}}}





\newcommand{\RR}{\mathbb R}
\newcommand{\R}{\mathbb R}
\newcommand{\N}{\mathbb N}
\newcommand{\Z}{\mathbb Z}

\newcommand{\sage}{\textsf{SageMath}}


%\renewcommand{\d}{\,d\!}
\renewcommand{\d}{\mathop{}\!d}
\newcommand{\dd}[2][]{\frac{\d #1}{\d #2}}
\newcommand{\pp}[2][]{\frac{\partial #1}{\partial #2}}
\renewcommand{\l}{\ell}
\newcommand{\ddx}{\frac{d}{\d x}}

\newcommand{\zeroOverZero}{\ensuremath{\boldsymbol{\tfrac{0}{0}}}}
\newcommand{\inftyOverInfty}{\ensuremath{\boldsymbol{\tfrac{\infty}{\infty}}}}
\newcommand{\zeroOverInfty}{\ensuremath{\boldsymbol{\tfrac{0}{\infty}}}}
\newcommand{\zeroTimesInfty}{\ensuremath{\small\boldsymbol{0\cdot \infty}}}
\newcommand{\inftyMinusInfty}{\ensuremath{\small\boldsymbol{\infty - \infty}}}
\newcommand{\oneToInfty}{\ensuremath{\boldsymbol{1^\infty}}}
\newcommand{\zeroToZero}{\ensuremath{\boldsymbol{0^0}}}
\newcommand{\inftyToZero}{\ensuremath{\boldsymbol{\infty^0}}}



\newcommand{\numOverZero}{\ensuremath{\boldsymbol{\tfrac{\#}{0}}}}
\newcommand{\dfn}{\textbf}
%\newcommand{\unit}{\,\mathrm}
\newcommand{\unit}{\mathop{}\!\mathrm}
\newcommand{\eval}[1]{\bigg[ #1 \bigg]}
\newcommand{\seq}[1]{\left( #1 \right)}
\renewcommand{\epsilon}{\varepsilon}
\renewcommand{\iff}{\Leftrightarrow}

\DeclareMathOperator{\arccot}{arccot}
\DeclareMathOperator{\arcsec}{arcsec}
\DeclareMathOperator{\arccsc}{arccsc}
\DeclareMathOperator{\si}{Si}
\DeclareMathOperator{\proj}{\vec{proj}}
\DeclareMathOperator{\scal}{scal}
\DeclareMathOperator{\sign}{sign}


%% \newcommand{\tightoverset}[2]{% for arrow vec
%%   \mathop{#2}\limits^{\vbox to -.5ex{\kern-0.75ex\hbox{$#1$}\vss}}}
\newcommand{\arrowvec}{\overrightarrow}
%\renewcommand{\vec}[1]{\arrowvec{\mathbf{#1}}}
\renewcommand{\vec}{\mathbf}
\newcommand{\veci}{{\boldsymbol{\hat{\imath}}}}
\newcommand{\vecj}{{\boldsymbol{\hat{\jmath}}}}
\newcommand{\veck}{{\boldsymbol{\hat{k}}}}
\newcommand{\vecl}{\boldsymbol{\l}}
\newcommand{\utan}{\mathbf{\hat{t}}}
\newcommand{\unormal}{\mathbf{\hat{n}}}
\newcommand{\ubinormal}{\mathbf{\hat{b}}}

\newcommand{\dotp}{\bullet}
\newcommand{\cross}{\boldsymbol\times}
\newcommand{\grad}{\boldsymbol\nabla}
\newcommand{\divergence}{\grad\dotp}
\newcommand{\curl}{\grad\cross}
%\DeclareMathOperator{\divergence}{divergence}
%\DeclareMathOperator{\curl}[1]{\grad\cross #1}
\newcommand{\lto}{\mathop{\longrightarrow\,}\limits}


\colorlet{textColor}{black} 
\colorlet{background}{white}
\colorlet{penColor}{blue!50!black} % Color of a curve in a plot
\colorlet{penColor2}{red!50!black}% Color of a curve in a plot
\colorlet{penColor3}{red!50!blue} % Color of a curve in a plot
\colorlet{penColor4}{green!50!black} % Color of a curve in a plot
\colorlet{penColor5}{orange!80!black} % Color of a curve in a plot
\colorlet{fill1}{penColor!20} % Color of fill in a plot
\colorlet{fill2}{penColor2!20} % Color of fill in a plot
\colorlet{fillp}{fill1} % Color of positive area
\colorlet{filln}{penColor2!20} % Color of negative area
\colorlet{fill3}{penColor3!20} % Fill
\colorlet{fill4}{penColor4!20} % Fill
\colorlet{fill5}{penColor5!20} % Fill
\colorlet{gridColor}{gray!50} % Color of grid in a plot

\newcommand{\surfaceColor}{violet}
\newcommand{\surfaceColorTwo}{redyellow}
\newcommand{\sliceColor}{greenyellow}




\pgfmathdeclarefunction{gauss}{2}{% gives gaussian
  \pgfmathparse{1/(#2*sqrt(2*pi))*exp(-((x-#1)^2)/(2*#2^2))}%
}


%%%%%%%%%%%%%
%% Vectors
%%%%%%%%%%%%%

%% Simple horiz vectors
\renewcommand{\vector}[1]{\left\langle #1\right\rangle}


%% %% Complex Horiz Vectors with angle brackets
%% \makeatletter
%% \renewcommand{\vector}[2][ , ]{\left\langle%
%%   \def\nextitem{\def\nextitem{#1}}%
%%   \@for \el:=#2\do{\nextitem\el}\right\rangle%
%% }
%% \makeatother

%% %% Vertical Vectors
%% \def\vector#1{\begin{bmatrix}\vecListA#1,,\end{bmatrix}}
%% \def\vecListA#1,{\if,#1,\else #1\cr \expandafter \vecListA \fi}

%%%%%%%%%%%%%
%% End of vectors
%%%%%%%%%%%%%

%\newcommand{\fullwidth}{}
%\newcommand{\normalwidth}{}



%% makes a snazzy t-chart for evaluating functions
%\newenvironment{tchart}{\rowcolors{2}{}{background!90!textColor}\array}{\endarray}

%%This is to help with formatting on future title pages.
\newenvironment{sectionOutcomes}{}{} 



%% Flowchart stuff
%\tikzstyle{startstop} = [rectangle, rounded corners, minimum width=3cm, minimum height=1cm,text centered, draw=black]
%\tikzstyle{question} = [rectangle, minimum width=3cm, minimum height=1cm, text centered, draw=black]
%\tikzstyle{decision} = [trapezium, trapezium left angle=70, trapezium right angle=110, minimum width=3cm, minimum height=1cm, text centered, draw=black]
%\tikzstyle{question} = [rectangle, rounded corners, minimum width=3cm, minimum height=1cm,text centered, draw=black]
%\tikzstyle{process} = [rectangle, minimum width=3cm, minimum height=1cm, text centered, draw=black]
%\tikzstyle{decision} = [trapezium, trapezium left angle=70, trapezium right angle=110, minimum width=3cm, minimum height=1cm, text centered, draw=black]


\title[Dig-In:]{Motion and paths in space}


\outcome{}


\begin{document}
\begin{abstract}
\end{abstract}
\maketitle

We know that if $v$ is a function that represents the \dfn{veloctiy}
(signed speed) of an object at time $t$, then
\begin{itemize}
\item $v'(t)$ tells us the \dfn{acceleration} (instaneous change in
  velocity) of the object, and
\item $\int_a^b v(t) \d t$ tells us the \dfn{displacement} (position
  with respect to an origin) of the object.
\end{itemize}

There is a similar story to be told with vector-valued functions.

\begin{definition}
Let $\vec{p}(t)$ be the position of some object in $\R^2$ or $\R^3$:
\begin{itemize}
\item \dfn{velocity}, denoted $\vec{v}(t)$, is the instantaneous rate
  of change in position , and so $\vec{v}(t) = \vec{p}'(t)$;
\item \dfn{speed} is the magnitude of velocity, $|\vec v(t)|$;
\item \dfn{acceleration}, denoted $\vec{a}(t)$, is the instantaneous
  rate of change in velocity, and so $\vec a(t) = \vec{v}'(t) =
  \vec{p}''(t)$;
\end{itemize}
\end{definition}

\begin{question}
  An object is moving according to the function $\vec{p}(t) = \vector{
    t^2-t,t^2+t}$, where $-3\leq t\leq 3$. Let distances be measured in
  feet and time measured in seconds. What is the velocity of our object at time $t$?
  \begin{prompt}
    The velocity is $\vector{\answer{2t-1},\answer{2t+1}}$ feet per
    second.
  \end{prompt}
  \begin{question}
    What is the acceleration of our object at time $t$?
    \begin{prompt}
      The acceleration is $\vector{\answer{2},\answer{2}}$ feet per second-squared.
    \end{prompt}
    \begin{question}
      What is speed of our object at time $t$?
      \begin{prompt}
        The speed of our object is $\answer{\sqrt{8t^2+2}}$ feet per second. 
      \end{prompt}
    \begin{question}
      When is the object's speed minimized?
      \begin{prompt}
        The speed is minimized when $t=\answer{0}$.
      \end{prompt}
    \end{question}
    \end{question}
  \end{question}
\end{question}

Now let's see an example.
\begin{example}
  You whirl a ball, attached to a string, above your head in a
  counter-clockwise circle. The ball follows a circular path and makes
  $2$ revolutions per second. The string has length $2$ft.
\begin{enumerate}
\item Find the position function $\vec{p}(t)$ that describes this
  situation.
\item Find the acceleration of the ball and derive a physical
  interpretation of it.
\item A tree stands 10ft in front of you. At what $t$-values should
  you release the string so that the ball hits the tree?
\end{enumerate}
\begin{explanation}
  \begin{enumerate}
  \item The ball whirls in a circle. Since the string is $2$ft long,
    the radius of the circle is $2$. We start by writing down a
    position function that describes
    \begin{enumerate}
    \item a circle with radius $2$,
    \item centered at the origin of the $(x,y)$-plane,
    \item that makes a full revolution every $2\pi$ seconds.
    \end{enumerate}
    \[
    \vector{\answer[given]{2\cos(t)}, \answer[given]{2\sin t}}
    \]
    However, we want HERE HERE HERE HERE
    not two revolutions
    per second. We modify the period of the trigonometric functions to
    be 1/2 by multiplying $t$ by $4\pi$. The final position function
    is thus $$\vrt = \vector{2\cos (4\pi t), 2\sin (4\pi t)}.$$ (Plot
    this for $0\leq t\leq 1/2$ to verify that one revolution is made
    in 1/2 a second.)
  \item To find $\vat$, we derive $\vrt$ twice.
    \begin{align*}
      \vvt = \vrp(t) &= \vector{-8\pi \sin (4\pi t), 8\pi \cos (4\pi t)}\\
      \vat =\vrp'(t) &= \vector{-32\pi^2 \cos (4\pi t), -32\pi^2 \sin (4\pi t) } \\
      &= -32\pi^2\vector{\cos (4\pi t), \sin (4\pi t)}.
    \end{align*}
    Note how $\vat$ is parallel to \vrt, but has a different magnitude
    and points in the opposite direction. Why is this?
    
    Recall the classic phyics equation, ``Force $=$ mass $\times$
    acceleration.'' A force acting on a mass induces acceleration
    (i.e., the mass moves); acceleration acting on a mass induces a
    force (gravity gives our mass a \emph{weight}). Thus force and
    acceleration are closely related. A moving ball ``wants'' to
    travel in a straight line. Why does the ball in our example move
    in a circle? It is attached to the boy's hand by a string. The
    string applies a force to the ball, affecting it's motion: the
    string \emph{accelerates} the ball. This is not acceleration in
    the sense of ``it travels faster;'' rather, this acceleration is
    \textbf{changing the velocity} of the ball. In what direction is
    this force/acceleration being applied? In the direction of the
    string, towards the boy's hand.
	
    The magnitude of the acceleration is related to the speed at which
    the ball is traveling. A ball whirling quickly is rapidly changing
    direction/velocity. When velocity is changing rapidly, the
    acceleration must be ``large.''
	
  \item When the boy releases the string, the string no longer applies
    a force to the ball, meaning acceleration is $\vec 0$ and the ball
    can now move in a straight line in the direction of $\vec v(t)$.
    \drawexampleline
	
    Let $t=t_0$ be the time when the boy lets go of the string. The
    ball will be at $\vec{p}(t_0)$, traveling in the direction of
    $\vec v(t_0)$. We want to find $t_0$ so that this line contains
    the point $(0,10)$ (since the tree is 10ft directly in front of
    the boy).  \mfigure{.45}{Modeling the flight of a ball in Example
      \ref{ex_motion3}.}{fig:motion3}{figures/figmotion3}
	
    There are many ways to find this time value. We choose one that is
    relatively simple computationally. As shown in Figure
    \ref{fig:motion3}, the vector from the release point to the tree
    is $\vector{0,10} - \vec{p}(t_0)$. This line segment is tangent to
    the circle, which means it is also perpendicular to $\vec{p}(t_0)$
    itself, so their dot product is 0.
    
    \begin{align*}
      \vec{p}(t_0) \cdot \big(\vector{0,10} - \vec{p}(t_0)\big)&=0\\
      \vector{2\cos (4\pi t_0), 2\sin (4\pi t_0)} \cdot\vector{-2\cos(4\pi t_0),10-2\sin (4\pi t_0)}&=0\\
      -4\cos^2(4\pi t_0) + 20\sin (4\pi t_0)-4\sin^2(4\pi t_0)&= 0\\
      20\sin (4\pi t_0) - 4 &=0\\ \sin (4\pi t_0) &=1/5\\
      4\pi t_0 &= \sin^{-1}(1/5)\\
      4\pi t_0 &\approx 0.2 + 2\pi n,
    \end{align*}
    where $n$ is an integer. Solving for $t_0$ we have:
    \[
    t_0 &\approx 0.016 + n/2
    \]
    This is a wonderful formula. Every 1/2 second after $t=0.016$s the
    boy can release the string (since the ball makes 2 revolutions per
    second, he has two chances each second to release the ball).
\end{enumerate}
\end{explanation}
\end{example}

\end{document}
\example{ex_motion4}{Analyzing motion in space}{
An object moves in spiral with position function $\vrt = \vector{\cos t, \sin t, t}$, where distances are measured in meters and time is in minutes. Describe the object's speed and acceleration at time $t$.
}
{With $\vrt = \vector{\cos t,\sin t, t}$, we have:
\begin{align*}
\vvt &= \vector{-\sin t, \cos t, 1} \quad \text{and} \\
\vat &= \vector{-\cos t, -\sin t, 0}.
\end{align*}

The speed of the object is $\norm{\vvt} = \sqrt{(-\sin t)^2+\cos^2t+1} = \sqrt{2}$m/min; it moves at a constant speed. Note that the object does not accelerate in the $z$-direction, but rather moves up at a constant rate of 1m/min.
}\\

The objects in Examples \ref{ex_motion3} and \ref{ex_motion4} traveled at a constant speed. That is, $\norm{\vvt} = c$ for some constant $c$. Recall Theorem \ref{thm:vects_of_constant_length}, which states that if a vector-valued function \vrt\ has constant length, then \vrt\ is perpendicular to its derivative: $\vrt\cdot\vrp(t) = 0$. In these examples, the velocity function has constant length, therefore we can conclude that the velocity is perpendicular to the acceleration: $\vvt\cdot\vat = 0$. A quick check verifies this.\index{vector-valued function!of constant length}

There is an intuitive understanding of this. If acceleration is parallel to velocity, then it is only affecting the object's speed; it does not change the direction of travel. (For example, consider a dropped stone. Acceleration and velocity are parallel -- straight down -- and the direction of velocity never changes, though speed does increase.) If acceleration is not perpendicular to velocity, then there is some acceleration in the direction of travel, influencing the speed. If speed is constant, then acceleration must be orthogonal to velocity, as it then only affects direction, and not speed.

\keyidea{idea:constant_speed}{Objects With Constant Speed}
{If an object moves with constant speed, then its velocity and acceleration vectors are orthogonal. That is, $\vvt\cdot\vat=0$.
\index{vector-valued function!of constant length}
}


\section{Projectile Motion}

An important application of vector-valued position functions is
\emph{projectile motion}: the motion of objects under the influence of
gravity. We will measure time in seconds, and distances will either be
in meters or feet. We will show that we can completely describe the
path of such an object knowing its initial position and initial
velocity (i.e., where it \emph{is} and where it \emph{is going.})
\index{vector-valued function!projectile motion}\index{projectile
  motion}

Suppose an object has initial position $\vec{p}(0) = \vector{x_0,y_0}$
and initial velocity $\vec v(0) = \vector{v_x,v_y}$. It is customary
to rewrite $\vec v(0)$ in terms of its speed $v_0$ and direction $\vec
u$, where $\vec u$ is a unit vector. Recall all unit vectors in
$\mathbb{R}^2$ can be written as $\vector{\cos \theta,\sin \theta}$,
where $\theta$ is an angle measure counter-clockwise from the
$x$-axis. (We refer to $\theta$ as the \textbf{angle of
  elevation.}\index{angle of elevation}) Thus $\vec v(0) =
v_0\vector{\cos \theta,\sin \theta}.$

Since the acceleration of the object is known, namely $\vat =
\vector{0,-g}$, where $g$ is the gravitational constant, we can find
$\vrt$ knowing our two initial conditions. We first find $\vvt$:
\mnote{.5}{\textbf{Note:} In this text we use $g=32$ft/s when using
  Imperial units, and $g=9.8$m/s when using SI units.}
\begin{align*}
\vec v(t) &= \int \vat \d t\\
\vvt &= \int \vector{0,-g} \d t\\
\vvt &= \vector{0,-gt} + \vec C.
\end{align*}
Knowing $\vec v(0) = v_0\vector{\cos \theta,\sin \theta}$, we have $\vec C = v_0\vector{\cos t,\sin t}$ and so
$$\vec v(t) = \vector{\rule{0pt}{9pt} v_0\cos \theta, -gt+v_0\sin\theta}.$$
We integrate once more to find $\vrt$:
\begin{align*}
\vrt &= \int \vvt\d t \\
\vrt &= \int \vector{\rule{0pt}{9pt} v_0\cos \theta, -gt+v_0\sin\theta}\d t\\
\vrt &= \vector{\big(v_0\cos \theta\big)t, -\frac12gt^2+\big(v_0\sin\theta\big)t} + \vec C
\intertext{Knowing $\vec{p}(0) = \vector{x_0,y_0}$, we conclude $\vec C = \vector{x_0,y_0}$ and }
\vrt &= \vector{\big(v_0\cos \theta\big)t+x_0\ , -\frac12gt^2+\big(v_0\sin\theta\big)t+y_0\ } %\\
%\vrt &= \vector{0,-\frac12g} t^2 + v_0\vector{\cos\theta,\sin \theta} t + \vector{x_0,y_0}.
\end{align*}
%\enlargethispage{2\baselineskip}

\keyidea{idea:projectile}{Projectile Motion}
{The position function of a projectile propelled from an initial position of $\vec{p}_0=\vector{x_0,y_0}$, with initial speed $v_0$, with angle of elevation $\theta$ and neglecting all accelerations but gravity is 
\index{vector-valued function!projectile motion}\index{projectile motion}
$$\vrt = \vector{\big(v_0\cos \theta\big)t+x_0\ , -\frac12gt^2+\big(v_0\sin\theta\big)t+y_0\ }.$$
Letting $\vec v_0 = v_0\vector{\cos \theta,\sin \theta}$,\ \ $\vrt$ can be written as
$$\vrt = \vector{0,-\frac12gt^2} + \vec v_0t+\vec{p}_0.$$
}

We demonstrate how to use this position function in the next two examples.\\

\example{ex_motion5}{Projectile Motion}{
 Sydney shoots her Red Ryder\textregistered\ bb gun across level ground from an elevation of 4ft, where the barrel of the gun makes a $5^\circ$ angle with the horizontal. Find how far the bb travels before landing, assuming the bb is fired at the advertised rate of 350ft/s and ignoring air resistance.}
{A direct application of Key Idea \ref{idea:projectile} gives
\begin{align*}
\vrt &= \vector{(350\cos 5^\circ)t, -16t^2 + (350\sin 5^\circ)t + 4}\\
&\approx \vector{346.67t, -16t^2+30.50t+4},
\end{align*}
where we set her initial position to be $\vector{0,4}$.
We need to find \emph{when} the bb lands, then we can find \emph{where}. We accomplish this by setting the $y$-component equal to 0 and solving for $t$:
\begin{align*}
-16t^2+30.50t+4 &= 0 \\
t &= \frac{-30.50 \pm \sqrt{30.50^2-4(-16)(4)}}{-32}\\
t &\approx 2.03s.
\end{align*}
(We discarded a negative solution that resulted from our quadratic equation.) 

We have found that the bb lands 2.03s after firing; with $t=2.03$, we find the $x$-component of our position function is $346.67(2.03) = 703.74$ft. The bb lands about 704 feet away.
}\\

\example{ex_motion61}{Projectile Motion}{
Alex holds his sister's bb gun at a height of 3ft and wants to shoot a target that is 6ft above the ground, 25ft away. At what angle should he hold the gun to hit his target? (We still assume the muzzle velocity is 350ft/s.)
}
{The position function for the path of Alex's bb is
$$\vrt = \vector{(350\cos \theta)t, -16t^2+(350\sin\theta)t+3}.$$ We need to find $\theta$ so that $\vrt =\vector{25,6}$ for some value of $t$. That is, we want to find $\theta$ and $t$ such that 
$$(350\cos\theta)t = 25 \quad \text{and}\quad -16t^2+(350\sin\theta)t+3 = 6.$$
This is not trivial (though not ``hard''). We start by solving each equation for $\cos\theta$ and $\sin \theta$, respectively.
$$\cos\theta = \frac{25}{350t} \quad \text{and} \quad \sin\theta = \frac{3+16t^2}{350t}.$$
Using the Pythagorean Identity $\cos^2\theta+\sin^2\theta=1$, we have
\begin{align*}
\left(\frac{25}{350t}\right)^2 + \left(\frac{3+16t^2}{350t}\right)^2 &=1
\intertext{Multiply both sides by $(350t)^2$:}
25^2 + (3+16t^2)^2 &=350^2t^2\\
256t^4-122,404t^2+634 &=0
%\end{align*}
\intertext{This is a quadratic \emph{in} $t^2$. That is, we can apply the quadratic formula  to find $t^2$, then solve for $t$ itself.}
%\begin{align*}
t^2 &= \frac{122,404\pm\sqrt{122,404^2-4(256)(634)}}{512}\\
t^2 &= 0.0052,\ 478.135\\
t &=  \pm 0.072,\ \pm 21.866
\end{align*}
Clearly the negative $t$ values do not fit our context, so we have $t=0.072$ and $t=21.866$. Using $\cos \theta = 25/(350 t)$, we can solve for $\theta$:
\begin{align*}
\theta &= \cos^{-1}\left(\frac{25}{350\cdot 0.072}\right)\quad \text{and}\quad \cos^{-1}\left(\frac{25}{350\cdot 21.866}\right)\\
\theta &= 7.03^\circ \quad \text{and} \quad 89.8^\circ.
\end{align*}
Alex has two choices of angle. He can hold the rifle at an angle of about $7^\circ$ with the horizontal and hit his target $0.07$s after firing, or he can hold his rifle almost straight up, with an angle of $89.8^\circ$, where he'll hit his target about 22s later. The first option is clearly the option he should choose.
}\\

\noindent\textbf{\vector{rge Distance Traveled}\\

Consider a driver who sets her cruise-control to 60mph, and travels at this speed for an hour. We can ask:
\begin{enumerate}
	\item How far did the driver travel?
	\item	How far from her starting position is the driver?
\end{enumerate} 
The first is easy to answer: she traveled 60 miles. The second is impossible to answer with the given information. We do not know if she traveled in a straight line, on an oval racetrack, or along a slowly-winding highway.

This highlights an important fact: to compute distance traveled, we need only to know the speed, given by $\norm{\vvt}$.

\theorem{thm:distance_traveled}{Distance Traveled}
{Let $\vvt$ be a velocity function for a moving object. The distance traveled by the object on $[a,b]$ is:
\index{distance!traveled}\index{vector-valued function!distance traveled}\index{integration!distance traveled}
$$\text{distance traveled} = \int_a^b \norm{\vvt}\d t.$$
}
Note that this is just a restatement of Theorem \ref{thm:vvf_arc_length}: arc length is the same as distance traveled, just viewed in a different context.\\

%Given a position function $\vrt = \vector{f(t),g(t)}$, Theorem \ref{thm:distance_traveled}
 %states that the distance traveled is
%$$\int_a^b \sqrt{\big(\fp(t)\big)^2+\big(g'(t)\big)^2}\d t.$$
%Comparing this to Key Idea \ref{idea:arc_length_parametric} we see that \emph{distance traveled} measures the same thing as \emph{arc length}, just in different contexts.\\

\enlargethispage{2\baselineskip}
\example{ex_motion6}{Distance Traveled, Displacement, and Average Speed}{
A particle moves in space with position function $\vrt = \vector{t,t^2,\sin (\pi t)}$ on $[-2,2]$, where $t$ is measured in seconds and distances are in meters. Find:
\begin{enumerate}
	\item The distance traveled by the particle on $[-2,2]$.
	\item	The displacement of the particle on $[-2,2]$.
	\item	The particle's average speed.
\end{enumerate}
}
{\begin{enumerate}
	\item We use Theorem \ref{thm:distance_traveled} to establish the integral:
	\begin{align*}
	\text{distance traveled} &= \int_{-2}^2 \norm{\vvt}\d t \\
							&= \int_{-2}^2 \sqrt{1+(2t)^2+ \pi^2\cos^2(\pi t)}\d t.
	\end{align*}
	This cannot be solved in terms of elementary functions so we turn to numerical integration, finding the distance to be 12.88m.
	
	\item		The displacement is the vector $$\vec{p}(2)-\vec{p}(-2) = \vector{2,4,0} - \vector{-2,4,0} = \vector{4,0,0}.$$ That is, the particle ends with an $x$-value increased by 4 and with $y$- and $z$-values the same (see Figure \ref{fig:motion6}).
	
	\item		We found above that the particle traveled 12.88m over 4 seconds. We can compute average speed by dividing: 12.88/4 = 3.22m/s. 
	
	\mtable{.7}{The path of the particle, from two perspectives, in Example \ref{ex_motion6}.}{fig:motion6}{%
	\begin{tabular}{c}
	\myincludegraphics[scale=1.25]{figures/figmotion6}\\[-20pt]
	(a)\\[20pt]
	\myincludegraphics[scale=1.25]{figures/figmotion6b}\\[-20pt]
	(b)
	\end{tabular}
	}
	We should also consider Definition \ref{def:av_val} of Section \ref{sec:FTC}, which says that the average value of a function $f$ on $[a,b]$ is $\frac{1}{b-a}\int_a^b f(x)\d x$. In our context, the average value of the speed is
	$$\text{average speed} = \frac{1}{2-(-2)}\int_{-2}^2 \norm{\vvt}\d t \approx \frac14 12.88 = 3.22\text{m/s}.$$
	Note how the physical context of a particle traveling gives meaning to a more abstract concept learned earlier.
\end{enumerate}
\vskip-1.5\baselineskip
}\\

In Definition \ref{def:av_val} we defined the average value of a function $f(x)$ on $[a,b]$ to be $$ \frac{1}{b-a}\int_a^bf(x)\d x.$$
Note how in Example \ref{ex_motion6} we computed the average speed as
$$\frac{\text{distance traveled}}{\text{travel time}} = \frac1{2-(-2)}\int_{-2}^2\norm{\vvt}\d t;$$
that is, we just found the average value of $\norm{\vvt}$ on $[-2,2]$.

Likewise, given position function $\vrt$, the average velocity on $[a,b]$ is
$$\frac{\text{displacement}}{\text{travel time}} = \frac1{b-a}\int_a^b \vec{r}'(t)\d t = \frac{\vec{p}(b)-\vec{p}(a)}{b-a};$$
that is, it is the average value of $\vec{p}'(t)$, or $\vvt$, on $[a,b]$.\\

The next two sections investigate more properties of the graphs of vector-valued functions and we'll apply these new ideas to what we just learned about motion.



Cycloid as motivating example? Given a wheel of radius $R$... 

\end{document}
