\documentclass{ximera}

%\usepackage{todonotes}

\newcommand{\todo}{}


\graphicspath{
{./}
{../functionsOfSeveralVariables/}
{../normalVectors/}
{../lagrangeMultipliers/}
}


\usepackage{tkz-euclide}
\tikzset{>=stealth} %% cool arrow head
\tikzset{shorten <>/.style={ shorten >=#1, shorten <=#1 } } %% allows shorter vectors

\usetikzlibrary{backgrounds} %% for boxes around graphs
\usetikzlibrary{shapes,positioning}  %% Clouds and stars
\usetikzlibrary{matrix} %% for matrix
\usepgfplotslibrary{polar} %% for polar plots
\usetkzobj{all}
\usepackage[makeroom]{cancel} %% for strike outs
%\usepackage{mathtools} %% for pretty underbrace % Breaks Ximera
\usepackage{multicol}





\usepackage{array}
\setlength{\extrarowheight}{+.1cm}   
\newdimen\digitwidth
\settowidth\digitwidth{9}
\def\divrule#1#2{
\noalign{\moveright#1\digitwidth
\vbox{\hrule width#2\digitwidth}}}





\newcommand{\RR}{\mathbb R}
\newcommand{\R}{\mathbb R}
\newcommand{\N}{\mathbb N}
\newcommand{\Z}{\mathbb Z}

\newcommand{\sage}{\textsf{SageMath}}


%\renewcommand{\d}{\,d\!}
\renewcommand{\d}{\mathop{}\!d}
\newcommand{\dd}[2][]{\frac{\d #1}{\d #2}}
\newcommand{\pp}[2][]{\frac{\partial #1}{\partial #2}}
\renewcommand{\l}{\ell}
\newcommand{\ddx}{\frac{d}{\d x}}

\newcommand{\zeroOverZero}{\ensuremath{\boldsymbol{\tfrac{0}{0}}}}
\newcommand{\inftyOverInfty}{\ensuremath{\boldsymbol{\tfrac{\infty}{\infty}}}}
\newcommand{\zeroOverInfty}{\ensuremath{\boldsymbol{\tfrac{0}{\infty}}}}
\newcommand{\zeroTimesInfty}{\ensuremath{\small\boldsymbol{0\cdot \infty}}}
\newcommand{\inftyMinusInfty}{\ensuremath{\small\boldsymbol{\infty - \infty}}}
\newcommand{\oneToInfty}{\ensuremath{\boldsymbol{1^\infty}}}
\newcommand{\zeroToZero}{\ensuremath{\boldsymbol{0^0}}}
\newcommand{\inftyToZero}{\ensuremath{\boldsymbol{\infty^0}}}



\newcommand{\numOverZero}{\ensuremath{\boldsymbol{\tfrac{\#}{0}}}}
\newcommand{\dfn}{\textbf}
%\newcommand{\unit}{\,\mathrm}
\newcommand{\unit}{\mathop{}\!\mathrm}
\newcommand{\eval}[1]{\bigg[ #1 \bigg]}
\newcommand{\seq}[1]{\left( #1 \right)}
\renewcommand{\epsilon}{\varepsilon}
\renewcommand{\iff}{\Leftrightarrow}

\DeclareMathOperator{\arccot}{arccot}
\DeclareMathOperator{\arcsec}{arcsec}
\DeclareMathOperator{\arccsc}{arccsc}
\DeclareMathOperator{\si}{Si}
\DeclareMathOperator{\proj}{\vec{proj}}
\DeclareMathOperator{\scal}{scal}
\DeclareMathOperator{\sign}{sign}


%% \newcommand{\tightoverset}[2]{% for arrow vec
%%   \mathop{#2}\limits^{\vbox to -.5ex{\kern-0.75ex\hbox{$#1$}\vss}}}
\newcommand{\arrowvec}{\overrightarrow}
%\renewcommand{\vec}[1]{\arrowvec{\mathbf{#1}}}
\renewcommand{\vec}{\mathbf}
\newcommand{\veci}{{\boldsymbol{\hat{\imath}}}}
\newcommand{\vecj}{{\boldsymbol{\hat{\jmath}}}}
\newcommand{\veck}{{\boldsymbol{\hat{k}}}}
\newcommand{\vecl}{\boldsymbol{\l}}
\newcommand{\utan}{\mathbf{\hat{t}}}
\newcommand{\unormal}{\mathbf{\hat{n}}}
\newcommand{\ubinormal}{\mathbf{\hat{b}}}

\newcommand{\dotp}{\bullet}
\newcommand{\cross}{\boldsymbol\times}
\newcommand{\grad}{\boldsymbol\nabla}
\newcommand{\divergence}{\grad\dotp}
\newcommand{\curl}{\grad\cross}
%\DeclareMathOperator{\divergence}{divergence}
%\DeclareMathOperator{\curl}[1]{\grad\cross #1}
\newcommand{\lto}{\mathop{\longrightarrow\,}\limits}


\colorlet{textColor}{black} 
\colorlet{background}{white}
\colorlet{penColor}{blue!50!black} % Color of a curve in a plot
\colorlet{penColor2}{red!50!black}% Color of a curve in a plot
\colorlet{penColor3}{red!50!blue} % Color of a curve in a plot
\colorlet{penColor4}{green!50!black} % Color of a curve in a plot
\colorlet{penColor5}{orange!80!black} % Color of a curve in a plot
\colorlet{fill1}{penColor!20} % Color of fill in a plot
\colorlet{fill2}{penColor2!20} % Color of fill in a plot
\colorlet{fillp}{fill1} % Color of positive area
\colorlet{filln}{penColor2!20} % Color of negative area
\colorlet{fill3}{penColor3!20} % Fill
\colorlet{fill4}{penColor4!20} % Fill
\colorlet{fill5}{penColor5!20} % Fill
\colorlet{gridColor}{gray!50} % Color of grid in a plot

\newcommand{\surfaceColor}{violet}
\newcommand{\surfaceColorTwo}{redyellow}
\newcommand{\sliceColor}{greenyellow}




\pgfmathdeclarefunction{gauss}{2}{% gives gaussian
  \pgfmathparse{1/(#2*sqrt(2*pi))*exp(-((x-#1)^2)/(2*#2^2))}%
}


%%%%%%%%%%%%%
%% Vectors
%%%%%%%%%%%%%

%% Simple horiz vectors
\renewcommand{\vector}[1]{\left\langle #1\right\rangle}


%% %% Complex Horiz Vectors with angle brackets
%% \makeatletter
%% \renewcommand{\vector}[2][ , ]{\left\langle%
%%   \def\nextitem{\def\nextitem{#1}}%
%%   \@for \el:=#2\do{\nextitem\el}\right\rangle%
%% }
%% \makeatother

%% %% Vertical Vectors
%% \def\vector#1{\begin{bmatrix}\vecListA#1,,\end{bmatrix}}
%% \def\vecListA#1,{\if,#1,\else #1\cr \expandafter \vecListA \fi}

%%%%%%%%%%%%%
%% End of vectors
%%%%%%%%%%%%%

%\newcommand{\fullwidth}{}
%\newcommand{\normalwidth}{}



%% makes a snazzy t-chart for evaluating functions
%\newenvironment{tchart}{\rowcolors{2}{}{background!90!textColor}\array}{\endarray}

%%This is to help with formatting on future title pages.
\newenvironment{sectionOutcomes}{}{} 



%% Flowchart stuff
%\tikzstyle{startstop} = [rectangle, rounded corners, minimum width=3cm, minimum height=1cm,text centered, draw=black]
%\tikzstyle{question} = [rectangle, minimum width=3cm, minimum height=1cm, text centered, draw=black]
%\tikzstyle{decision} = [trapezium, trapezium left angle=70, trapezium right angle=110, minimum width=3cm, minimum height=1cm, text centered, draw=black]
%\tikzstyle{question} = [rectangle, rounded corners, minimum width=3cm, minimum height=1cm,text centered, draw=black]
%\tikzstyle{process} = [rectangle, minimum width=3cm, minimum height=1cm, text centered, draw=black]
%\tikzstyle{decision} = [trapezium, trapezium left angle=70, trapezium right angle=110, minimum width=3cm, minimum height=1cm, text centered, draw=black]


\title[Dig-In:]{Quadric surfaces}

\outcome{Define a quadric surface.}
\outcome{Identify basic quadric surfaces.}
\outcome{Use sections to identify surfaces.}
\outcome{Use the second derivative test to identify quadric surfaces.}

\begin{document}
\begin{abstract}
  We will get to know some basic quadric surfaces.
\end{abstract}
\maketitle

As we have seen, if we look at the set of points that satisfy an
equation
\[
F(x,y,z)=0
\]
where $F:\R^3\to\R$, we obtain a surface in $\R^3$. A basic class of
surfaces are the \textit{quadric surfaces}.

\begin{definition}
A \dfn{quadric surface} in $\R^3$ is a surface of the form
\[
Ax^2 + By^2 + Cz^2 + Dxy + Exz+ Fyz + Gx + Hy + I z + J = 0
\]
where $A$, $B$, $C$, $D$, $E$, $F$, $G$, $H$, $I$, and $J$ are
constants and at least one of $A$, $B$, $C$, $D$, $E$, or $F$ are
nonzero.
\end{definition}

\begin{question}
  Which of the following are quadric surfaces?
  \begin{selectAll}
    \choice[correct]{$x^2 = 0$}
    \choice{$y=0$}
    \choice[correct]{$z=y^2$}
  \end{selectAll}
\end{question}

\begin{warning}
  Do not confuse a \textit{quadric} with a quadratic, or quartic, as
  these are different beasts entirely.
\end{warning}

We will be interested in a special class of quadric surfaces, those
that arise naturally when computing the Taylor polynomial of a surface
$z=F(x,y)$ at a point $\vec{c}$ where:
\[
F^{(1,0)}(\vec{c}) = 0 = F^{(0,1)}(\vec{c})
\]
In this case, the quadric is of the form:
\begin{align*}
  z = &F^{(2,0)}(\vec{c})(x-c_1)^2 \\
  &+ F^{(0,2)}(\vec{c})(y-c_2)^2 \\
  &+ F^{(1,1)}(\vec{c}) (x-c_1)(y-c_2)\\
  &+ F(\vec{c})
\end{align*}
Why are we doing this?
\begin{quote}
  \textbf{Understanding quadric surfaces will help us find extrema of
    surfaces.}
\end{quote}

In what follows, we will study each shape by considering various
\textit{sections} of the surface.

\begin{definition}
  The \dfn{section} of a surface is the intersections of
  a surface with a plane.
\end{definition}

\begin{question}
  Consider the following surface:
  \[
  z = 6x^2 + y^2 + 5 xy
  \]
  Compute the section of the surface given by the plane $y=0$.
  \begin{prompt}
    \[
    z = \answer{6x^2}
    \]
  \end{prompt}
  \begin{question}
    Does this parabola open ``up'' or ``down?''
    \begin{prompt}
    \begin{multipleChoice}
      \choice[correct]{up}
      \choice{down}
    \end{multipleChoice}
    \end{prompt}
    \begin{question}
      Compute the section of the surface given by the plane $y=-2.5 x$.
      \begin{prompt}
        \[
        z = \answer{6x^2 + 6.25 x^2 - 12.5x^2}
        \]
      \end{prompt}
      \begin{question}
        Does this parabola open ``up'' or ``down?''
        \begin{prompt}
          \begin{multipleChoice}
            \choice{up}
            \choice[correct]{down}
          \end{multipleChoice}
        \end{prompt}
      \end{question}
    \end{question}
  \end{question}
\end{question}

  





\section{Elliptic paraboloids}

An \dfn{elliptic paraboloid} is a surface with graph:
\begin{image}
  \begin{tikzpicture}
    \begin{axis}%
      [width=175pt,tick label style={font=\scriptsize},axis on top,
	axis lines=center,
	view={145}{20},
	name=myplot,
	xtick=\empty,
	ytick=\empty,
	ztick=\empty,
	ymin=-2.5,ymax=2.5,
	xmin=-3.5,xmax=3.5,
	zmin=-.1, zmax=5.5,
	every axis x label/.style={at={(axis cs:\pgfkeysvalueof{/pgfplots/xmax},0,0)},xshift=-3pt,yshift=-3pt},
	xlabel={\scriptsize $x$},
	every axis y label/.style={at={(axis cs:0,\pgfkeysvalueof{/pgfplots/ymax},0)},xshift=5pt,yshift=-2pt},
	ylabel={\scriptsize $y$},
	every axis z label/.style={at={(axis cs:0,0,\pgfkeysvalueof{/pgfplots/zmax})},xshift=0pt,yshift=4pt},
	zlabel={\scriptsize $z$},
        colormap/cool
      ]
      
      \addplot3[domain=0:360,y domain=0:2,color=black!40,mesh,samples=40,samples y=10,very thin,z buffer=sort] ({2*cos(x)*y},{sin(x)*y},{y^2});
    \end{axis}
  \end{tikzpicture}
\end{image}
and equation:
\[
z=\pm\frac{x^2}{a^2}\pm\frac{y^2}{b^2}
\]
To understand this surface better consider the sections when:
\begin{itemize}
  \item $x=d$, in this case we now have $z = \pm\frac{d^2}{a^2} \pm
    \frac{y^2}{b^2}$, a parabola.
  \item $y=d$, in this case we now have $z = \pm\frac{x^2}{a^2} \pm
    \frac{d^2}{b^2}$, a parabola.
  \item $z=d$, in this case we now have $d = \pm\frac{x^2}{a^2} \pm
    \frac{y^2}{b^2}$, an ellipse.
\end{itemize}
\begin{image}
  \begin{tikzpicture}
    \begin{axis}%
      [width=175pt,tick label style={font=\scriptsize},axis on top,
	axis lines=center,
        clip=false,
	view={145}{20},
	name=myplot,
	xtick=\empty,
	ytick=\empty,
	ztick=\empty,
	ymin=-2.5,ymax=2.5,
	xmin=-3.5,xmax=3.5,
	zmin=-.1, zmax=5.5,
	every axis x label/.style={at={(axis cs:\pgfkeysvalueof{/pgfplots/xmax},0,0)},xshift=-3pt,yshift=-3pt},
	xlabel={\scriptsize $x$},
	every axis y label/.style={at={(axis cs:0,\pgfkeysvalueof{/pgfplots/ymax},0)},xshift=5pt,yshift=-2pt},
	ylabel={\scriptsize $y$},
	every axis z label/.style={at={(axis cs:0,0,\pgfkeysvalueof{/pgfplots/zmax})},xshift=0pt,yshift=4pt},
	zlabel={\scriptsize $z$},colormap/cool
      ]
      
      \addplot3[domain=0:360,y domain=0:2,mesh,samples=40,samples y=10,very thin,z buffer=sort] ({2*cos(x)*y},{sin(x)*y},{y^2});
      
      \addplot3[domain=170:360,dashed,ultra thick,smooth,samples y=0,black!60,%surf,%fill=white,
        samples=30,] ({2*cos(x)*sqrt(2)},{sin(x)*sqrt(2)},{2});

      \addplot3[domain=-2:0,dashed,ultra thick,smooth,samples y=0,black!60,%surf,%fill=white,
        samples=30,] ({2*x},{0},{x^2});

      \addplot3[domain=-2:0,dashed,ultra thick,smooth,samples y=0,black!60,%surf,%fill=white,
        samples=30,] ({0},{x},{x^2});
      
      \addplot3[domain=0:170,ultra thick,smooth,samples y=0,black,%surf,%fill=white,
        samples=30,] ({2*cos(x)*sqrt(2)},{sin(x)*sqrt(2)},{2});
      
      \addplot3[domain=0:2,ultra thick,smooth,samples y=0,black,%surf,%fill=white,
        samples=30,] ({2*x},{0},{x^2});

      \addplot3[domain=0:2,ultra thick,smooth,samples y=0,black,%surf,%fill=white,
          samples=30,] ({0},{x},{x^2});
      
        \draw (axis cs:4.3,0,6) node[align=center] (A1) {\scriptsize In plane\\[-4pt] \scriptsize $y=0$};
        \draw (axis cs:4,0,4) node (A2) {};
        \draw [->](A1)--(A2);
        
        \draw (axis cs:0,2,6.5) node[align=center] (B1) {\scriptsize In plane\\[-4pt] \scriptsize $x=0$};
        \draw (axis cs:0,2,4) node (B2) {};
        \draw [->](B1)--(B2);

        \draw (axis cs:-2,2,1) node[align=center] (C1) {\scriptsize In plane\\[-4pt] \scriptsize $z=d$};
        \draw (axis cs:-2.83,0,2) node [below] (C2) {};
        \draw [->](C1)--(C2);
        %\foreach \z in {-6.28,-4.71,-1.57,1.57,6.28}
        %{\addplot3[domain=-2:2,,thick,smooth,samples y=0,{\colortwo},%surf,%fill=white,
        %samples=30,] ({sin(deg(\z))},{x},{\z});
        %}
    \end{axis}
  \end{tikzpicture}
\end{image}
\[
\begin{array}{cc}
\textbf{Plane} & \textbf{Section} \\ \hline
x=d  & \text{Parabola} \\
y=d  & \text{Parabola}\\
z=d  & \text{Ellipse}
\end{array}
\]




\section{Hyperbolic paraboloids}

A \dfn{hyperbolic paraboloid} is a surface with graph:
\begin{image}
  \begin{tikzpicture}[>=stealth]
    \begin{axis}%
      [width=175pt,tick label style={font=\scriptsize},axis on top,
	axis lines=center,
	view={155}{30},
	name=myplot,
	xtick=\empty,
	ytick=\empty,
	ztick=\empty,
	ymin=-1.2,ymax=1.2,
	xmin=-1.2,xmax=1.2,
	zmin=-1.2, zmax=1.2,
	every axis x label/.style={at={(axis cs:\pgfkeysvalueof{/pgfplots/xmax},0,0)},xshift=-3pt,yshift=-3pt},
	xlabel={\scriptsize $x$},
	every axis y label/.style={at={(axis cs:0,\pgfkeysvalueof{/pgfplots/ymax},0)},xshift=5pt,yshift=-2pt},
	ylabel={\scriptsize $y$},
	every axis z label/.style={at={(axis cs:0,0,\pgfkeysvalueof{/pgfplots/zmax})},xshift=0pt,yshift=4pt},
	zlabel={\scriptsize $z$},
        colormap/cool
      ]

      \addplot3[domain=-1:1,smooth,y domain=-1:1,mesh,samples=20,samples y=25,very thin,z buffer=sort] ({x},{y},{x^2-y^2});
    \end{axis}
  \end{tikzpicture}
\end{image}
and equation:
\[
z=\pm\frac{x^2}{a^2}\mp\frac{y^2}{b^2}
\]
To understand this surface better consider the sections when:
\begin{itemize}
  \item $x=d$, in this case we now have $z = \pm\frac{d^2}{a^2} \mp
    \frac{y^2}{b^2}$, a parabola.
  \item $y=d$, in this case we now have $z = \pm\frac{x^2}{a^2} \mp
    \frac{d^2}{b^2}$, a parabola that opens the \textit{opposite}
    direction as the previous one.
  \item $z=d$, in this case we now have $d = \pm\frac{x^2}{a^2} \mp
    \frac{y^2}{b^2}$, a hyperbola.
\end{itemize}
\begin{image}
  \begin{tikzpicture}
    \begin{axis}%
      [width=175pt,tick label style={font=\scriptsize},axis on top,
	axis lines=center,
	view={155}{30},
	name=myplot,
	xtick=\empty,
	ytick=\empty,
	ztick=\empty,
	ymin=-1.2,ymax=1.2,
	xmin=-1.2,xmax=1.2,
	zmin=-1.2, zmax=1.2,
	every axis x label/.style={at={(axis cs:\pgfkeysvalueof{/pgfplots/xmax},0,0)},xshift=-3pt,yshift=-3pt},
	xlabel={\scriptsize $x$},
	every axis y label/.style={at={(axis cs:0,\pgfkeysvalueof{/pgfplots/ymax},0)},xshift=5pt,yshift=-2pt},
	ylabel={\scriptsize $y$},
	every axis z label/.style={at={(axis cs:0,0,\pgfkeysvalueof{/pgfplots/zmax})},xshift=0pt,yshift=4pt},
	zlabel={\scriptsize $z$},colormap/cool
      ]
      
      \addplot3[domain=-1:1,smooth,y domain=-1:1,mesh,samples=20,samples y=25,very thin,z buffer=sort] ({x},{y},{x^2-y^2});

      \addplot3[domain=-1:-.4,dashed,ultra thick,smooth,samples y=0,black!60,%surf,%fill=white,
        samples=30,] ({0},{x},{-x^2});
      
      \addplot3[domain=-1:1,ultra thick,smooth,samples y=0,black,%surf,%fill=white,
        samples=30,] ({x},{0},{x^2});
      
      \addplot3[domain=-.4:1,ultra thick,smooth,samples y=0,black,%surf,%fill=white,
        samples=30,] ({0},{x},{-x^2});
      
      
      \draw (axis cs:1.,0,1) node (A2) {};
      \draw (axis cs:0,1,-1) node (B2) {};
      %\draw (axis cs:0,1.5,0.3) node (C2) {};
      
    \end{axis}
    
    \draw (0,3) node [align=center](A1) {\scriptsize In plane\\[-4pt] \scriptsize $y=0$};
    \draw (0,1.1) node [align=center](B1) {\scriptsize In plane\\[-4pt] \scriptsize $x=0$};
    %
    \draw [->](A1)--(A2.center);
    \draw [->](B1)--(B2.center);
    %\draw [->](C1)--(C2.center);
  \end{tikzpicture}
\end{image}
We'll give an additional graph to show the hyperbolas:
\begin{image}
  \begin{tikzpicture}
    \begin{axis}%
      [width=175pt,tick label style={font=\scriptsize},axis on top,
        axis lines=center,
        view={155}{30},
        name=myplot,
        xtick=\empty,
        ytick=\empty,
        ztick=\empty,
        ymin=-1.2,ymax=1.2,
        xmin=-1.2,xmax=1.2,
        zmin=-1.2, zmax=1.2,
        every axis x label/.style={at={(axis cs:\pgfkeysvalueof{/pgfplots/xmax},0,0)},xshift=-3pt,yshift=-3pt},
        xlabel={\scriptsize $x$},
        every axis y label/.style={at={(axis cs:0,\pgfkeysvalueof{/pgfplots/ymax},0)},xshift=5pt,yshift=-2pt},
        ylabel={\scriptsize $y$},
        every axis z label/.style={at={(axis cs:0,0,\pgfkeysvalueof{/pgfplots/zmax})},xshift=0pt,yshift=4pt},
        zlabel={\scriptsize $z$},
        colormap/cool
      ]
      
  \addplot3[domain=-1:1,y domain=-1:1,mesh,samples=20,samples y=25,very thin,z buffer=sort] ({x},{y},{x^2-y^2});
  
  \addplot3[domain=-10:60,ultra thick,smooth,samples y=0,black,%surf,%fill=white,
    samples=30,] ({.5*sec(x)},{.5*tan(x)},{.25});
  \addplot3[domain=-35:-10,ultra thick,smooth,samples y=0,black!60,%surf,%fill=white,
    samples=30,] ({.5*sec(x)},{.5*tan(x)},{.25});
  \addplot3[domain=-60:-35,ultra thick,smooth,samples y=0,black,%surf,%fill=white,
    samples=30,] ({.5*sec(x)},{.5*tan(x)},{.25});
  
  \addplot3[domain=-60:60,ultra thick,smooth,samples y=0,black,%surf,%fill=white,
    samples=30,] ({-.5*sec(x)},{.5*tan(x)},{.25});
  
  
  \addplot3[domain=-60:60,ultra thick,smooth,samples y=0,black,%surf,%fill=white,
    samples=30,] ({.5*tan(x)},{.5*sec(x)},{-.25});
  \addplot3[domain=40:60,ultra thick,smooth,samples y=0,black,%surf,%fill=white,
    samples=30,] ({.5*tan(x)},{-.5*sec(x)},{-.25});
  \addplot3[domain=-60:40,dashed,thick,smooth,samples y=0,black!60,%surf,%fill=white,
    samples=30,] ({.5*tan(x)},{-.5*sec(x)},{-.25});

  \draw (axis cs:.5,0,.25) node (A2) {};
  \draw (axis cs:-.5,0,.25) node (A3) {};
  \draw (axis cs:.866,1,-.25) node (B2) {};
  \draw (axis cs:.866,-1,-.25) node (B3) {};
  %\draw (axis cs:0,1.5,0.3) node (C2) {};
\end{axis}
\draw (0,3) node [align=center](A1) {\scriptsize In plane\\[-4pt] \scriptsize $z=d$\\[-4pt] \scriptsize $(d>0)$};
\draw (0,1.1) node [align=center](B1) {\scriptsize In plane\\[-4pt] \scriptsize $z=d$\\[-4pt] \scriptsize $(d<0)$};
%\draw (4.6,2) node [align=center](C1) {\scriptsize in plane\\[-4pt] \scriptsize $z=0$};

\draw [->](A1)--(A2.center);
\draw [->](A1)--(A3.center);
\draw [->](B1)--(B2.center);
\draw [->](B1)--(B3.center);
%\draw [->](C1)--(C2.center);
  \end{tikzpicture}
\end{image}
\[
\begin{array}{cc}
\textbf{Plane}  & \textbf{Section} \\ \hline
x=d & \text{Parabola}\\
y=d & \text{Parabola}\\
z=d & \text{Hyperbola}
\end{array}
\]

\section{Identifying quadric surfaces}

In this course, we will be looking at quadric surfaces of the form: 
\begin{align*}
  z = &\left(\frac{1}{2}\right)F^{(2,0)}(\vec{c})(x-c_1)^2\\
  &+ \left(\frac{1}{2}\right)F^{(0,2)}(\vec{c})(y-c_2)^2 \\
  &+ F^{(1,1)}(\vec{c}) (x-c_1)(y-c_2)\\
  &+ F(\vec{c}).
\end{align*}
and trying to identify them as either a elliptic paraboloid, or a
hyperbolic paraboloid. In what follows, let
\[
\sign(x) =
\begin{cases}
  1  &\text{if $x>0$,}\\
  -1 &\text{if $x<0$,}\\
  0  &\text{if $x=0$.}
\end{cases}
\]
This will aid in our analysis of the quadric surfaces.

\subsection{The pure partials have opposite signs}
If
\[
\sign(F^{(2,0)}(\vec{c})) = -\sign(F^{(0,2)}(\vec{c})),
\]
then we can examine the following sections:
\[
y= c_2 \quad\text{and}\quad x = c_1
\]
If $y=c_2$ then the surface
\begin{align*}
  z = &\left(\frac{1}{2}\right)F^{(2,0)}(\vec{c})(x-c_1)^2\\
  &+ \left(\frac{1}{2}\right)F^{(0,2)}(\vec{c})(y-c_2)^2 \\
  &+ F^{(1,1)}(\vec{c}) (x-c_1)(y-c_2)\\
  &+ F(\vec{c}).
\end{align*}
becomes
\[
z = \left(\frac{1}{2}\right)F^{(2,0)}(\vec{c})(x-c_1)^2 + F(\vec{c}),
\]
and this is a parabola that opens in the $z$-direction of the sign of
$F^{(2,0)}(\vec{c})$.

If $x=c_1$ then the surface becomes
\[
z = \left(\frac{1}{2}\right)F^{(0,2)}(\vec{c})(y-c_2)^2 + F(\vec{c}),
\]
and this is a parabola that opens in the $z$-direction of the sign of
$F^{(0,2)}(\vec{c})$. Since 
\[
\sign(F^{(2,0)}(\vec{c})) = -\sign(F^{(0,2)}(\vec{c})),
\]
we see that when the pure partials have opposite signs, then the
quadric surface is a hyperbolic paraboloid.

\subsection{The pure partials have the same sign}
If 
\[
\sign(F^{(2,0)}(\vec{c})) = \sign(F^{(0,2)}(\vec{c})),
\]
then we start by examining the section
\[
y= c_2
\]
If $y=c_2$ then the surface
\begin{align*}
  z = &\left(\frac{1}{2}\right)F^{(2,0)}(\vec{c})(x-c_1)^2\\
  &+ \left(\frac{1}{2}\right)F^{(0,2)}(\vec{c})(y-c_2)^2 \\
  &+ F^{(1,1)}(\vec{c}) (x-c_1)(y-c_2)\\
  &+ F(\vec{c}).
\end{align*}
becomes
\[
z = F^{(2,0)}(\vec{c})(x-c_1)^2 + F(\vec{c}).
\]
and this is a parabola that opens in the $z$-direction of the sign of
$F^{(2,0)}(\vec{c})$.

When can we find another parabola on this surface that opens in the
opposite direction? Well, let's examine the section:
\[
y -c_2 = K (x-c_1)
\]
Substituting this into the surface above, we find
\begin{align*}
  z = &\left(\frac{1}{2}\right)F^{(2,0)}(\vec{c})(x-c_1)^2\\
  &+ K^2 \left(\frac{1}{2}\right)F^{(0,2)}(\vec{c})(x-c_1)^2\\
  &+ K F^{(1,1)}(\vec{c}) (x-c_1)^2 \\
  &+ F(\vec{c}).
\end{align*}
Factoring and rearranging, we find 
\begin{align*}
&z = \\
&(x-c_1)^2 \left(
K^2 \left(\frac{1}{2}\right)F^{(0,2)}(\vec{c}) + K F^{(1,1)}(\vec{c}) + \left(\frac{1}{2}\right)F^{(2,0)}(\vec{c})
\right)+ F(\vec{c}).
\end{align*}
and this is a parabola that opens in the $z$-direction of the sign of
$F^{(2,0)}(\vec{c})$ when
\[
K^2 \left(\frac{1}{2}\right)F^{(0,2)}(\vec{c}) + K F^{(1,1)}(\vec{c}) + \left(\frac{1}{2}\right)F^{(2,0)}(\vec{c})
\]
is has the same sign as $F^{(2,0)}(\vec{c})$ and it opens in the
opposite direction, when it has the opposite sign as
$F^{(2,0)}(\vec{c})$.  We can find a $K$ that produces this opposite
sign when the quadratic equation (in the variable $K$)
\[
K^2 \left(\frac{1}{2}\right)F^{(0,2)}(\vec{c}) + K F^{(1,1)}(\vec{c}) + \left(\frac{1}{2}\right)F^{(2,0)}(\vec{c}) = 0
\]
has two real solutions. Let's investigate using the quadratic formula:
\[
K = \frac{- F^{(1,1)}\pm(\vec{c})\sqrt{F^{(1,1)}(\vec{c})^2-F^{(2,0)}(\vec{c})F^{(0,2)}(\vec{c})}}{F^{(0,2)}(\vec{c})}
\]
We see that there are two real solutions for $K$ when
\[
F^{(1,1)}(\vec{c})^2-F^{(2,0)}(\vec{c})F^{(0,2)}(\vec{c})>0
\]
or equivalently when
\[
F^{(2,0)}(\vec{c})F^{(0,2)}(\vec{c})-F^{(1,1)}(\vec{c})^2<0.
\]
So, we see that when the pure partials have the same sign, the quadric
surface is a hyperbolic paraboloid when
\[
F^{(2,0)}(\vec{c})F^{(0,2)}(\vec{c})-F^{(1,1)}(\vec{c})^2<0,
\]
and an elliptic paraboloid when 
\[
F^{(2,0)}(\vec{c})F^{(0,2)}(\vec{c})-F^{(1,1)}(\vec{c})^2>0.
\]

\section{The second derivative test}

Given a function $F:\R^2\to \R$, and a point $\vec{c}$ where
\[
F^{(1,0)}(\vec{c}) = 0 = F^{(0,1)}(\vec{c}) 
\]
Our work above allows us to identify what a surface looks like locally.

\begin{theorem}[Second derivative test]
  Given a function $F:\R^2\to \R$, and a point $\vec{c}$ where
  \[
  F^{(1,0)}(\vec{c}) = 0 = F^{(0,1)}(\vec{c}) 
  \]
  set
  \[
  D = F^{(2,0)}(\vec{c})F^{(0,2)}(\vec{c})-F^{(1,1)}(\vec{c})^2.
  \]
  \begin{itemize}
  \item If $D>0$ then $F$ locally looks like an elliptic paraboloid.
  \item	If $D<0$, then $F$ locally looks like a hyperbolic paraboloid.
  \item If $D=0$, the test is inconclusive.
  \end{itemize}
\end{theorem}

Try your hand at identifying local behavior of a surface.

\begin{question}
  Consider $F(x,y) = x y e^{-xy}$. Does this surface locally look like
  an elliptic paraboloid or a hyperbolic paraboloid at the point
  $(0,0)$?
  \begin{prompt}
    \begin{multipleChoice}
    \choice{An elliptic paraboloid.}
    \choice[correct]{A hyperbolic paraboloid.}
    \choice{We cannot tell.}
    \end{multipleChoice}
  \end{prompt}
\end{question}


\begin{question}
  Consider $F(x,y) = e^{x^2+y^2}$. Does this surface locally look like
  an elliptic paraboloid or a hyperbolic paraboloid at the point
  $(0,0)$?
  \begin{prompt}
    \begin{multipleChoice}
    \choice[correct]{An elliptic paraboloid.}
    \choice{A hyperbolic paraboloid.}
    \choice{We cannot tell.}
    \end{multipleChoice}
  \end{prompt}
\end{question}


\begin{question}
  Consider $F(x,y) = \sqrt{x^2+y^2+2x-4y}$. Does this surface locally look like
  an elliptic paraboloid or a hyperbolic paraboloid at the point
  $(-1,2)$?
  \begin{prompt}
    \begin{multipleChoice}
      \choice{An elliptic paraboloid.}
      \choice[correct]{A hyperbolic paraboloid.}
      \choice{We cannot tell.}
    \end{multipleChoice}
  \end{prompt}
\end{question}








\end{document}

