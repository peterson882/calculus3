\documentclass{ximera}

%\usepackage{todonotes}

\newcommand{\todo}{}


\graphicspath{
{./}
{../functionsOfSeveralVariables/}
{../normalVectors/}
{../lagrangeMultipliers/}
}


\usepackage{tkz-euclide}
\tikzset{>=stealth} %% cool arrow head
\tikzset{shorten <>/.style={ shorten >=#1, shorten <=#1 } } %% allows shorter vectors

\usetikzlibrary{backgrounds} %% for boxes around graphs
\usetikzlibrary{shapes,positioning}  %% Clouds and stars
\usetikzlibrary{matrix} %% for matrix
\usepgfplotslibrary{polar} %% for polar plots
\usetkzobj{all}
\usepackage[makeroom]{cancel} %% for strike outs
%\usepackage{mathtools} %% for pretty underbrace % Breaks Ximera
\usepackage{multicol}





\usepackage{array}
\setlength{\extrarowheight}{+.1cm}   
\newdimen\digitwidth
\settowidth\digitwidth{9}
\def\divrule#1#2{
\noalign{\moveright#1\digitwidth
\vbox{\hrule width#2\digitwidth}}}





\newcommand{\RR}{\mathbb R}
\newcommand{\R}{\mathbb R}
\newcommand{\N}{\mathbb N}
\newcommand{\Z}{\mathbb Z}

\newcommand{\sage}{\textsf{SageMath}}


%\renewcommand{\d}{\,d\!}
\renewcommand{\d}{\mathop{}\!d}
\newcommand{\dd}[2][]{\frac{\d #1}{\d #2}}
\newcommand{\pp}[2][]{\frac{\partial #1}{\partial #2}}
\renewcommand{\l}{\ell}
\newcommand{\ddx}{\frac{d}{\d x}}

\newcommand{\zeroOverZero}{\ensuremath{\boldsymbol{\tfrac{0}{0}}}}
\newcommand{\inftyOverInfty}{\ensuremath{\boldsymbol{\tfrac{\infty}{\infty}}}}
\newcommand{\zeroOverInfty}{\ensuremath{\boldsymbol{\tfrac{0}{\infty}}}}
\newcommand{\zeroTimesInfty}{\ensuremath{\small\boldsymbol{0\cdot \infty}}}
\newcommand{\inftyMinusInfty}{\ensuremath{\small\boldsymbol{\infty - \infty}}}
\newcommand{\oneToInfty}{\ensuremath{\boldsymbol{1^\infty}}}
\newcommand{\zeroToZero}{\ensuremath{\boldsymbol{0^0}}}
\newcommand{\inftyToZero}{\ensuremath{\boldsymbol{\infty^0}}}



\newcommand{\numOverZero}{\ensuremath{\boldsymbol{\tfrac{\#}{0}}}}
\newcommand{\dfn}{\textbf}
%\newcommand{\unit}{\,\mathrm}
\newcommand{\unit}{\mathop{}\!\mathrm}
\newcommand{\eval}[1]{\bigg[ #1 \bigg]}
\newcommand{\seq}[1]{\left( #1 \right)}
\renewcommand{\epsilon}{\varepsilon}
\renewcommand{\iff}{\Leftrightarrow}

\DeclareMathOperator{\arccot}{arccot}
\DeclareMathOperator{\arcsec}{arcsec}
\DeclareMathOperator{\arccsc}{arccsc}
\DeclareMathOperator{\si}{Si}
\DeclareMathOperator{\proj}{\vec{proj}}
\DeclareMathOperator{\scal}{scal}
\DeclareMathOperator{\sign}{sign}


%% \newcommand{\tightoverset}[2]{% for arrow vec
%%   \mathop{#2}\limits^{\vbox to -.5ex{\kern-0.75ex\hbox{$#1$}\vss}}}
\newcommand{\arrowvec}{\overrightarrow}
%\renewcommand{\vec}[1]{\arrowvec{\mathbf{#1}}}
\renewcommand{\vec}{\mathbf}
\newcommand{\veci}{{\boldsymbol{\hat{\imath}}}}
\newcommand{\vecj}{{\boldsymbol{\hat{\jmath}}}}
\newcommand{\veck}{{\boldsymbol{\hat{k}}}}
\newcommand{\vecl}{\boldsymbol{\l}}
\newcommand{\utan}{\mathbf{\hat{t}}}
\newcommand{\unormal}{\mathbf{\hat{n}}}
\newcommand{\ubinormal}{\mathbf{\hat{b}}}

\newcommand{\dotp}{\bullet}
\newcommand{\cross}{\boldsymbol\times}
\newcommand{\grad}{\boldsymbol\nabla}
\newcommand{\divergence}{\grad\dotp}
\newcommand{\curl}{\grad\cross}
%\DeclareMathOperator{\divergence}{divergence}
%\DeclareMathOperator{\curl}[1]{\grad\cross #1}
\newcommand{\lto}{\mathop{\longrightarrow\,}\limits}


\colorlet{textColor}{black} 
\colorlet{background}{white}
\colorlet{penColor}{blue!50!black} % Color of a curve in a plot
\colorlet{penColor2}{red!50!black}% Color of a curve in a plot
\colorlet{penColor3}{red!50!blue} % Color of a curve in a plot
\colorlet{penColor4}{green!50!black} % Color of a curve in a plot
\colorlet{penColor5}{orange!80!black} % Color of a curve in a plot
\colorlet{fill1}{penColor!20} % Color of fill in a plot
\colorlet{fill2}{penColor2!20} % Color of fill in a plot
\colorlet{fillp}{fill1} % Color of positive area
\colorlet{filln}{penColor2!20} % Color of negative area
\colorlet{fill3}{penColor3!20} % Fill
\colorlet{fill4}{penColor4!20} % Fill
\colorlet{fill5}{penColor5!20} % Fill
\colorlet{gridColor}{gray!50} % Color of grid in a plot

\newcommand{\surfaceColor}{violet}
\newcommand{\surfaceColorTwo}{redyellow}
\newcommand{\sliceColor}{greenyellow}




\pgfmathdeclarefunction{gauss}{2}{% gives gaussian
  \pgfmathparse{1/(#2*sqrt(2*pi))*exp(-((x-#1)^2)/(2*#2^2))}%
}


%%%%%%%%%%%%%
%% Vectors
%%%%%%%%%%%%%

%% Simple horiz vectors
\renewcommand{\vector}[1]{\left\langle #1\right\rangle}


%% %% Complex Horiz Vectors with angle brackets
%% \makeatletter
%% \renewcommand{\vector}[2][ , ]{\left\langle%
%%   \def\nextitem{\def\nextitem{#1}}%
%%   \@for \el:=#2\do{\nextitem\el}\right\rangle%
%% }
%% \makeatother

%% %% Vertical Vectors
%% \def\vector#1{\begin{bmatrix}\vecListA#1,,\end{bmatrix}}
%% \def\vecListA#1,{\if,#1,\else #1\cr \expandafter \vecListA \fi}

%%%%%%%%%%%%%
%% End of vectors
%%%%%%%%%%%%%

%\newcommand{\fullwidth}{}
%\newcommand{\normalwidth}{}



%% makes a snazzy t-chart for evaluating functions
%\newenvironment{tchart}{\rowcolors{2}{}{background!90!textColor}\array}{\endarray}

%%This is to help with formatting on future title pages.
\newenvironment{sectionOutcomes}{}{} 



%% Flowchart stuff
%\tikzstyle{startstop} = [rectangle, rounded corners, minimum width=3cm, minimum height=1cm,text centered, draw=black]
%\tikzstyle{question} = [rectangle, minimum width=3cm, minimum height=1cm, text centered, draw=black]
%\tikzstyle{decision} = [trapezium, trapezium left angle=70, trapezium right angle=110, minimum width=3cm, minimum height=1cm, text centered, draw=black]
%\tikzstyle{question} = [rectangle, rounded corners, minimum width=3cm, minimum height=1cm,text centered, draw=black]
%\tikzstyle{process} = [rectangle, minimum width=3cm, minimum height=1cm, text centered, draw=black]
%\tikzstyle{decision} = [trapezium, trapezium left angle=70, trapezium right angle=110, minimum width=3cm, minimum height=1cm, text centered, draw=black]



\title[Dig-In:]{Continuity}

\begin{document}
\begin{abstract}
We investigate what continuity means for functions of several variables.
\end{abstract}
\maketitle

%% HAVE STUDENTS FILL IN DEFINITION OF LIMITS IN R^2 and R^3 from
%% general definition.


This section investigates what it means for functions
\[
F:\R^n\to R
\]
to be ``continuous.''

We begin with a series of definitions. We are used to ``open
intervals'' such as $(1,3)$, which represents the set of all $x$ such
that $1<x<3$, and ``closed intervals'' such as $[1,3]$, which
represents the set of all $x$ such that $1\leq x\leq 3$. We need
analogous definitions for open and closed sets in the $(x,y)$-plane.

%% \definition{def:open}{Open Disk, Boundary and Interior Points, Open and Closed Sets, Bounded Sets}
%% {An \textbf{open disk} $B$ in $\mathbb{R}^2$ centered at $(x_0,y_0)$ with radius $r$ is the set of all points $(x,y)$ such that $\ds\sqrt{(x-x_0)^2+(y-y_0)^2} < r$. \\

%% Let $S$ be a set of points in $\mathbb{R}^2$. A point $P$ in $\mathbb{R}^2$ is a \textbf{boundary point} of $S$  if all open disks centered at $P$ contain both points in $S$ and points not in $S$.\\

%% A point $P$ in $S$ is an \textbf{interior point} of $S$ if there is an open disk centered at $P$ that contains only points in $S$. \\

%% A set $S$ is \textbf{open} if every point in $S$ is an interior point.\\

%% A set $S$ is \textbf{closed} if it contains all of its boundary points.\\

%% A set $S$ is \textbf{bounded} if there is an $M>0$ such that the open disk, centered at the origin, with radius $M$ contains $S$. A set that is not bounded is \textbf{unbounded}.
%% \index{open}\index{closed}\index{open disk}\index{closed disk}\index{boundary point}\index{interior point}\index{bounded set}\index{unbounded set}
%% }

%% Figure \ref{fig:multilimit_intro} shows several sets in the $x$-$y$ plane. In each set, point $P_1$ lies on the boundary of the set as all open disks centered there contain both points in, and not in, the set. In contrast, point $P_2$ is an interior point for there is an open disk centered there that lies entirely within the set.
%% \mtable{.6}{Illustrating open and closed sets in the $x$-$y$ plane.}{fig:multilimit_intro}{%
%% \begin{tabular}{c}
%% \myincludegraphics{figures/figmultilimit_introa}\\
%% (a)\\[10pt]
%% \myincludegraphics{figures/figmultilimit_introb}\\
%% (b)\\[10pt]
%% \myincludegraphics{figures/figmultilimit_introc}\\
%% (c)\\[10pt]
%% \end{tabular}
%% }

%% The set depicted in Figure \ref{fig:multilimit_intro}(a) is a closed set as it contains all of its boundary points. The set in (b) is open, for all of its points are interior points (or, equivalently, it does not contain any of its boundary points). The set in (c) is neither open nor closed as it contains just some of its boundary points.\\
%% \clearpage

%% \example{ex_multilimit1}{Determining open/closed, bounded/unbounded}{
%% Determine if the domain of the function $f(x,y)=\sqrt{1-\frac{x^2}9-\frac{y^2}4}$ is open, closed, or neither, and if it is bounded.}
%% {This domain of this function was found in Example \ref{ex_multi2} to be $D = \{(x,y)\ |\ \frac{x^2}9+\frac{y^2}4\leq 1\}$, the region \textit{bounded} by the ellipse $\frac{x^2}9+\frac{y^2}4=1$. Since the region includes the boundary (indicated by the use of ``$\leq$''), the set contains all of its boundary points and hence is closed. The region is bounded as a disk of radius 4, centered at the origin, contains $D$.
%% }\\

%% \example{ex_multilimit2}{Determining open/closed, bounded/unbounded}{
%% Determine if the domain of $f(x,y) = \frac1{x-y}$ is open, closed, or neither.}
%% {As we cannot divide by 0, we find the domain to be $D = \{(x,y)\ |\ x-y\neq 0\}$. In other words, the domain is the set of all points $(x,y)$ \emph{not} on the line $y=x$. 

%% \mfigure{.75}{Sketching the domain of the function in Example \ref{ex_multilimit2}.}{fig:multilimit2}{figures/figmultilimit2}
%% The domain is sketched in Figure \ref{fig:multilimit2}. Note how we can draw an open disk around any point in the domain that lies entirely inside the domain, and also note how the only boundary points of the domain are the points on the line $y=x$. We conclude the domain is an open set. The set is unbounded.
%% }\\


%% \section{Limits}

%% We'll start with limits.

%% \begin{definition}
%%  Suppose that $F:\R^n\to\R$, intuitively,
%%   \begin{center}
%%     the \dfn{limit} of $F$ as $\vec{x}$ approaches $\vec{a}$ is $L$,
%%   \end{center}
%%   written
%%   \[
%%   \lim_{\vec{x}\to \vec{a}} F(\vec{x}) = L,
%%   \]
%%   if the value of $F(\vec{x})$ can be made as close as one wishes to $L$ for
%%   all $\vec{x}$ sufficiently close, but not equal to, $\vec{a}$.
%% \end{definition}

%% \begin{question}
%%   Suppose that $F:\R^2\to\R$, $\vec{x} = \vector{x,y}$, and $\vec{a} =
%%   \vec{a,b}$. What do we write in this case for $\lim_{\vec{x}\to
%%     \vec{a}} F(\vec{x}) = L$?
%%   \begin{prompt}
%%     \[
%%     \lim_{\vector{\answer{x},\answer{y}}\to \vector{a,b}} F(\vector{x,y}) = L
%%     \]
%%   \end{prompt}
%%   \begin{question}
%%     Suppose that $F:\R^3\to\R$, $\vec{x} = \vector{x,y}$, and $\vec{a} =
%%     \vec{a,b}$. What do we write in this case for $\lim_{\vec{x}\to
%%       \vec{a}} F(\vec{x}) = L$?
%%     \begin{prompt}
%%       \[
%%       \lim_{\vector{\answer{x},\answer{y}}\to \vector{a,b}} F(\vector{x,y}) = L
%%       \]
%%     \end{prompt}
%%   \end{question}
%% \end{question}

\end{document}
